% \iffalse
%% File: bpchem.dtx Copyright (C) 2001-2017 
%%                 Bjoern Pedersen <Bjoern.Pedersen@ch.tum.de>
%% This file may be distributed and used freely under 
%% the LaTeX Project Public License
%%
%% 
%
%<*dtx>
          \ProvidesFile{bpchem.dtx}[2017/08/23 v1.1 Chemical input helper]
%</dtx>
%<bpchem>\NeedsTeXFormat{LaTeX2e}
%<bpchem>\ProvidesPackage{bpchem}[2004/08/21 v1.1 Chemical input helper]
%<bpchem>\RequirePackage{xspace}
%<driver>\ProvidesFile{bpchem.drv}
% \fi
%\iffalse
%<*driver>
\documentclass{ltxdoc}
\usepackage[latin1]{inputenc}
\usepackage[T1]{fontenc}
\usepackage{bpchem}
\RecordChanges
\begin{document}
\DocInput{bpchem.dtx}
\end{document}
%</driver>
%\fi
% \changes{v1.1}{2017/08/23}{
%  Resolved conflict with recent \LaTeX{} releases (thanks to Ulrike Fischer 
%  and Martin Sievers)
% }
%
% \changes{v1.05}{2004/11/06}{
%  changed email
% }
% \changes{v1.03}{2002/11/01}{
%  re-added lost email-address
% }
% \GetFileInfo{bpchem.dtx}
% \title{The \textsf{bpchem} package\thanks{This file
%        has version number \fileversion, last
%        revised \filedate.}}
% \author{Bj{\o}rn Pedersen\\ \texttt{Bjoern.Pedersen@frm2.tum.de}}
% \date{\filedate}
% \maketitle
% \section{Introduction}
% This package has been written to alleviate the task of writing
% publications containing lots of chemistry. It provides methods for
% typesetting chemical names, sum formulae and isotopes. It provides the
% possibility to break very long names even over several lines.
%
% This package also provides a way to automatically enumerate your
% chemical compounds, allowing for one-level subgrouping.
%
% What this package does not provide: Methods to draw chemical
% compounds. Although there exist some packages,which where designed for
% this purpose (e.g. xymtex, PPChTex) they are quite limited once you
% get to complex organic, or metal organic compounds. I recommend using
% an external drawing program, possibly in conjunction with psfrag, in
% these cases.
%
%\section{Package options}
% Currently this package supports only one option:
%
% \verb|cbgreek|
%
% this option causes the definitions of some macros to be changed to
% use the cbgreek fonts. As they are not available on all  systems,
% and only in mf format, the default is to use the math fonts for
% greek symbols.  
%
%\section{User commands in this package}
%\subsection{Setting chemical sum formulae: BPChem{<chemical formula>}}
%\DescribeMacro{\BPChem}
%Within this macro you can use \verb|\_| and \verb|\^| for correct
%chemical sub- and superscripts.
%Example:
%\begin{verbatim}
%\BPChem{C\_2H\_5OH} or \BPChem{SO\_4\^{2-}}
%\end{verbatim}
%\begin{minipage}{10cm}
%\BPChem{C\_2H\_5OH} or \BPChem{SO\_4\^{2-}}
%\end{minipage}
%\subsection{Setting long chemical names: IUPAC{<formula or name>}}
%\DescribeMacro{\IUPAC}
%in addition to sub/superscripts as above, \verb|\-| is a hyphen which 
%allows further breakpoints,\verb!\|! is an (invisible) Multibreakpoint.
%
%This environment is especially useful for your long IUPAC-compound names.
%
%Example:
%\begin{verbatim}
 %\IUPAC{Tetra\|cyclo[2.2.2.1\^{1,4}]\-
%un\|decane-2\-dodecyl\-5\-(hepta\|decyl\|iso\|dodecyl\|thio\|ester)}
%\end{verbatim}
%\begin{minipage}{15em}
%\IUPAC{Tetra\|cyclo[2.2.2.1\^{1,4}]\-^^A
%un\|decane-2\-dodecyl\-5\-(hepta\|decyl\|iso\|dodecyl\|thio\|ester)}
%\end{minipage}
%
%\subsection{Enumerating and referencing chemical compounds: CNlabel\{<label>\}, CNlabelnoref\{<label>\}, CNref\{<label>\}}
%\DescribeMacro{\CNlabel}\DescribeMacro{\CNlabelnoref}\DescribeMacro{\CNref}
%\verb|CNlabel| defines  and use \verb|#1| (via \verb|ref|) as label for
%numbering of chemical compounds. If the label has not yet been
%defined, it is created, otherwise it is just referenced. if you just
%want to define the label, use \verb|\CNlabelnoref| instead.
%
%If you want to get just the reference, use \verb|\CNref|. This comes
%handy for figure captions or section titles, as you would get
%dissorder in the numbering due to the moving argument otherwise.
%
%The default style is: \verb|\textbf{\arabic{\counter}}|
%
%To change, use something like \begin{verbatim}
%\renewcommand{\theBPCno}{\textbf{\arabic{BPCno}}}
%\end{verbatim}

%Example: 
%\begin{verbatim}
%Alkohol \CNlabel{al} is converted to aldehyd \CNlabel{ad}. \CNref{al}
%can also be used otherwise, while \CNref{ad} cannot.
%\end{verbatim}
%\begin{minipage}{20em}
%Alkohol \CNlabel{al} is converted to aldehyd \CNlabel{ad}. \CNref{al}
%can also be used otherwise, while \CNref{ad} cannot.
%\end{minipage}
% 
%
%\subsection{Using sub-labels for classes of compounds: CNlabelsub\{<label>\}\{<sublabel>\},
%\\CNlabelsubnoref\{<label>\}\{<sublabel>\},\\CNrefsub\{<label>\}\{<sublabel>\}}
%\DescribeMacro{\CNlabelsub}\DescribeMacro{\CNlabelsubnoref}\DescribeMacro{\CNrefsub}
%These commands are the same as above, with additional sub identifier
%\verb|#2| added. If the primary identifier is not yet used, it will be
%created and can also be referenced via the normal commands.
%
%The default style is:\verb|\textbf{\arabic{BPCno}\alph{BPCnoa}}}|
%
%To change, use something like \begin{verbatim}
%\renewcommand{\theBPCnoa}{\textbf{\arabic{BPCno}\alph{BPCnoa}}}
%\end{verbatim}
%
%\begin{verbatim}
%To demonstrate the use of sublabels, methanol \CNlabelsub{alk}{a} and
%ethanol \CNlabelsub{alk}{b} are both natural products. The acohols \CNref{alk}
%can synthezied bio-chemically. \CNrebsub{alk}{a} is toxic, while
%\CNrefsub{alk}{b} is only mildly toxic.
%\end{verbatim}
%\begin{minipage}{20em}
%To demonstrate the use of sublabels, methanol \CNlabelsub{alk}{a} and
%ethanol \CNlabelsub{alk}{b} are both natural products. The alcohols \CNref{alk}
%can synthezied bio-chemically. \CNrefsub{alk}{a} is toxic, while
%\CNrefsub{alk}{b} is only mildly toxic.
%\end{minipage}
%
%\subsection{Shortcuts for common idioms in chemical literature}
%
%
%\begin{description}
%\item[\HNMR] \verb|\HNMR|
%\item[\CNMR] \verb|\CNMR|
%\item[\cis] \verb|\cis|
%\item[\trans] \verb|\trans|
%\item[\bpalpha] \verb|\bpalpha|
%\item[\bpbeta] \verb|\bpbeta|
%\item[\bpDelta] \verb|\bpdelta|
%\item[\IUPAC{\hapto{<nummer>}}] \verb|\hapto{<number>}|
%\end{description}
%\textbf{Note:} Some of these macros are influenced by the
%\verb|cbgreek| option! Use is only recommended with the
%\verb|\BPChem| and \verb|\IUPAC| commands. Some will not even work
%outside  those commands.
% 
%\section{Example}
%\begin{verbatim}
%\begin{minipage}[b]{15em}
%some normal text and math: $A*2=B$
%
%Test \BPChem{ C\_{2}H\_{4}\^{+}} 
%or using math in superscript  \BPChem{ C\_{2}H\_{4}\^{$+$}}
%
%\BPChem{Example\_{longer subscript}\^{superscript}}
%
%Isotope: \BPChem{\_{A}\^{B}X\^{C}\_{D}}
%
%\IUPAC{Tetra\|cyclo[2.2.2.1\^{1,4}]\-^^A
%un\|decane-2\-dodecyl-5-(heptadecyl\|iso\|dodecyl\|thio\|ester)}
%\end{minipage}
%\end{verbatim}
% and the resulting output:
%
%\begin{minipage}[b]{15em}
%some normal text and math: $A*2=B$, just to show it.
%
%Test \BPChem{ C\_{2}H\_{4}\^{+}} 
%or using math in superscript  \BPChem{ C\_{2}H\_{4}\^{$+$}}
%
%\BPChem{Example\_{longer subscript}\^{superscript}} And normal Text again
%
%Isotope: \BPChem{\_{A}\^{B}X\^{C}\_{D}}
%
%
%\IUPAC{Tetra\|cyclo[2.2.2.1\^{1,4}]\-^^A
%un\|decane-2\-dodecyl\-5\-(hepta\|decyl\|iso\|dodecyl\|thio\|ester)}
%\end{minipage}
%
%\StopEventually
%
%\section{The code}
%<*bpchem>^^J
% first comes some option setup
%    \begin{macrocode}
\newif\ifusecbgreek%
\usecbgreekfalse%
\DeclareOption{cbgreek}{\PackageInfo{bpchem}{cbgreek selected}\usecbgreektrue}
\ProcessOptions\relax
%    \end{macrocode}
%\begin{macro}{\textsubscript}
%% Define a  textsubscript corresponing to textsuperscript.
%% This is now also available as the package textsubscript by
%% D.Arsenau or as part of KOMA-Script2 by M. Kohm.
%% 
%% 
%    \begin{macrocode}
\providecommand*\textsubscript[1]{%
  \@textsubscript{\selectfont#1}}
\def\@textsubscript#1{%
  {\m@th\ensuremath{_{\mbox{\fontsize\sf@size\z@#1}}}}}
%    \end{macrocode}
%\end{macro}
%% a register to save the length to backspace
%% two registers needed to get back to correct
%% working position if one is longer than the other.
%    \begin{macrocode}
\newlength{\BPClensub}
\newlength{\BPClensuper}
\newlength{\BPCdelta}
%    \end{macrocode}
%% we are in subscript and maybe  the superscript was longer
%    \begin{macrocode}
\DeclareRobustCommand{\BPCadjustsub}{%
  \setlength\BPCdelta{\BPClensuper}\addtolength\BPCdelta{-\BPClensub}%
  \ifdim\BPCdelta>0pt{\kern\BPCdelta}\else\relax\fi%
  \setlength{\BPClensub}{0pt}% reset 
  \setlength{\BPClensuper}{0pt}% reset 
}%
%    \end{macrocode}
%% we are in superscript and maybe  the subscript was longer
%    \begin{macrocode}
\DeclareRobustCommand{\BPCadjustsuper}{%
  \setlength\BPCdelta{\BPClensub}\addtolength\BPCdelta{-\BPClensuper}%
  \ifdim\BPCdelta>0pt{\kern\BPCdelta}\else\relax\fi%
  \setlength{\BPClensub}{0pt}% reset 
  \setlength{\BPClensuper}{0pt}% reset 
}%
%    \end{macrocode}
%% make a subscript and remember length in BPClen
%    \begin{macrocode}
\DeclareRobustCommand{\BPCsub}[1]{%
  \ifmmode_{#1}\settowidth\BPClensub{_{#1}}%
  \else\textsubscript{#1}\settowidth\BPClensub{\textsubscript{#1}}\fi%
  \futurelet\next\lookforsuper%
}%
%    \end{macrocode}
%% make a superscript and remember length in BPClen
%% raise by 0.15 em, else e.g. + collides with subscript
%    \begin{macrocode}
\DeclareRobustCommand{\BPCsuper}[1]{%
  \ifmmode^{#1}\settowidth\BPClensuper{^{#1}}%
  \else\raisebox{0.15em}{\textsuperscript{#1}}%
  \settowidth\BPClensuper{\textsuperscript{#1}}\fi%
  \futurelet\next\lookforsub%
}%
%    \end{macrocode}
%% see if next token is BPCsuper,
%    \begin{macrocode}
\DeclareRobustCommand\lookforsuper{%
  \ifx\next\BPCsuper\let\next=\BPCsuperbs%
  \else\let\next=\BPCadjustsub\fi\next%
}%
%    \end{macrocode}
%% see if next token is BPCsub
%    \begin{macrocode}
\DeclareRobustCommand\lookforsub{%
  \ifx\next\BPCsub\let\next=\BPCsubbs%
  \else\let\next=\BPCadjustsuper\fi\next%
}%
%    \end{macrocode}
%%  backspace BPClen and make superscript
%%  eats the old \verb|\^|
%    \begin{macrocode}
\DeclareRobustCommand{\BPCsuperbs}[1]{\kern-\BPClensub\BPCsuper}%
%    \end{macrocode}
%% backspace and make subscript
%% eats the old \_
%    \begin{macrocode}
\DeclareRobustCommand{\BPCsubbs}[1]{\kern-\BPClensuper\BPCsub}%
%    \end{macrocode}
%% needed to get catcodes right
%    \begin{macrocode}
\DeclareRobustCommand{\DoBPChem}{}%
\def\DoBPChem#1{%
  #1\endgroup%
}%
\DeclareRobustCommand{\BPCSetupCat}{}
\def\BPCSetupCat{%
  %\catcode`^=\active%
  %\catcode`\_=\active%
  \BPCSetup%
}%
\DeclareRobustCommand{\BPCSetup}{}
\def\BPCSetup{%
 \let\_=\BPCsub%
 \let\^=\BPCsuper%
}%end BPCSetup
%    \end{macrocode}
%% setup for chemical formula
%    \begin{macrocode}
\DeclareRobustCommand\BPChem{%
  \begingroup% endgroup in DoBPChem
  \BPCSetupCat%
  \DoBPChem%
}
%    \end{macrocode}
%% these are taken from german.sty and allow more than one break
%% or breaks and hyphens in a word. Very useful for chemical names, 
%% as they tend to grow rather long. Two short versions are also defined
% \changes{v1.05}{2004/11/06}{
%  fixed whitespace bug in \cs{DoIUPAC}
% }
%    \begin{macrocode}
\DeclareRobustCommand{\allowhyphens}{\penalty\@M \hskip\z@skip}
\DeclareRobustCommand{\BreakHyph}{\penalty\@M -\allowhyphens}
\DeclareRobustCommand{\MultiBreak}%
  {\penalty\@M\discretionary{-}{}{\kern.03em}%
   \allowhyphens}
\let\MB=\MultiBreak \let\BH=\BreakHyph
\DeclareRobustCommand{\DoIUPAC}[1]{%
  #1\endgroup}
\def\Prep{%
  \let\-=\BreakHyph%
  \let\|=\MultiBreak%
  \DoIUPAC%
}
\DeclareRobustCommand*{\IUPAC}{%
  \begingroup\BPCSetup\ignorespaces%
  \Prep}%
%    \end{macrocode}
% \changes{v1.03}{2002/11/01}{%
%  fixed bug in \cs{IUPAC}, \texttt{\textbackslash$\vert$} is not getting 
%  translated to \cs{delimiter} during \cs{write} anymore
% }
%    \begin{macrocode}
\expandafter\DeclareRobustCommand\expandafter\|\expandafter{\|}
%    \end{macrocode}
% \changes{v1.04}{2002/12/14}{also \cs{-} was affected, fixed}
%% Trick by David Kastrup <David.Kastrup@t-online.de> to make
%% non-fragile. Otherwise \verb+\|+ would become \verb+\delimiter"026B30D+ in
%% e.g. the toc
% \changes{v1.1}{2017/08/23}{%
%   Recent \LaTeX{} releases make \cs{-} robust as 
%   well. This results in an infinite loop.
% }
%% Recent \LaTeX{} releases (starting with <2017-04-15>) define \cs{-} robust. 
%% This resulted in a infinite loop with older version of bpchem. We could use 
%% \cs{IncludeInRelease}, but it might be easier to just use the definition 
%% from the latest \LaTeX{} releases.
%    \begin{macrocode}
\DeclareRobustCommand{\-}{%
  \discretionary{%
    \char \ifnum\hyphenchar\font<\z@%
            \defaulthyphenchar%
          \else%
            \hyphenchar\font%
          \fi%
  }{}{}%
}%    \end{macrocode}
%% counters for numbering of chemical substances
%    \begin{macrocode}
\newcounter{BPCno}
\renewcommand{\theBPCno}{\textbf{\arabic{BPCno}}}
%
\newcounter{BPCnoa}[BPCno]
\renewcommand{\theBPCnoa}{\textbf{\arabic{BPCno}\alph{BPCnoa}}}
%    \end{macrocode}
%% helper functions to mark first definition
%    \begin{macrocode}
\newcommand{\newchems@b}[2]{
\expandafter\gdef\csname cna@#1#2\endcsname{#2}%
}
%    \end{macrocode}
%% reference a CNlabel (useful for section titles, captions etc.)
%    \begin{macrocode}
\DeclareRobustCommand*{\CNref}[1]{%
  \ref{cn:#1}%
}
%    \end{macrocode}
%% reference a CNlabel/sublabel
%    \begin{macrocode}
\DeclareRobustCommand*{\CNrefsub}[2]{%
  \ref{cn:#1#2}
%%\textbf{\csname cna@#1#2\endcsname}%
}
%    \end{macrocode}
%% label a substance and insert the number
% \changes{v1.06}{2004/11/25}{
%  fixed whitespace bug in CNlabel
% }
%    \begin{macrocode}
\DeclareRobustCommand*{\CNlabel}[1]{%
  \CNlabelnoref{#1}%
  \CNref{#1}%
}
%    \end{macrocode}
%    \begin{macrocode}
\DeclareRobustCommand*{\CNlabelnoref}[1]{%
  \expandafter\ifx\csname cnd@#1\endcsname\relax%
  {\refstepcounter{BPCno}\label{cn:#1}}%
  \expandafter\gdef\csname cnd@#1\endcsname{x}%
  \fi%
}
%    \end{macrocode}
%    \begin{macrocode}
\DeclareRobustCommand*{\CNlabelsub}[2]{%
  \CNlabelsubnoref{#1}{#2}%
  \CNrefsub{#1}{#2}%
}
\DeclareRobustCommand*{\CNlabelsubnoref}[2]{%
  \CNlabelnoref{#1}%
  \expandafter\ifx\csname cna@#1#2\endcsname\relax%
  {\refstepcounter{BPCnoa}\label{cn:#1#2}}%
  \expandafter\gdef\csname cna@#1#2\endcsname{x}%
%% \newchems@b{#1}{#2}%
%%   \write\@auxout{\string\newchems@b{#1}{#2}}%
  \fi%
}
%    \end{macrocode}
%% more helper mcors
% special symbols and macros for math-symbols without math-mode
% \changes{v1.03}{2002/11/01}{added cbgreek option for using upright
% greek and fixed namespace, old names will still work}
%    \begin{macrocode}
\DeclareRobustCommand{\HNMR}{\IUPAC{\^{1}H-NMR}: $\delta$\xspace}
\DeclareRobustCommand{\CNMR}{\IUPAC{\^{13}C-NMR}: $\delta$\xspace}
\DeclareRobustCommand{\cis}{\textit{cis}\xspace}
\DeclareRobustCommand{\trans}{\textit{trans}\xspace}
%\DeclareRobustCommand{\R}{\textit{R}}
%\DeclareRobustCommand{\S}{\textit{S}}
%%%%%%%%%%%%%%%%%%%%%
\ifusecbgreek% code with roman greek
  \PackageInfo{bpchem}{using upright greek fonts from cbgreek}
  % lgrenc.def
% ***********
%
% LGR Greek font encoding definitions
% ===================================
%
% :Copyright: © 2010, 2014 Günter Milde
% :Licence:   This work may be distributed and/or modified under the
%             conditions of the `LaTeX Project Public License`_, either
%             version 1.3 of this license or any later version.
%
% :Abstract:  The LGR font encoding is the de-facto standard for Greek
%             typesetting with LaTeX. This file provides a comprehensive set
%             of macros to typeset Greek with LGR encoded fonts. It works for
%             both, monotonic and polytonic Greek, independent of the `babel`
%             package.
%
% .. contents::
%
%
% Changelog
% ---------
%
% Development of this file started under the name "lgrxenc.def" as part of the
% lgrx_ bundle. With version 0.8, it moved to ``greek-fontenc/lgrx.def``.
%
% .. class:: borderless
%
% ====== ============  ========================================================
% 0.1    2010-03-31    initial attempt
% 0.2    2010-04-20    diacritics dropped in UPPERCASE.
% 0.3    2010-06-08    handle Titlecase and UPPERCASE extending
%                      ``\@uclclist``.
% 0.4    2010-06-15    more text symbols.
% 0.5    2010-08-27    support ``\textalpha`` ... ``\textOmega``.
% 0.5.1  2012-05-08    upload to CTAN.
% 0.6    2012-06-29    ``\TextGreek`` wrapper command,
% ..                   aliases for ``puenc.def`` compatibility.
% 0.6.1  2013-02-12    ``\@uclclist`` entries for `PU` aliases.
% 0.7    2013-05-13    documentation update, new accent macros names,
% ..                   ``\TextGreek`` renamed to ``\ensuregreek``.
% 0.8    2013-05-13    rename to lgrenc.def (superseding the babel_ version),
% ..                   move to greek-fontenc,
% ..                   drop the old (<0.7) internal accent macros names.
% 0.8.1  2013-05-22    auxiliary macro \update@uclc@with@greek,
% ..                   conservative naming: "text" prefix for archaic letters,
% ..                   "LGR@" prefix for internal commands,
% ..                   remove not required alias macros.
% 0.8.2  2013-05-23    fix name: ``\textpentehkaton`` -> ``\textpentehekaton``,
% ..                   fix ``\textperiodcentered``,
% ..                   add composite command for Ῥ (Rho with Dasia).
% 0.9    2013-07-16    move common definitions to ``greek-fontenc.def``,
% ..                   add composite commands for single quotation marks.
% 0.11.2 2014-09-04    remove duplicate code.
% 0.13   2015-08-04    Support for symbol variants.
% 0.13.1 2015-12-07    Fix `rho with dasia bug` (Linus Romer).
% 0.13.3 2019-07-09    Drop error font declaration (cf. `ltxbugs 4399`_).
% ====== ============  ========================================================
%
% Usage
% -----
%
% Load this file by calling fontenc_ with the ``LGR`` option
% or indirectly via the babel_, textalpha_ or alphabeta_ packages.
%
% Example:
%   Select font encodings `T1` (as default) and `LGR` (for Greek):
%
%     ``\usepackage[LGR,T1]{fontenc}``
%
%
% Implementation
% --------------
%
% Read this file only once
% ~~~~~~~~~~~~~~~~~~~~~~~~
% .. note:: The use of ``\@ifundefined`` has the side-effect that the tested
%    macro becomes defined and set to ``\relax`` (polluting the namespace and
%    interfering with tests via eTeX's ``\ifdefined``). However, in this case
%    no harm is done as the macro is defined by the identification_ below
%    anyway.
%
% ::

% read this file only once
\@ifundefined{ver@lgrenc.def}
  {}
  {\message{LGR font encoding definitions already loaded}
   \expandafter\endinput % "return"
  }


% Identification
% ~~~~~~~~~~~~~~
% ::

\ProvidesFile{lgrenc.def}
[2019/07/09 v0.13.3 LGR Greek font encoding definitions]
% Copyright © 2010 Günter Milde
% This file is part of the "greek-fontenc" package.
% It may be distributed and/or modified under the
% conditions of the "LaTeX Project Public License", either
% version 1.3 of this license or any later version.

% You can test the version date using
%
% .. code:: latex
%
%   \@ifl@ter {extension}{filename}{date}{YES}{NO}
%
% or using ltxcmds_' ``\@iffilelater``.
%
%
% Base setup
% ~~~~~~~~~~
%
% Declare the LGR font encoding and base substitutions::

\DeclareFontEncoding{LGR}{}{}
\DeclareFontSubstitution{LGR}{cmr}{m}{n}

% Text symbols
% ~~~~~~~~~~~~
%
% Greek Alphabet
% """"""""""""""
%
% Greek letters are accessible by the Latin transcription, but the mapping is
% specific to the LGR font encoding.
%
% The LICR macros provide a way to access the symbols independent of the
% specific font encoding, in any font encoding supporting Greek. An
% alternative Greek font encodings is LGI (ibycus_). Greek script is also
% supported by the Unicode-based font encodings PU (`PDF Unicode` used by
% hyperref_ for PDF metadata), TU, EU1 (XeTeX) and EU2 (LuaTeX). The textalpha_
% package makes these macros available independent of the current font
% encoding.
% ::

\DeclareTextSymbol{\textAlpha}{LGR}{65}
\DeclareTextSymbol{\textBeta}{LGR}{66}
\DeclareTextSymbol{\textGamma}{LGR}{71}
\DeclareTextSymbol{\textDelta}{LGR}{68}
\DeclareTextSymbol{\textEpsilon}{LGR}{69}
\DeclareTextSymbol{\textZeta}{LGR}{90}
\DeclareTextSymbol{\textEta}{LGR}{72}
\DeclareTextSymbol{\textTheta}{LGR}{74}
\DeclareTextSymbol{\textIota}{LGR}{73}
\DeclareTextSymbol{\textKappa}{LGR}{75}
\DeclareTextSymbol{\textLambda}{LGR}{76}
\DeclareTextSymbol{\textMu}{LGR}{77}
\DeclareTextSymbol{\textNu}{LGR}{78}
\DeclareTextSymbol{\textXi}{LGR}{88}
\DeclareTextSymbol{\textOmicron}{LGR}{79}
\DeclareTextSymbol{\textPi}{LGR}{80}
\DeclareTextSymbol{\textRho}{LGR}{82}
\DeclareTextSymbol{\textSigma}{LGR}{83}
\DeclareTextSymbol{\textTau}{LGR}{84}
\DeclareTextSymbol{\textUpsilon}{LGR}{85}
\DeclareTextSymbol{\textPhi}{LGR}{70}
\DeclareTextSymbol{\textChi}{LGR}{81}
\DeclareTextSymbol{\textPsi}{LGR}{89}
\DeclareTextSymbol{\textOmega}{LGR}{87}
%
\DeclareTextSymbol{\textalpha}{LGR}{97}
\DeclareTextSymbol{\textbeta}{LGR}{98}
\DeclareTextSymbol{\textgamma}{LGR}{103}
\DeclareTextSymbol{\textdelta}{LGR}{100}
\DeclareTextSymbol{\textepsilon}{LGR}{101}
\DeclareTextSymbol{\textzeta}{LGR}{122}
\DeclareTextSymbol{\texteta}{LGR}{104}
\DeclareTextSymbol{\texttheta}{LGR}{106}
\DeclareTextSymbol{\textiota}{LGR}{105}
\DeclareTextSymbol{\textkappa}{LGR}{107}
\DeclareTextSymbol{\textlambda}{LGR}{108}
\DeclareTextSymbol{\textmu}{LGR}{109}
\DeclareTextSymbol{\textnu}{LGR}{110}
\DeclareTextSymbol{\textxi}{LGR}{120}
\DeclareTextSymbol{\textomicron}{LGR}{111}
\DeclareTextSymbol{\textpi}{LGR}{112}
\DeclareTextSymbol{\textrho}{LGR}{114}
\DeclareTextCommand{\textsigma}{LGR}{s\noboundary} % σ
\DeclareTextSymbol{\textvarsigma}{LGR}{99}         % ς final sigma
\DeclareTextCommand{\textfinalsigma}{LGR}{\textvarsigma} % ς final sigma
\DeclareTextSymbol{\textautosigma}{LGR}{115}       % σ (ς at end of word)
\DeclareTextSymbol{\texttau}{LGR}{116}
\DeclareTextSymbol{\textupsilon}{LGR}{117}
\DeclareTextSymbol{\textphi}{LGR}{102}
\DeclareTextSymbol{\textchi}{LGR}{113}
\DeclareTextSymbol{\textpsi}{LGR}{121}
\DeclareTextSymbol{\textomega}{LGR}{119}


% Additional Greek symbols
% """"""""""""""""""""""""
%
% Ancient Greek Numbers (Athenian Numerals)
% '''''''''''''''''''''''''''''''''''''''''
%
% Names from ucs_ (``ucsencs.def``)
% (In ucs, five hundred is (wrongly?) named \textpentehkaton.) ::

\DeclareTextSymbol{\textpentedeka}{LGR}{2}    % GREEK ACROPHONIC ATTIC FIFTY
\DeclareTextSymbol{\textpentehekaton}{LGR}{3} % GREEK ACROPHONIC ATTIC FIVE HUNDRED
\DeclareTextSymbol{\textpenteqilioi}{LGR}{4}  % GREEK ACROPHONIC ATTIC FIVE THOUSAND
\DeclareTextSymbol{\textpentemuria}{LGR}{5}  % GREEK ACROPHONIC ATTIC FIFTY THOUSAND


% Archaic letters
% '''''''''''''''
% Names after teubner_ and ucs_::

\DeclareTextSymbol{\textstigma}{LGR}{6}       % ϛ
\DeclareTextSymbol{\textvarstigma}{LGR}{7}    % stigma variant (CB.enc, teubner)
\DeclareTextSymbol{\textkoppa}{LGR}{18}       % ϟ (greek small letter koppa)
\DeclareTextSymbol{\textqoppa}{LGR}{19}       % ϙ (archaic koppa)
\DeclareTextSymbol{\textQoppa}{LGR}{21}       % Ϙ (archaic Koppa)
\DeclareTextSymbol{\textStigma}{LGR}{22}      % Ϛ in some fonts ϹΤ ligature
\DeclareTextSymbol{\textSampi}{LGR}{23}       % Ϡ
\DeclareTextSymbol{\textsampi}{LGR}{27}       % ϡ
\DeclareTextSymbol{\textanoteleia}{LGR}{38}   % ·
\DeclareTextSymbol{\texterotimatiko}{LGR}{63} % ;
\DeclareTextSymbol{\textdigamma}{LGR}{147}    % ϝ (ucs) \digamma used by amsmath for math-macro
\DeclareTextSymbol{\textDigamma}{LGR}{195}    % Ϝ (ucs)


% Numeral signs
% '''''''''''''
% See also http://en.wikipedia.org/wiki/Greek_numerals
%
% Names from ucs_::

\DeclareTextSymbol{\textdexiakeraia}{LGR}{254}      % ʹ (Dexia keraia)
\DeclareTextSymbol{\textaristerikeraia}{LGR}{255}   % ͵ (Aristeri keraia)

% variant symbols
% '''''''''''''''
%
% Mathematical notation distinguishes variant shapes for pi, phi, rho, theta
% (small and capital), beta, and kappa (characters for the latter three
% symbols are not included in TeX’s math fonts). These variations have no
% syntactic meaning in Greek text and are not given code-points in the LGR
% encoding. Greek text fonts use the shape variants interchangeabely.
%
% However, as Unicode defines separate code points for the symbol variants, we
% provide fallback LICR macros with detailed error reporting::

\providecommand*{\LGR@TextSymbolUnavailable}[1]{%
  \PackageError{greek-fontenc}{character \string#1 not available \MessageBreak
    in 8-bit TeX}
    {Use XeTeX/LuaTeX if you need to distinguish the symbol from the letter.\MessageBreak
     The package `textalpha' with `normalize-symbols' maps the GREEK SYMBOL\MessageBreak
     character to the corresponding GREEK LETTER.\MessageBreak
     Press <return> to proceed (dropping the symbol from the document).}
}

\providecommand*{\LGR@TextSymbolOnlyMath}[2]{%
  \PackageError{greek-fontenc}{character \string#1 not available in text mode}
    {Use #2 or XeTeX/LuaTeX if you need to distinguish\MessageBreak
     the symbol from the letter.\MessageBreak
     The package `textalpha' with `normalize-symbols' maps the GREEK SYMBOL\MessageBreak
     character to the corresponding GREEK LETTER.\MessageBreak
     Press <return> to proceed (dropping the symbol from the document).}
}

\ProvideTextCommand{\textbetasymbol} {LGR}{\LGR@TextSymbolUnavailable{ϐ beta symbol}}
\ProvideTextCommand{\textkappasymbol}{LGR}{\LGR@TextSymbolUnavailable{ϰ kappa symbol}}
\ProvideTextCommand{\textThetasymbol}{LGR}{\LGR@TextSymbolUnavailable{ϴ Theta symbol}}

\ProvideTextCommand{\textthetasymbol}  {LGR}{\LGR@TextSymbolOnlyMath{ϑ theta symbol}{$\vartheta$}}
\ProvideTextCommand{\textphisymbol}    {LGR}{\LGR@TextSymbolOnlyMath{ϕ phi symbol}{$\phi$}}
\ProvideTextCommand{\textpisymbol}     {LGR}{\LGR@TextSymbolOnlyMath{ϖ pi symbol}{$\varpi$}}
\ProvideTextCommand{\textrhosymbol}    {LGR}{\LGR@TextSymbolOnlyMath{ϱ rho symbol}{$\varrho$}}
\ProvideTextCommand{\textepsilonsymbol}{LGR}{\LGR@TextSymbolOnlyMath{ϵ lunate epsilon symbol}{$\epsilon$}}

% Other
% """""
%
% Characters that also exist in other font encodings:
%
% * define them for LGR if there is a matching glyph,
% * use established macro names that match with other ``*enc.def`` files and
%   ``textcomp.sty``.
%
% All encodings: See usrguide.pdf, chapter 3.14 ff::

\DeclareTextSymbol{\textendash}{LGR}{0}          % EN DASH
\DeclareTextSymbol{\textquoteleft}{LGR}{28}      % ‘
\DeclareTextSymbol{\textquoteright}{LGR}{29}     % ’
\DeclareTextSymbol{\textperiodcentered}{LGR}{38} % · = \textanoteleia
\DeclareTextSymbol{\textcompwordmark}{LGR}{118}  % ZERO WIDTH NO JOINER
\DeclareTextSymbol{\textemdash}{LGR}{127}        % EM DASH

% T1::

\DeclareTextSymbol{\guillemotleft}{LGR}{123}     % «
\DeclareTextSymbol{\guillemotright}{LGR}{125}    % »

% TS1::

\ProvideTextCommand{\textasciibreve}{LGR}{\char30\textcompwordmark}  % ˘
\ProvideTextCommand{\textasciimacron}{LGR}{\char31\textcompwordmark} % ¯
\ProvideTextCommand{\textasciidieresis}{LGR}{"\textcompwordmark}     % "
\ProvideTextCommand{\textasciiacute}{LGR}{'\textcompwordmark}        % '
\ProvideTextCommand{\textasciigrave}{LGR}{`\textcompwordmark}        % `
\DeclareTextCommand{\textasciitilde}{LGR}{\char126\textcompwordmark} % ~
\DeclareTextSymbol{\textohm}{LGR}{87}            % Ω (Ohm sign -> Omega)
\DeclareTextSymbol{\texteuro}{LGR}{24}           % € \euro in greek.ldf
\DeclareTextSymbol{\textpercent}{LGR}{37}        % %
\DeclareTextSymbol{\textperthousand}{LGR}{25}    % ‰ \permill in greek.ldf

% T3 (tipa)::

\DeclareTextSymbol{\textschwa}{LGR}{26}          % ə

% T3 also defines some greek letters as well as Latin characters with Greek
% names:
%
% According to the Unicode standard, the following IPA characters are identic
% to the Greek letters:
%
% - 03B2 greek small letter beta
% - 03B8 greek small letter theta
% - 03BB greek small letter lamda
% - 03C7 greek small letter chi
%
% The following characters in the IPA block refer to Greek letters:
%
% - 0251  LATIN SMALL LETTER ALPHA → greek small letter alpha - 03B1
% - 025B  LATIN SMALL LETTER OPEN E = epsilon → greek small letter epsilon - 03B5
% - 0263  LATIN SMALL LETTER GAMMA → greek small letter gamma - 03B3
% - 0269  LATIN SMALL LETTER IOTA  → greek small letter iota - 03B9
% - 0278  LATIN SMALL LETTER PHI → greek small letter phi - 03C6
% - 028A  LATIN SMALL LETTER UPSILON  → greek small letter upsilon - 03C5
%
% `tipa` uses the macro names ``\textepsilon``, ``\textgamma``, ``\textiota``,
% ``\textphi``, and ``\textupsilon`` for the Latin characters. This is no
% problem with 8-bit fonts (set the font encoding to T3 vs. LGR to
% disambiguate), but leads to ambiguity with Unicode fonts. xunicode_ uses the
% ``gr`` infix and alias names to disambiguate:
%
% - 03B3 ``\textgrgamma``
% - 03B9 ``\textgriota``
% - 03C6 ``\textgrphi`` (see x0278 )
%
% but not for epsilon and upsilon (03C2 ``\textgrsigma`` is the final sigma).
%
% It also assigns alias names to the Latin counterparts, however not on a
% systematical basis:
%
% - 0194 ``\textGammaafrican``
% - 0196 ``\textIotaafrican``
% - 0251 ``\textscripta``
% - 025B ``\texteopen``
% - 0263 ``\textbabygamma``, ``\textgammalatinsmall``
% - 0269 ``\textiotalatin``
% - 028A ``\textscupsilon``  % TIPA-U
%
%
%
% Not implemented
% """""""""""""""
%
% The characters at position 16 and 17 are used in the ``\nexus`` macro
% of the teubner_ package to form an extensible, hat-like bracket. In
% ``CB.enc``, they are mapped to
%
%   16--MODIFIER LETTER LOW ACUTE ACCENT
%   17--MODIFIER LETTER LOW GRAVE ACCENT
%
% but according to the documentation of Werner Lemberg's `babel patch`_,
% they rather represent the left and right part of U+23E0 TOP TORTOISE
% SHELL BRACKET.
%
% Babel's ``lgrenc.def`` has some definitions to prevent surprises with
% macros that expect a Latin script. These definitions are problematic in a
% font-encoding definition file as they require ``\textlatin`` which is
% defined in ``babel.def``. They were moved to ``greek.ldf``.
%
% Diacritics
% ~~~~~~~~~~
%
% This section defines `named macros`_ for Greek diacritics and standard
% `symbol macros`_ as alias for simple diacritics. Convenient input
% conventions for `composite diacritics`_ are defined as `composite
% definitions`_.
%
% Advantages:
%
% * input convention similar to LGR input encodings (just prepend a ``\``).
%
% * Access pre-composed characters without resorting to the
%   ligature mechanism (allows proper kerning, cf. teubner-doc_).
%
% * Named macros can be used to ensure a font encoding supporting Greek is
%   used. However, substitution with pre-composed characters works only for
%   the active font encoding (cf. textalpha_). Named accent macros can also be
%   used instead of the ``\@tabacckludge`` by to-LaTeX converters to ensure
%   working also inside the tabbing environment.
%
% * Named macros are required for upcasing according to Greek typesetting
%   conventions (cf. `UPPERCASE`_ below) and for composite diacritics.
%
%
% Named macros
% """"""""""""
%
% Definitions are based on the teubner_ package by Claudio Beccari, the ucs_
% package, and the `babel patch`_ by Werner Lemberg. Names are derived from
% the Unicode name with the prefix "acc" (cf. `How to name accent macros?`
% in the README_) reducing the probability of incompatibilities compared to the
% two-letter teubner_ diacritic macros or the "text" prefix used by ucs_.
%
% Simple Greek diacritics::

\DeclareTextAccent{\accdialytika}{LGR}{34} % διαλυτικά (diaeresis/trema)
\DeclareTextAccent{\acctonos}{LGR}{39} % τόνος/ὀξεῖα tonos/oxia (acute)
\DeclareTextAccent{\accdasia}{LGR}{60} % δασεῖα spiritus asper (rough breathing)
\DeclareTextAccent{\accpsili}{LGR}{62} % ψιλή spiritus lenis (smooth breathing)
\DeclareTextAccent{\accvaria}{LGR}{96} % βαρεῖα (grave)
\DeclareTextAccent{\accperispomeni}{LGR}{126} % περισπωμένη (circonflex/tilde)

% The sub-iota__ is input after the base character. In LGR fonts, ligatures are
% defined for pre-composed characters, the postfix ligature does not interfere
% with kerning::

\DeclareTextSymbol{\prosgegrammeni}{LGR}{8}  % ι "capital" sub-iota
\DeclareTextSymbol{\ypogegrammeni}{LGR}{124} % ͺ "small" sub-iota

% __ http://en.wikipedia.org/wiki/Hypogegrammeni
%
% Composite diacritics::

%                                                       Teubner name
\DeclareTextAccent{\accdialytikaperispomeni}{LGR}{32} % Cd
\DeclareTextAccent{\accdialytikatonos}{LGR}{35}       % Ad
\DeclareTextAccent{\accdialytikavaria}{LGR}{36}       % Gd

\DeclareTextAccent{\accdasiaperispomeni}{LGR}{64}     % Cr
\DeclareTextAccent{\accdasiavaria}{LGR}{67}           % Gr
\DeclareTextAccent{\accdasiaoxia}{LGR}{86}            % Ar

\DeclareTextAccent{\accpsiliperispomeni}{LGR}{92}     % Cs
\DeclareTextAccent{\accpsilioxia}{LGR}{94}            % As
\DeclareTextAccent{\accpsilivaria}{LGR}{95}           % Gs

% For classical Greek and linguistics, the LGR font encoding contains a number
% of additional diacritic symbols::

\DeclareTextAccent{\accinvertedbrevebelow}{LGR}{1} % INVERTED BREVE BELOW
\DeclareTextAccent{\textsubarch}{LGR}{1}      % (tipa)
\DeclareTextAccent{\accbrevebelow}{LGR}{20}   % BREVE BELOW
\DeclareTextAccent{\u}{LGR}{30}               % BREVE (Greek vrachy)
\DeclareTextAccent{\=}{LGR}{31}               % MACRON

% Aliases
% """""""
%
% The standard _`symbol macros` are exclusively used for Greek diacritics
% in LGR::

\DeclareTextCommand{\"}{LGR}{\accdialytika}
\DeclareTextCommand{\'}{LGR}{\acctonos}
\DeclareTextCommand{\`}{LGR}{\accvaria}
\DeclareTextCommand{\~}{LGR}{\accperispomeni}

% Common Greek font encoding definitions
% ~~~~~~~~~~~~~~~~~~~~~~~~~~~~~~~~~~~~~~
%
% The file greek-fontenc.def contains font encoding definitions that are
% shared by font encodings providing Greek characters::

% greek-fontenc.def
% *****************
%
% Common Greek font encoding definitions
% ======================================
%
% :Copyright: © 2013 Günter Milde
% :Licence:   This work may be distributed and/or modified under the
%             conditions of the `LaTeX Project Public License`_, either
%             version 1.3 of this license or any later version.
% :Identification:
%  ::

\ProvidesFile{greek-fontenc.def}
[2015/08/04 v0.13.4 Common Greek font encoding definitions]

% :Abstract:  This file provides definitions that are shared between
%             font encodings providing Greek characters.
%
% Changelog:
%   .. class:: borderless
%
%   ====== ============  =========================================================
%   0.9    2013-07-03    code "outsourced" from lgrxenc.def
%   0.9.1  2013-07-17    composite definitions starting with standard accent cmds
%   0.11   2013-11-28    ``\greekscript`` TextCommand
%   0.12   2014-12-12    Remove symbol macros for breathing accents.
%   0.13   2015-08-04    No @uclclist entry for ypogegrammeni/prosgegrammeni.
%   0.13.4 2019-07-10    ``@uclclist`` entry for ``\prosgegrammeni``.
%   ====== ============  =========================================================
%
% Usage
% -----
%
% This file is input by the font encoding definition files lgrenc.def_. and
% greek-euenc.def_.
%
% Implementation
% --------------
%
% ensuregreek, greekscript
% ~~~~~~~~~~~~~~~~~~~~~~~~
%
% Provide "empty" encoding-specific definitions for ``\ensuregreek`` and
% ``\greekscript``. The definitions in this file make the font encoding fit
% for use with the Greek script. (See ``textalpha.sty`` for default
% definitions.)
%
% The ``\ensuregreek`` macro can be used to ensure that its argument is set
% in a font encoding with support for Greek. If the active font encoding
% supports it, kerning between adjacent command calls (e.g.
% ``\ensuregreek{A}\ensuregreek{\Upsilon}``) is preserved. This allows, e.g.,
% wrapping of compound Unicode character definitions. ::

\ProvideTextCommand{\ensuregreek}{\LastDeclaredEncoding}[1]{#1}

% The ``\greekscript`` declaration switches to a font encoding supporting
% the Greek script, if required. The following definition announces the
% ``\LastDeclaredEncoding`` as Greek-supporting::

\ProvideTextCommand{\greekscript}{\LastDeclaredEncoding}{}


% Aliases
% ~~~~~~~
%
% Aliases are defined via `DeclareTextCommand` so that up/downcasing works
% without the need for additional uclclist_ entries.
%
% Compatibility aliases for hyperref_'s puenc.def::

\DeclareTextCommand{\textstigmagreek}{\LastDeclaredEncoding}{\textstigma}
\DeclareTextCommand{\textkoppagreek}{\LastDeclaredEncoding}{\textkoppa}
\DeclareTextCommand{\textStigmagreek}{\LastDeclaredEncoding}{\textStigma}
\DeclareTextCommand{\textSampigreek}{\LastDeclaredEncoding}{\textSampi}
\DeclareTextCommand{\textsampigreek}{\LastDeclaredEncoding}{\textsampi}
\DeclareTextCommand{\textdigammagreek}{\LastDeclaredEncoding}{\textdigamma}
\DeclareTextCommand{\textDigammagreek}{\LastDeclaredEncoding}{\textDigamma}

\DeclareTextCommand{\textnumeralsigngreek}{\LastDeclaredEncoding}{\textdexiakeraia}
\DeclareTextCommand{\textnumeralsignlowergreek}{\LastDeclaredEncoding}{\textaristerikeraia}

% Names from babel_ and teubner_ (do we want to define them here as alias?)
%
% .. code:: latex
%
%   \DeclareTextCommand{\anwtonos}{\LastDeclaredEncoding}{\textdexiakeraia}     % ʹ
%   \DeclareTextCommand{\katwtonos}{\LastDeclaredEncoding}{\textaristerikeraia} % ͵
%
% Two Unicode code points and names for one character::

\DeclareTextCommand{\accoxia}{\LastDeclaredEncoding}{\acctonos}
\DeclareTextCommand{\acckoronis}{\LastDeclaredEncoding}{\accpsili}

% Symbol macros for the breathings:
%
% Moved to ``textalpha.sty`` to avoid clashes with local definitions
% of ``\<`` and ``\>`` in documents using LGR or babel-greek.
% (Bugreport David Kastrup). If you want to use the short macros without
% loading `textalpha`, you can define them in the document preamble like
%
% .. code:: latex
%
%   \DeclareTextCommand{\<}{LGR}{\accdasia}
%   \DeclareTextCommand{\>}{LGR}{\accpsili}
%   \DeclareTextCompositeCommand{\>}{LGR}{'}{\accpsilioxia}
%   \DeclareTextCompositeCommand{\>}{LGR}{`}{\accpsilivaria}
%   \DeclareTextCompositeCommand{\>}{LGR}{~}{\accpsiliperispomeni}
%   \DeclareTextCompositeCommand{\<}{LGR}{'}{\accdasiaoxia}
%   \DeclareTextCompositeCommand{\<}{LGR}{`}{\accdasiavaria}
%   \DeclareTextCompositeCommand{\<}{LGR}{~}{\accdasiaperispomeni}
%
% Or use alternative symbols for the breathings (cf. X compose
% table, LCI encoding)?
%
% .. code:: latex
%
%   \DeclareTextCommand{\(}{\LastDeclaredEncoding}{\accdasia}
%   \DeclareTextCommand{\)}{\LastDeclaredEncoding}{\accpsili}
%
% Composite diacritics
% ~~~~~~~~~~~~~~~~~~~~
%
% Composite accents can be input as sequence of simple diacritics (in
% arbitrary order) via named macro, symbol macro and "transcription
% characters", e.g. one of
%
% .. code:: latex
%
%   \accdialytica\accvaria\textalpha,
%   \"\'\textalpha,
%   \"'\textalpha, or
%   \'"\textalpha.
%
% Separate definitions for the supported variants of the second input token
% are required as composition is based on the non-expanded input
%
% The composition with simple "transcription characters" must also be defined
% for the symbol macros, otherwise it fails with ``\Makeuppercase``.
%
% For Unicode encodings (where they do not resolve to Greek named macros,
% composite definitions starting with standard macros are requried::

\DeclareTextCompositeCommand{\accdialytika}{\LastDeclaredEncoding}{\acctonos}{\accdialytikatonos}
\DeclareTextCompositeCommand{\accdialytika}{\LastDeclaredEncoding}{\'}{\accdialytikatonos}
\DeclareTextCompositeCommand{\accdialytika}{\LastDeclaredEncoding}{'}{\accdialytikatonos}
\DeclareTextCompositeCommand{\"}{\LastDeclaredEncoding}{\'}{\accdialytikatonos}
\DeclareTextCompositeCommand{\"}{\LastDeclaredEncoding}{'}{\accdialytikatonos}
\DeclareTextCompositeCommand{\accdialytika}{\LastDeclaredEncoding}{\accvaria}{\accdialytikavaria}
\DeclareTextCompositeCommand{\accdialytika}{\LastDeclaredEncoding}{\`}{\accdialytikavaria}
\DeclareTextCompositeCommand{\accdialytika}{\LastDeclaredEncoding}{`}{\accdialytikavaria}
\DeclareTextCompositeCommand{\"}{\LastDeclaredEncoding}{\`}{\accdialytikavaria}
\DeclareTextCompositeCommand{\"}{\LastDeclaredEncoding}{`}{\accdialytikavaria}
\DeclareTextCompositeCommand{\accdialytika}{\LastDeclaredEncoding}{\accperispomeni}{\accdialytikaperispomeni}
\DeclareTextCompositeCommand{\accdialytika}{\LastDeclaredEncoding}{\~}{\accdialytikaperispomeni}
\DeclareTextCompositeCommand{\accdialytika}{\LastDeclaredEncoding}{~}{\accdialytikaperispomeni}
\DeclareTextCompositeCommand{\"}{\LastDeclaredEncoding}{\~}{\accdialytikaperispomeni}
\DeclareTextCompositeCommand{\"}{\LastDeclaredEncoding}{~}{\accdialytikaperispomeni}

\DeclareTextCompositeCommand{\acctonos}{\LastDeclaredEncoding}{\accdialytika}{\accdialytikatonos}
\DeclareTextCompositeCommand{\acctonos}{\LastDeclaredEncoding}{\"}{\accdialytikatonos}
\DeclareTextCompositeCommand{\acctonos}{\LastDeclaredEncoding}{"}{\accdialytikatonos}
\DeclareTextCompositeCommand{\'}{\LastDeclaredEncoding}{\"}{\accdialytikatonos}
\DeclareTextCompositeCommand{\'}{\LastDeclaredEncoding}{"}{\accdialytikatonos}
\DeclareTextCompositeCommand{\acctonos}{\LastDeclaredEncoding}{\accdasia}{\accdasiaoxia}
\DeclareTextCompositeCommand{\acctonos}{\LastDeclaredEncoding}{\<}{\accdasiaoxia}
\DeclareTextCompositeCommand{\acctonos}{\LastDeclaredEncoding}{<}{\accdasiaoxia}
\DeclareTextCompositeCommand{\'}{\LastDeclaredEncoding}{\<}{\accdasiaoxia}
\DeclareTextCompositeCommand{\'}{\LastDeclaredEncoding}{<}{\accdasiaoxia}
\DeclareTextCompositeCommand{\acctonos}{\LastDeclaredEncoding}{\accpsili}{\accpsilioxia}
\DeclareTextCompositeCommand{\acctonos}{\LastDeclaredEncoding}{\>}{\accpsilioxia}
\DeclareTextCompositeCommand{\acctonos}{\LastDeclaredEncoding}{>}{\accpsilioxia}
\DeclareTextCompositeCommand{\'}{\LastDeclaredEncoding}{\>}{\accpsilioxia}
\DeclareTextCompositeCommand{\'}{\LastDeclaredEncoding}{>}{\accpsilioxia}
\DeclareTextCompositeCommand{\'}{\LastDeclaredEncoding}{\'}{\textquoteright}
\DeclareTextCompositeCommand{\'}{\LastDeclaredEncoding}{'}{\textquoteright}

\DeclareTextCompositeCommand{\accvaria}{\LastDeclaredEncoding}{\accdialytika}{\accdialytikavaria}
\DeclareTextCompositeCommand{\accvaria}{\LastDeclaredEncoding}{\"}{\accdialytikavaria}
\DeclareTextCompositeCommand{\accvaria}{\LastDeclaredEncoding}{"}{\accdialytikavaria}
\DeclareTextCompositeCommand{\`}{\LastDeclaredEncoding}{\"}{\accdialytikavaria}
\DeclareTextCompositeCommand{\`}{\LastDeclaredEncoding}{"}{\accdialytikavaria}
\DeclareTextCompositeCommand{\accvaria}{\LastDeclaredEncoding}{\accdasia}{\accdasiavaria}
\DeclareTextCompositeCommand{\accvaria}{\LastDeclaredEncoding}{\<}{\accdasiavaria}
\DeclareTextCompositeCommand{\accvaria}{\LastDeclaredEncoding}{<}{\accdasiavaria}
\DeclareTextCompositeCommand{\`}{\LastDeclaredEncoding}{\<}{\accdasiavaria}
\DeclareTextCompositeCommand{\`}{\LastDeclaredEncoding}{<}{\accdasiavaria}
\DeclareTextCompositeCommand{\accvaria}{\LastDeclaredEncoding}{\accpsili}{\accpsilivaria}
\DeclareTextCompositeCommand{\accvaria}{\LastDeclaredEncoding}{\>}{\accpsilivaria}
\DeclareTextCompositeCommand{\accvaria}{\LastDeclaredEncoding}{>}{\accpsilivaria}
\DeclareTextCompositeCommand{\`}{\LastDeclaredEncoding}{\>}{\accpsilivaria}
\DeclareTextCompositeCommand{\`}{\LastDeclaredEncoding}{>}{\accpsilivaria}
\DeclareTextCompositeCommand{\`}{\LastDeclaredEncoding}{\`}{\textquoteleft}
\DeclareTextCompositeCommand{\`}{\LastDeclaredEncoding}{`}{\textquoteleft}

\DeclareTextCompositeCommand{\accperispomeni}{\LastDeclaredEncoding}{\accdialytika}{\accdialytikaperispomeni}
\DeclareTextCompositeCommand{\accperispomeni}{\LastDeclaredEncoding}{\"}{\accdialytikaperispomeni}
\DeclareTextCompositeCommand{\accperispomeni}{\LastDeclaredEncoding}{"}{\accdialytikaperispomeni}
\DeclareTextCompositeCommand{\~}{\LastDeclaredEncoding}{\"}{\accdialytikaperispomeni}
\DeclareTextCompositeCommand{\~}{\LastDeclaredEncoding}{"}{\accdialytikaperispomeni}
\DeclareTextCompositeCommand{\accperispomeni}{\LastDeclaredEncoding}{\accdasia}{\accdasiaperispomeni}
\DeclareTextCompositeCommand{\accperispomeni}{\LastDeclaredEncoding}{\<}{\accdasiaperispomeni}
\DeclareTextCompositeCommand{\accperispomeni}{\LastDeclaredEncoding}{<}{\accdasiaperispomeni}
\DeclareTextCompositeCommand{\~}{\LastDeclaredEncoding}{\<}{\accdasiaperispomeni}
\DeclareTextCompositeCommand{\~}{\LastDeclaredEncoding}{<}{\accdasiaperispomeni}
\DeclareTextCompositeCommand{\accperispomeni}{\LastDeclaredEncoding}{\accpsili}{\accpsiliperispomeni}
\DeclareTextCompositeCommand{\accperispomeni}{\LastDeclaredEncoding}{\>}{\accpsiliperispomeni}
\DeclareTextCompositeCommand{\accperispomeni}{\LastDeclaredEncoding}{>}{\accpsiliperispomeni}
\DeclareTextCompositeCommand{\~}{\LastDeclaredEncoding}{\>}{\accpsiliperispomeni}
\DeclareTextCompositeCommand{\~}{\LastDeclaredEncoding}{>}{\accpsiliperispomeni}

\DeclareTextCompositeCommand{\accpsili}{\LastDeclaredEncoding}{\acctonos}{\accpsilioxia}
\DeclareTextCompositeCommand{\accpsili}{\LastDeclaredEncoding}{\'}{\accpsilioxia}
\DeclareTextCompositeCommand{\accpsili}{\LastDeclaredEncoding}{'}{\accpsilioxia}
\DeclareTextCompositeCommand{\accpsili}{\LastDeclaredEncoding}{\accvaria}{\accpsilivaria}
\DeclareTextCompositeCommand{\accpsili}{\LastDeclaredEncoding}{\`}{\accpsilivaria}
\DeclareTextCompositeCommand{\accpsili}{\LastDeclaredEncoding}{`}{\accpsilivaria}
\DeclareTextCompositeCommand{\accpsili}{\LastDeclaredEncoding}{\accperispomeni}{\accpsiliperispomeni}
\DeclareTextCompositeCommand{\accpsili}{\LastDeclaredEncoding}{\~}{\accpsiliperispomeni}
\DeclareTextCompositeCommand{\accpsili}{\LastDeclaredEncoding}{~}{\accpsiliperispomeni}

\DeclareTextCompositeCommand{\accdasia}{\LastDeclaredEncoding}{\acctonos}{\accdasiaoxia}
\DeclareTextCompositeCommand{\accdasia}{\LastDeclaredEncoding}{\'}{\accdasiaoxia}
\DeclareTextCompositeCommand{\accdasia}{\LastDeclaredEncoding}{'}{\accdasiaoxia}
\DeclareTextCompositeCommand{\accdasia}{\LastDeclaredEncoding}{\accvaria}{\accdasiavaria}
\DeclareTextCompositeCommand{\accdasia}{\LastDeclaredEncoding}{\`}{\accdasiavaria}
\DeclareTextCompositeCommand{\accdasia}{\LastDeclaredEncoding}{`}{\accdasiavaria}
\DeclareTextCompositeCommand{\accdasia}{\LastDeclaredEncoding}{\accperispomeni}{\accdasiaperispomeni}
\DeclareTextCompositeCommand{\accdasia}{\LastDeclaredEncoding}{\~}{\accdasiaperispomeni}

% .. _uclclist:
%
% UPPERCASE
% ~~~~~~~~~
%
% If UPPERCASE (all-caps) is generated with ``\MakeUppercase``, macro
% substitutions in the ``\@uclclist`` apply. This can be used to drop the
% diacritics (except dialytika and `iota subscript`__). Different conventions
% exist for the `treatment of the iota subscript with uppercase letters`__. In
% the CB Fonts, a small capital adscript iota is used.
%
% __ https://en.wikipedia.org/wiki/Iota_subscript
% __ https://opoudjis.net/unicode/unicode_adscript.html
%
% @uclclist extension
% """""""""""""""""""
%
% Based on the `babel patch`_ and ``fontenc.sty``.
%
% The definition of an auxiliary, self-restting macro makes this idempotent
% (only the first use of this function will expand the ``@uclclist``).
% The second and third lines are TeX' way of writing ``uclclist += …``::

\providecommand*\update@uclc@with@greek{%
  \expandafter\def\expandafter\@uclclist\expandafter{%
    \@uclclist
    \accdasia\LGR@accdropped
    \accdasiaoxia\LGR@hiatus
    \accdasiavaria\LGR@accdropped
    \accdasiaperispomeni\LGR@accdropped
    \accpsili\LGR@hiatus
    \accpsilioxia\LGR@hiatus
    \accpsilivaria\LGR@hiatus
    \accpsiliperispomeni\LGR@accdropped
    \acctonos\LGR@hiatus
    \accvaria\LGR@accdropped
    \accdialytikatonos\LGR@accDialytika
    \accdialytikavaria\LGR@accDialytika
    \accdialytikaperispomeni\LGR@accDialytika
    \accperispomeni\LGR@accdropped
    \LGR@ypogegrammeni\prosgegrammeni
    \textalpha  \textAlpha
    \textbeta   \textBeta
    \textgamma  \textGamma
    \textdelta  \textDelta
    \textepsilon\textEpsilon
    \textzeta   \textZeta
    \texteta    \textEta
    \texttheta  \textTheta
    \textiota   \textIota
    \textkappa  \textKappa
    \textlambda \textLambda
    \textmu     \textMu
    \textnu     \textNu
    \textxi     \textXi
    \textomicron\textOmicron
    \textpi     \textPi
    \textrho    \textRho
    \textvarsigma  \textSigma
    \textsigma     \textSigma
    \textautosigma \textSigma
    \texttau    \textTau
    \textupsilon\textUpsilon
    \textphi    \textPhi
    \textchi    \textChi
    \textpsi    \textPsi
    \textomega  \textOmega
    \textqoppa     \textQoppa
    \textvarstigma \textStigma
    \textstigma    \textStigma
    \textsampi     \textSampi
    \textdigamma   \textDigamma
  }%
  \let\update@uclc@with@greek\relax
}

% Expand the uclc list using the just defined macro::

\update@uclc@with@greek


% Substitution macros
% """""""""""""""""""
%
% Drop diacritics (The argument passing ensures that kerning is preserved) ::

\DeclareTextCommand{\LGR@accdropped}{\LastDeclaredEncoding}[1]{#1}


% Keep dialytika: Define a dummy alias so that downcasing with
% ``\MakeLowercase`` does not convert a simple dialytika to a composite
% accent::

\DeclareTextCommand{\LGR@accDialytika}{\LastDeclaredEncoding}{\accdialytika}
\DeclareTextCompositeCommand{\accdialytika}{\LastDeclaredEncoding}{\LGR@hiatus}{\accdialytika}

% Convert ``\prosgegrammeni`` (adscript iota) to ``\ypogegrammeni``
% (subscript iota) but not vice versa::

\DeclareTextCommand{\LGR@ypogegrammeni}{\LastDeclaredEncoding}{\ypogegrammeni}


% Mark hiatus
% """""""""""
%
% Tonos and dasia mark a hiatus if placed on the
% first vowel of a diphthong (ΑΙ, ΑΥ, ΕΙ). A dialytika must be placed on the
% second vowel if they are dropped (άυλος → ΑΫΛΟΣ, μάινα → ΜΑΪΝΑ, κέικ → ΚΕΪΚ,
% ἀυπνία → ΑΫΠΝΙΑ)::

\DeclareTextCommand{\LGR@hiatus}{\LastDeclaredEncoding}[1]{#1}
\DeclareTextCompositeCommand{\LGR@hiatus}{\LastDeclaredEncoding}{>}{\LGR@hiatus}
\DeclareTextCompositeCommand{\LGR@hiatus}{\LastDeclaredEncoding}{<}{\LGR@accdropped}
\DeclareTextCompositeCommand{\LGR@hiatus}{\LastDeclaredEncoding}{\textAlpha}{\LGR@A@hiatus}
\DeclareTextCompositeCommand{\LGR@hiatus}{\LastDeclaredEncoding}{\textEpsilon}{\LGR@E@hiatus}

% The font encoding file must define the ``\LGR@hiatus`` TextCommand. See
% ``lgrenc.def`` for an example.
%
% The following macro can be used to test if the next character is an Alpha or
% Epsilon. As ``\@ifnextchar`` gobbles spaces the lookahead macro definition
% from ``amsgen.sty`` is used:
%
%   This macro is a new version of LaTeX’s ``\@ifnextchar``, macro
%   that does not skip over spaces.
%
% ::

\providecommand*{\LGR@ifnextchar}[3]{%
  \let\reserved@d= #1%
  \def\reserved@a{#2}\def\reserved@b{#3}%
  \futurelet\@let@token\LGR@ifnch
}
\providecommand*{\LGR@ifnch}{%
  \ifx\@let@token\reserved@d \let\reserved@b\reserved@a \fi
  \reserved@b
}



% .. References
%    ----------
%
% .. _alphabeta: alphabeta.sty.html
% .. _athnum: http://www.ctan.org/pkg/athnum
% .. _Babel: http://www.ctan.org/pkg/babel
% .. _babel patch: http://www.eutypon.gr/eutypon/pdf/e2008-20/e20-a03.pdf
% .. _fntguide: http://mirror.ctan.org/macros/latex/doc/fntguide.pdf
% .. _fontenc:  http://www.ctan.org/pkg/fontenc
% .. _hyperref: http://www.ctan.org/pkg/hyperref
% .. _ibycus: http://www.ctan.org/pkg/ibycus-babel
% .. _LaTeX Project Public License: http://www.latex-project.org/lppl.txt
% .. _lgrx: http://www.ctan.org/pkg/lgrx
% .. _ltxcmds: http://www.ctan.org/pkg/ltxcmds
% .. _teubner: http://www.ctan.org/pkg/teubner
% .. _teubner-doc:
%     http://mirror.ctan.org/macros/latex/contrib/teubner/teubner-doc.pdf
% .. _textalpha: textalpha.sty.html
% .. _ucs: http://www.ctan.org/pkg/unicode
%
% .. _lgrenc.def: lgrenc.def.html
% .. _greek-euenc.def: greek-euenc.def.html



% Composite definitions
% ~~~~~~~~~~~~~~~~~~~~~
%
% Composite diacritics
% """"""""""""""""""""
%
% Composite accents can be input as sequence of simple diacritics (in
% arbitrary order) via named macro, symbol macro and "transcription
% characters", e.g. ``\accdialytica\accvaria\textalpha``, ``\"\'\textalpha``,
% ``\"'\textalpha``, or ``\'"\textalpha``.
%
% The actual declarations are part of the `common Greek font encoding
% definitions`_.
%
%
% Pre-composed letters
% """"""""""""""""""""
% Small letters with diacritics
% '''''''''''''''''''''''''''''
%
% (from teubner_ with changed names of the composite accents)::

\DeclareTextComposite{\accvaria}{LGR}{a}{128}
\DeclareTextComposite{\accdasia}{LGR}{a}{129}
\DeclareTextComposite{\accpsili}{LGR}{a}{130}
\DeclareTextComposite{\accdasiavaria}{LGR}{a}{131}
\DeclareTextComposite{\acctonos}{LGR}{a}{136}
\DeclareTextComposite{\accdasiaoxia}{LGR}{a}{137}
\DeclareTextComposite{\accpsilioxia}{LGR}{a}{138}
\DeclareTextComposite{\accpsilivaria}{LGR}{a}{139}
\DeclareTextComposite{\accperispomeni}{LGR}{a}{144}
\DeclareTextComposite{\accdasiaperispomeni}{LGR}{a}{145}
\DeclareTextComposite{\accpsiliperispomeni}{LGR}{a}{146}
\DeclareTextComposite{\accvaria}{LGR}{h}{152}
\DeclareTextComposite{\accdasia}{LGR}{h}{153}
\DeclareTextComposite{\accpsili}{LGR}{h}{154}
\DeclareTextComposite{\acctonos}{LGR}{h}{160}
\DeclareTextComposite{\accdasiaoxia}{LGR}{h}{161}
\DeclareTextComposite{\accpsilioxia}{LGR}{h}{162}
\DeclareTextComposite{\accdasiavaria}{LGR}{h}{163}
\DeclareTextComposite{\accperispomeni}{LGR}{h}{168}
\DeclareTextComposite{\accdasiaperispomeni}{LGR}{h}{169}
\DeclareTextComposite{\accpsiliperispomeni}{LGR}{h}{170}
\DeclareTextComposite{\accpsilivaria}{LGR}{h}{171}
\DeclareTextComposite{\accvaria}{LGR}{w}{176}
\DeclareTextComposite{\accdasia}{LGR}{w}{177}
\DeclareTextComposite{\accpsili}{LGR}{w}{178}
\DeclareTextComposite{\accdasiavaria}{LGR}{w}{179}
\DeclareTextComposite{\acctonos}{LGR}{w}{184}
\DeclareTextComposite{\accdasiaoxia}{LGR}{w}{185}
\DeclareTextComposite{\accpsilioxia}{LGR}{w}{186}
\DeclareTextComposite{\accpsilivaria}{LGR}{w}{187}
\DeclareTextComposite{\accperispomeni}{LGR}{w}{192}
\DeclareTextComposite{\accdasiaperispomeni}{LGR}{w}{193}
\DeclareTextComposite{\accpsiliperispomeni}{LGR}{w}{194}
\DeclareTextComposite{\accvaria}{LGR}{i}{200}
\DeclareTextComposite{\accdasia}{LGR}{i}{201}
\DeclareTextComposite{\accpsili}{LGR}{i}{202}
\DeclareTextComposite{\accdasiavaria}{LGR}{i}{203}
\DeclareTextComposite{\acctonos}{LGR}{i}{208}
\DeclareTextComposite{\accdasiaoxia}{LGR}{i}{209}
\DeclareTextComposite{\accpsilioxia}{LGR}{i}{210}
\DeclareTextComposite{\accpsilivaria}{LGR}{i}{211}
\DeclareTextComposite{\accperispomeni}{LGR}{i}{216}
\DeclareTextComposite{\accdasiaperispomeni}{LGR}{i}{217}
\DeclareTextComposite{\accpsiliperispomeni}{LGR}{i}{218}
\DeclareTextComposite{\accdialytika}{LGR}{i}{240}
\DeclareTextComposite{\accdialytikavaria}{LGR}{i}{241}
\DeclareTextComposite{\accdialytikatonos}{LGR}{i}{242}
\DeclareTextComposite{\accdialytikaperispomeni}{LGR}{i}{243}
\DeclareTextComposite{\accvaria}{LGR}{u}{204}
\DeclareTextComposite{\accdasia}{LGR}{u}{205}
\DeclareTextComposite{\accpsili}{LGR}{u}{206}
\DeclareTextComposite{\accdasiavaria}{LGR}{u}{207}
\DeclareTextComposite{\acctonos}{LGR}{u}{212}
\DeclareTextComposite{\accdasiaoxia}{LGR}{u}{213}
\DeclareTextComposite{\accpsilioxia}{LGR}{u}{214}
\DeclareTextComposite{\accpsilivaria}{LGR}{u}{215}
\DeclareTextComposite{\accperispomeni}{LGR}{u}{220}
\DeclareTextComposite{\accdasiaperispomeni}{LGR}{u}{221}
\DeclareTextComposite{\accpsiliperispomeni}{LGR}{u}{222}
\DeclareTextComposite{\accdialytika}{LGR}{u}{244}
\DeclareTextComposite{\accdialytikavaria}{LGR}{u}{245}
\DeclareTextComposite{\accdialytikatonos}{LGR}{u}{246}
\DeclareTextComposite{\accdialytikaperispomeni}{LGR}{u}{247}
\DeclareTextComposite{\accvaria}{LGR}{e}{224}
\DeclareTextComposite{\accdasia}{LGR}{e}{225}
\DeclareTextComposite{\accpsili}{LGR}{e}{226}
\DeclareTextComposite{\accdasiavaria}{LGR}{e}{227}
\DeclareTextComposite{\acctonos}{LGR}{e}{232}
\DeclareTextComposite{\accdasiaoxia}{LGR}{e}{233}
\DeclareTextComposite{\accpsilioxia}{LGR}{e}{234}
\DeclareTextComposite{\accpsilivaria}{LGR}{e}{235}
\DeclareTextComposite{\accvaria}{LGR}{o}{228}
\DeclareTextComposite{\accdasia}{LGR}{o}{229}
\DeclareTextComposite{\accpsili}{LGR}{o}{230}
\DeclareTextComposite{\accdasiavaria}{LGR}{o}{231}
\DeclareTextComposite{\acctonos}{LGR}{o}{236}
\DeclareTextComposite{\accdasiaoxia}{LGR}{o}{237}
\DeclareTextComposite{\accpsilioxia}{LGR}{o}{238}
\DeclareTextComposite{\accpsilivaria}{LGR}{o}{239}
\DeclareTextComposite{\accdasia}{LGR}{r}{251}
\DeclareTextComposite{\accpsili}{LGR}{r}{252}

% Pre-composed letters with diacritics + LICR macros::

\DeclareTextComposite{\accvaria}{LGR}{\textalpha}{128}
\DeclareTextComposite{\accdasia}{LGR}{\textalpha}{129}
\DeclareTextComposite{\accpsili}{LGR}{\textalpha}{130}
\DeclareTextComposite{\accdasiavaria}{LGR}{\textalpha}{131}
\DeclareTextComposite{\acctonos}{LGR}{\textalpha}{136}
\DeclareTextComposite{\accdasiaoxia}{LGR}{\textalpha}{137}
\DeclareTextComposite{\accpsilioxia}{LGR}{\textalpha}{138}
\DeclareTextComposite{\accpsilivaria}{LGR}{\textalpha}{139}
\DeclareTextComposite{\accperispomeni}{LGR}{\textalpha}{144}
\DeclareTextComposite{\accdasiaperispomeni}{LGR}{\textalpha}{145}
\DeclareTextComposite{\accpsiliperispomeni}{LGR}{\textalpha}{146}
\DeclareTextComposite{\accvaria}{LGR}{\texteta}{152}
\DeclareTextComposite{\accdasia}{LGR}{\texteta}{153}
\DeclareTextComposite{\accpsili}{LGR}{\texteta}{154}
\DeclareTextComposite{\acctonos}{LGR}{\texteta}{160}
\DeclareTextComposite{\accdasiaoxia}{LGR}{\texteta}{161}
\DeclareTextComposite{\accpsilioxia}{LGR}{\texteta}{162}
\DeclareTextComposite{\accdasiavaria}{LGR}{\texteta}{163}
\DeclareTextComposite{\accperispomeni}{LGR}{\texteta}{168}
\DeclareTextComposite{\accdasiaperispomeni}{LGR}{\texteta}{169}
\DeclareTextComposite{\accpsiliperispomeni}{LGR}{\texteta}{170}
\DeclareTextComposite{\accpsilivaria}{LGR}{\texteta}{171}
\DeclareTextComposite{\accvaria}{LGR}{\textomega}{176}
\DeclareTextComposite{\accdasia}{LGR}{\textomega}{177}
\DeclareTextComposite{\accpsili}{LGR}{\textomega}{178}
\DeclareTextComposite{\accdasiavaria}{LGR}{\textomega}{179}
\DeclareTextComposite{\acctonos}{LGR}{\textomega}{184}
\DeclareTextComposite{\accdasiaoxia}{LGR}{\textomega}{185}
\DeclareTextComposite{\accpsilioxia}{LGR}{\textomega}{186}
\DeclareTextComposite{\accpsilivaria}{LGR}{\textomega}{187}
\DeclareTextComposite{\accperispomeni}{LGR}{\textomega}{192}
\DeclareTextComposite{\accdasiaperispomeni}{LGR}{\textomega}{193}
\DeclareTextComposite{\accpsiliperispomeni}{LGR}{\textomega}{194}
\DeclareTextComposite{\accvaria}{LGR}{\textiota}{200}
\DeclareTextComposite{\accdasia}{LGR}{\textiota}{201}
\DeclareTextComposite{\accpsili}{LGR}{\textiota}{202}
\DeclareTextComposite{\accdasiavaria}{LGR}{\textiota}{203}
\DeclareTextComposite{\acctonos}{LGR}{\textiota}{208}
\DeclareTextComposite{\accdasiaoxia}{LGR}{\textiota}{209}
\DeclareTextComposite{\accpsilioxia}{LGR}{\textiota}{210}
\DeclareTextComposite{\accpsilivaria}{LGR}{\textiota}{211}
\DeclareTextComposite{\accperispomeni}{LGR}{\textiota}{216}
\DeclareTextComposite{\accdasiaperispomeni}{LGR}{\textiota}{217}
\DeclareTextComposite{\accpsiliperispomeni}{LGR}{\textiota}{218}
\DeclareTextComposite{\accdialytika}{LGR}{\textiota}{240}
\DeclareTextComposite{\accdialytikavaria}{LGR}{\textiota}{241}
\DeclareTextComposite{\accdialytikatonos}{LGR}{\textiota}{242}
\DeclareTextComposite{\accdialytikaperispomeni}{LGR}{\textiota}{243}
\DeclareTextComposite{\accvaria}{LGR}{\textupsilon}{204}
\DeclareTextComposite{\accdasia}{LGR}{\textupsilon}{205}
\DeclareTextComposite{\accpsili}{LGR}{\textupsilon}{206}
\DeclareTextComposite{\accdasiavaria}{LGR}{\textupsilon}{207}
\DeclareTextComposite{\acctonos}{LGR}{\textupsilon}{212}
\DeclareTextComposite{\accdasiaoxia}{LGR}{\textupsilon}{213}
\DeclareTextComposite{\accpsilioxia}{LGR}{\textupsilon}{214}
\DeclareTextComposite{\accpsilivaria}{LGR}{\textupsilon}{215}
\DeclareTextComposite{\accperispomeni}{LGR}{\textupsilon}{220}
\DeclareTextComposite{\accdasiaperispomeni}{LGR}{\textupsilon}{221}
\DeclareTextComposite{\accpsiliperispomeni}{LGR}{\textupsilon}{222}
\DeclareTextComposite{\accdialytika}{LGR}{\textupsilon}{244}
\DeclareTextComposite{\accdialytikavaria}{LGR}{\textupsilon}{245}
\DeclareTextComposite{\accdialytikatonos}{LGR}{\textupsilon}{246}
\DeclareTextComposite{\accdialytikaperispomeni}{LGR}{\textupsilon}{247}
\DeclareTextComposite{\accvaria}{LGR}{\textepsilon}{224}
\DeclareTextComposite{\accdasia}{LGR}{\textepsilon}{225}
\DeclareTextComposite{\accpsili}{LGR}{\textepsilon}{226}
\DeclareTextComposite{\accdasiavaria}{LGR}{\textepsilon}{227}
\DeclareTextComposite{\acctonos}{LGR}{\textepsilon}{232}
\DeclareTextComposite{\accdasiaoxia}{LGR}{\textepsilon}{233}
\DeclareTextComposite{\accpsilioxia}{LGR}{\textepsilon}{234}
\DeclareTextComposite{\accpsilivaria}{LGR}{\textepsilon}{235}
\DeclareTextComposite{\accvaria}{LGR}{\textomicron}{228}
\DeclareTextComposite{\accdasia}{LGR}{\textomicron}{229}
\DeclareTextComposite{\accpsili}{LGR}{\textomicron}{230}
\DeclareTextComposite{\accdasiavaria}{LGR}{\textomicron}{231}
\DeclareTextComposite{\acctonos}{LGR}{\textomicron}{236}
\DeclareTextComposite{\accdasiaoxia}{LGR}{\textomicron}{237}
\DeclareTextComposite{\accpsilioxia}{LGR}{\textomicron}{238}
\DeclareTextComposite{\accpsilivaria}{LGR}{\textomicron}{239}
\DeclareTextComposite{\accdasia}{LGR}{\textrho}{251}
\DeclareTextComposite{\accpsili}{LGR}{\textrho}{252}


% Capital letters with diacritics
% '''''''''''''''''''''''''''''''
%
%  In Greek, diacritics are omitted in all-uppercase words, but kept as part
%  of an uppercase initial (written before rather than above the letter).
%
%  The diaytika should also always be used in all-uppercase words (even
%  in cases where they are not needed when writing in lowercase)
%
%  -- http://en.wikipedia.org/wiki/Capitalization
%
% In Title Case, place diacritics before instead of above the base character::

\DeclareTextCompositeCommand{\accdasia}{LGR}{A}{<A}
\DeclareTextCompositeCommand{\accdasiavaria}{LGR}{A}{<`A}
\DeclareTextCompositeCommand{\accdasiaoxia}{LGR}{A}{<'A}
\DeclareTextCompositeCommand{\accdasiaperispomeni}{LGR}{A}{<\char126A}
\DeclareTextCompositeCommand{\accpsili}{LGR}{A}{>A}
\DeclareTextCompositeCommand{\accpsilivaria}{LGR}{A}{>`A}
\DeclareTextCompositeCommand{\accpsilioxia}{LGR}{A}{>'A}
\DeclareTextCompositeCommand{\accpsiliperispomeni}{LGR}{A}{>\char126A}
\DeclareTextCompositeCommand{\acctonos}{LGR}{A}{'A}
\DeclareTextCompositeCommand{\accvaria}{LGR}{A}{`A}

\DeclareTextCompositeCommand{\accdasia}{LGR}{E}{<E}
\DeclareTextCompositeCommand{\accdasiaoxia}{LGR}{E}{<'E}
\DeclareTextCompositeCommand{\accdasiavaria}{LGR}{E}{<`E}
\DeclareTextCompositeCommand{\accpsili}{LGR}{E}{>E}
\DeclareTextCompositeCommand{\accpsilioxia}{LGR}{E}{>'E}
\DeclareTextCompositeCommand{\accpsilivaria}{LGR}{E}{>`E}
\DeclareTextCompositeCommand{\acctonos}{LGR}{E}{'E}
\DeclareTextCompositeCommand{\accvaria}{LGR}{E}{`E}

\DeclareTextCompositeCommand{\accdasia}{LGR}{H}{<H}
\DeclareTextCompositeCommand{\accdasiavaria}{LGR}{H}{<`H}
\DeclareTextCompositeCommand{\accdasiaoxia}{LGR}{H}{<'H}
\DeclareTextCompositeCommand{\accdasiaperispomeni}{LGR}{H}{<\char126H}
\DeclareTextCompositeCommand{\accpsili}{LGR}{H}{>H}
\DeclareTextCompositeCommand{\accpsilivaria}{LGR}{H}{>`H}
\DeclareTextCompositeCommand{\accpsilioxia}{LGR}{H}{>'H}
\DeclareTextCompositeCommand{\accpsiliperispomeni}{LGR}{H}{>\char126H}
\DeclareTextCompositeCommand{\acctonos}{LGR}{H}{'H}
\DeclareTextCompositeCommand{\accvaria}{LGR}{H}{`H}

\DeclareTextCompositeCommand{\accdasia}{LGR}{I}{<I}
\DeclareTextCompositeCommand{\accdasiavaria}{LGR}{I}{<`I}
\DeclareTextCompositeCommand{\accdasiaoxia}{LGR}{I}{<'I}
\DeclareTextCompositeCommand{\accdasiaperispomeni}{LGR}{I}{<\char126I}
\DeclareTextCompositeCommand{\accpsili}{LGR}{I}{>I}
\DeclareTextCompositeCommand{\accpsilivaria}{LGR}{I}{>`I}
\DeclareTextCompositeCommand{\accpsilioxia}{LGR}{I}{>'I}
\DeclareTextCompositeCommand{\accpsiliperispomeni}{LGR}{I}{>\char126I}
\DeclareTextCompositeCommand{\acctonos}{LGR}{I}{'I}
\DeclareTextCompositeCommand{\accvaria}{LGR}{I}{`I}

\DeclareTextCompositeCommand{\accdasia}{LGR}{O}{<O}
\DeclareTextCompositeCommand{\accdasiavaria}{LGR}{O}{<`O}
\DeclareTextCompositeCommand{\accdasiaoxia}{LGR}{O}{<'O}
\DeclareTextCompositeCommand{\accpsili}{LGR}{O}{>O}
\DeclareTextCompositeCommand{\accpsilivaria}{LGR}{O}{>`O}
\DeclareTextCompositeCommand{\accpsilioxia}{LGR}{O}{>'O}
\DeclareTextCompositeCommand{\acctonos}{LGR}{O}{'O}
\DeclareTextCompositeCommand{\accvaria}{LGR}{O}{`O}

\DeclareTextCompositeCommand{\accdasia}{LGR}{R}{<R}

\DeclareTextCompositeCommand{\accdasia}{LGR}{U}{<U}
\DeclareTextCompositeCommand{\accdasiavaria}{LGR}{U}{<`U}
\DeclareTextCompositeCommand{\accdasiaoxia}{LGR}{U}{<'U}
\DeclareTextCompositeCommand{\accdasiaperispomeni}{LGR}{U}{<\char126U}
\DeclareTextCompositeCommand{\acctonos}{LGR}{U}{'U}
\DeclareTextCompositeCommand{\accvaria}{LGR}{U}{`U}

\DeclareTextCompositeCommand{\accdasia}{LGR}{W}{<W}
\DeclareTextCompositeCommand{\accdasiavaria}{LGR}{W}{<`W}
\DeclareTextCompositeCommand{\accdasiaoxia}{LGR}{W}{<'W}
\DeclareTextCompositeCommand{\accdasiaperispomeni}{LGR}{W}{<\char126W}
\DeclareTextCompositeCommand{\accpsili}{LGR}{W}{>W}
\DeclareTextCompositeCommand{\accpsilivaria}{LGR}{W}{>`W}
\DeclareTextCompositeCommand{\accpsilioxia}{LGR}{W}{>'W}
\DeclareTextCompositeCommand{\accpsiliperispomeni}{LGR}{W}{>\char126W}
\DeclareTextCompositeCommand{\acctonos}{LGR}{W}{'W}
\DeclareTextCompositeCommand{\accvaria}{LGR}{W}{`W}

% Except for the dialytika::

\DeclareTextComposite{\accdialytika}{LGR}{I}{219}
\DeclareTextComposite{\accdialytika}{LGR}{U}{223}

% Do the same for the LICR macros::

\DeclareTextCompositeCommand{\accdasia}{LGR}{\textAlpha}{<A}
\DeclareTextCompositeCommand{\accdasiavaria}{LGR}{\textAlpha}{<`A}
\DeclareTextCompositeCommand{\accdasiaoxia}{LGR}{\textAlpha}{<'A}
\DeclareTextCompositeCommand{\accdasiaperispomeni}{LGR}{\textAlpha}{<\char126A}
\DeclareTextCompositeCommand{\accpsili}{LGR}{\textAlpha}{>A}
\DeclareTextCompositeCommand{\accpsilivaria}{LGR}{\textAlpha}{>`A}
\DeclareTextCompositeCommand{\accpsilioxia}{LGR}{\textAlpha}{>'A}
\DeclareTextCompositeCommand{\accpsiliperispomeni}{LGR}{\textAlpha}{>\char126A}
\DeclareTextCompositeCommand{\acctonos}{LGR}{\textAlpha}{'A}
\DeclareTextCompositeCommand{\accvaria}{LGR}{\textAlpha}{`A}

\DeclareTextCompositeCommand{\accdasia}{LGR}{\textEpsilon}{<E}
\DeclareTextCompositeCommand{\accdasiaoxia}{LGR}{\textEpsilon}{<'E}
\DeclareTextCompositeCommand{\accdasiavaria}{LGR}{\textEpsilon}{<`E}
\DeclareTextCompositeCommand{\accpsili}{LGR}{\textEpsilon}{>E}
\DeclareTextCompositeCommand{\accpsilioxia}{LGR}{\textEpsilon}{>'E}
\DeclareTextCompositeCommand{\accpsilivaria}{LGR}{\textEpsilon}{>`E}
\DeclareTextCompositeCommand{\acctonos}{LGR}{\textEpsilon}{'E}
\DeclareTextCompositeCommand{\accvaria}{LGR}{\textEpsilon}{`E}

\DeclareTextCompositeCommand{\accdasia}{LGR}{\textEta}{<H}
\DeclareTextCompositeCommand{\accdasiavaria}{LGR}{\textEta}{<`H}
\DeclareTextCompositeCommand{\accdasiaoxia}{LGR}{\textEta}{<'H}
\DeclareTextCompositeCommand{\accdasiaperispomeni}{LGR}{\textEta}{<\char126H}
\DeclareTextCompositeCommand{\accpsili}{LGR}{\textEta}{>H}
\DeclareTextCompositeCommand{\accpsilivaria}{LGR}{\textEta}{>`H}
\DeclareTextCompositeCommand{\accpsilioxia}{LGR}{\textEta}{>'H}
\DeclareTextCompositeCommand{\accpsiliperispomeni}{LGR}{\textEta}{>\char126H}
\DeclareTextCompositeCommand{\acctonos}{LGR}{\textEta}{'H}
\DeclareTextCompositeCommand{\accvaria}{LGR}{\textEta}{`H}

\DeclareTextCompositeCommand{\accdasia}{LGR}{\textIota}{<I}
\DeclareTextCompositeCommand{\accdasiavaria}{LGR}{\textIota}{<`I}
\DeclareTextCompositeCommand{\accdasiaoxia}{LGR}{\textIota}{<'I}
\DeclareTextCompositeCommand{\accdasiaperispomeni}{LGR}{\textIota}{<\char126I}
\DeclareTextCompositeCommand{\accpsili}{LGR}{\textIota}{>I}
\DeclareTextCompositeCommand{\accpsilivaria}{LGR}{\textIota}{>`I}
\DeclareTextCompositeCommand{\accpsilioxia}{LGR}{\textIota}{>'I}
\DeclareTextCompositeCommand{\accpsiliperispomeni}{LGR}{\textIota}{>\char126I}
\DeclareTextCompositeCommand{\acctonos}{LGR}{\textIota}{'I}
\DeclareTextCompositeCommand{\accvaria}{LGR}{\textIota}{`I}

\DeclareTextCompositeCommand{\accdasia}{LGR}{\textOmicron}{<O}
\DeclareTextCompositeCommand{\accdasiavaria}{LGR}{\textOmicron}{<`O}
\DeclareTextCompositeCommand{\accdasiaoxia}{LGR}{\textOmicron}{<'O}
\DeclareTextCompositeCommand{\accpsili}{LGR}{\textOmicron}{>O}
\DeclareTextCompositeCommand{\accpsilivaria}{LGR}{\textOmicron}{>`O}
\DeclareTextCompositeCommand{\accpsilioxia}{LGR}{\textOmicron}{>'O}
\DeclareTextCompositeCommand{\acctonos}{LGR}{\textOmicron}{'O}
\DeclareTextCompositeCommand{\accvaria}{LGR}{\textOmicron}{`O}

\DeclareTextCompositeCommand{\accdasia}{LGR}{\textRho}{<R}

\DeclareTextCompositeCommand{\accdasia}{LGR}{\textUpsilon}{<U}
\DeclareTextCompositeCommand{\accdasiavaria}{LGR}{\textUpsilon}{<`U}
\DeclareTextCompositeCommand{\accdasiaoxia}{LGR}{\textUpsilon}{<'U}
\DeclareTextCompositeCommand{\accdasiaperispomeni}{LGR}{\textUpsilon}{<\char126U}
\DeclareTextCompositeCommand{\acctonos}{LGR}{\textUpsilon}{'U}
\DeclareTextCompositeCommand{\accvaria}{LGR}{\textUpsilon}{`U}

\DeclareTextCompositeCommand{\accdasia}{LGR}{\textOmega}{<W}
\DeclareTextCompositeCommand{\accdasiavaria}{LGR}{\textOmega}{<`W}
\DeclareTextCompositeCommand{\accdasiaoxia}{LGR}{\textOmega}{<'W}
\DeclareTextCompositeCommand{\accdasiaperispomeni}{LGR}{\textOmega}{<\char126W}
\DeclareTextCompositeCommand{\accpsili}{LGR}{\textOmega}{>W}
\DeclareTextCompositeCommand{\accpsilivaria}{LGR}{\textOmega}{>`W}
\DeclareTextCompositeCommand{\accpsilioxia}{LGR}{\textOmega}{>'W}
\DeclareTextCompositeCommand{\accpsiliperispomeni}{LGR}{\textOmega}{>\char126W}
\DeclareTextCompositeCommand{\acctonos}{LGR}{\textOmega}{'W}
\DeclareTextCompositeCommand{\accvaria}{LGR}{\textOmega}{`W}

% Except for the dialytika::

\DeclareTextComposite{\accdialytika}{LGR}{\textIota}{219}
\DeclareTextComposite{\accdialytika}{LGR}{\textUpsilon}{223}


% UPPERCASE
% ~~~~~~~~~
%
% The `common Greek font encoding definitions`_ in ``greek-fontenc.def`` extend
% the `uclclist` with Greek LICR macros. Here, we add LGR specific "hiatus"
% handling.
%
% Composite commands for Latin transliteration::

\DeclareTextCompositeCommand{\LGR@hiatus}{LGR}{A}{\LGR@A@hiatus}
\DeclareTextCompositeCommand{\LGR@hiatus}{LGR}{E}{\LGR@E@hiatus}

% TODO: What does the \LGR@hiatus command "see" if a Unicode literal follows?
%
% .. code:: latex
%
%   % \DeclareTextCompositeCommand{\LGR@hiatus}{LGR}{\symbol{"ce}}{bluff}
%   % \DeclareTextCompositeCommand{\LGR@hiatus}{LGR}{"cf}{blaff}
%   % \DeclareTextCompositeCommand{\LGR@hiatus}{LGR}{ι}{blaff}
%
% Look ahead and place a diaeresis on Ι or Υ::

\DeclareTextCommand{\LGR@A@hiatus}{LGR}{%
  \LGR@ifnextchar{I}{A\"}{%
    \LGR@ifnextchar{U}{A\"}{%
      \LGR@ifnextchar{\textIota}{A\"}{%
        \LGR@ifnextchar{\textUpsilon}{A\"}{A}%
      }%
    }%
  }%
}
\DeclareTextCommand{\LGR@E@hiatus}{LGR}{%
  \LGR@ifnextchar{I}{E\"}{%
    \LGR@ifnextchar{U}{E\"}{%
      \LGR@ifnextchar{\textIota}{E\"}{%
        \LGR@ifnextchar{\textUpsilon}{E\"}{E}%
      }%
    }%
  }%
}

% Unfortunately, the lookahead breaks kerning.
%
% Alternatives tried:
%
% * CompositeCommands fail at the end of a macro, e.g. ``\emph{\'a}``
%
%   .. code:: latex
%
%     \DeclareTextCommand{\LGR@A@hiatus}{LGR}{A}
%     \DeclareTextCommand{\LGR@E@hiatus}{LGR}{E}
%
%     \DeclareTextCompositeCommand{\LGR@A@hiatus}{LGR}{}{A}
%     \DeclareTextCompositeCommand{\LGR@A@hiatus}{LGR}{I}{A\"I}
%     \DeclareTextCompositeCommand{\LGR@A@hiatus}{LGR}{U}{A\"U}
%
%
% * The glyph No 12 is a special "Upcase Alpha" that in ligature with Y
%   and I adds a diaresis to them. However, it seems that it has only
%   kerning definitions for I and Y (as it is not intended for direct
%   use)
%
%   .. code:: latex
%
%     \DeclareTextComposite{\LGR@hiatus}{LGR}{A}{12}
%
%   No such glyph exists for E but this is no problem as E does
%   not require kerning anyway.
%
% .. References
%    ----------
%
% .. _README: README.html
% .. _alphabeta: alphabeta.sty.html
% .. _athnum: http://www.ctan.org/pkg/athnum
% .. _Babel: http://www.ctan.org/pkg/babel
% .. _babel patch: http://www.eutypon.gr/eutypon/pdf/e2008-20/e20-a03.pdf
% .. _fntguide: http://mirror.ctan.org/macros/latex/doc/fntguide.pdf
% .. _fontenc:  http://www.ctan.org/pkg/fontenc
% .. _hyperref: http://www.ctan.org/pkg/hyperref
% .. _ibycus: http://www.ctan.org/pkg/ibycus-babel
% .. _LaTeX Project Public License: http://www.latex-project.org/lppl.txt
% .. _lgrx: http://www.ctan.org/pkg/lgrx
% .. _ltxcmds: http://www.ctan.org/pkg/ltxcmds
% .. _teubner: http://www.ctan.org/pkg/teubner
% .. _teubner-doc:
%     http://mirror.ctan.org/macros/latex/contrib/teubner/teubner-doc.pdf
% .. _textalpha: textalpha.sty.html
% .. _ucs: http://www.ctan.org/pkg/unicode
% .. _xunicode: http://www.ctan.org/pkg/xunicode
% .. _ltxbugs 4399:
%     https://www.latex-project.org/cgi-bin/ltxbugs2html?pr=latex%2F4399&search=
%

  \DeclareRobustCommand{\rm@greekletter}[1]{{\fontencoding{LGR}\selectfont%
    \def\encodingdefault{LGR}#1}}%
% some examples
  \DeclareRobustCommand{\bpalpha}{\rm@greekletter{a}}
  \DeclareRobustCommand{\bpbeta}{\rm@greekletter{b}}
  \DeclareRobustCommand{\bpDelta}{\rm@greekletter{D}}
  \DeclareRobustCommand{\hapto}[1]{\rm@greekletter{h}\^{#1}}
\else
% code with standard math greek
  \PackageInfo{bpchem}{using default math greek fonts}
  \DeclareRobustCommand{\bpalpha}{\ensuremath{\alpha}\xspace}
  \DeclareRobustCommand{\bpbeta}{\ensuremath{\beta}\xspace} 
  \DeclareRobustCommand{\bpDelta}{\ensuremath{\Delta}\xspace}
  \DeclareRobustCommand{\hapto}[1]{\ensuremath{\eta^{#1}}}
\fi%
\let\talpha\bpalpha
\let\tbeta\bpbeta
%%%%%%
\DeclareRobustCommand*{\dreh}[1]%
  {$\lbrack \alpha \rbrack _{\mathrm D}^{#1}$}
%    \end{macrocode}
%</bpchem>
% \PrintChanges
% \CheckSum{372}
% \Finale
