% \iffalse meta-comment
%
% Copyright (C) 1993-2019
% The LaTeX3 Project and any individual authors listed elsewhere
% in this file.
%
% This file is part of the LaTeX base system.
% -------------------------------------------
%
% It may be distributed and/or modified under the
% conditions of the LaTeX Project Public License, either version 1.3c
% of this license or (at your option) any later version.
% The latest version of this license is in
%    https://www.latex-project.org/lppl.txt
% and version 1.3c or later is part of all distributions of LaTeX
% version 2008 or later.
%
% This file has the LPPL maintenance status "maintained".
%
% The list of all files belonging to the LaTeX base distribution is
% given in the file `manifest.txt'. See also `legal.txt' for additional
% information.
%
% The list of derived (unpacked) files belonging to the distribution
% and covered by LPPL is defined by the unpacking scripts (with
% extension .ins) which are part of the distribution.
%
% \fi
%\iffalse
% Copyright (C) 1985 by Leslie Lamport
% Copyright (C) 1994-98 by \LaTeX3 Project and Johannes Braams
%\fi
%
% \changes{1.0b}{1994/01/21}{added a missing \cs{end\{macro\}}, a
%    missing docstrip end of module and corrected a few typos}
% \changes{1.0d}{1994/02/11}{Corrected a few documentation errors;
%    added \cs{Checksum}}
% \changes{1.0e}{1994/02/22}{Another documentation flaw fixed}
% \changes{1.0f}{1994/03/01}{Moved driver further up; corrected a few
%    errors}
% \changes{1.0f}{1994/03/01}{Added overview of {\sc docstrip} modules}
% \changes{1.0i}{1994/11/11}{(DPC) Removed spurious \cs{wlog}}
% \changes{1.0k}{1998/08/17}{(RmS) Minor documentation corrections.}
% \changes{1.0m}{2000/03/29}{(RmS) Added macros \cs{seealso} and \cs{alsoname}.}
%
%\title{Standard \LaTeXe\ packages \texttt{makeidx} and
%      \texttt{showidx}}
%
% \author{%
%  Johannes Braams\and
%  David Carlisle\and
%  Alan Jeffrey\and
%  Leslie Lamport\and
%  Frank Mittelbach\and
%  Chris Rowley\and
%  Rainer Sch\"opf}
%
% \date{2014/09/29}
% \MaintainedByLaTeXTeam{latex}
% \maketitle
%
% \section{Description}
%
% \subsection{Makeidx}
%
%    The package \texttt{makeidx} provides two new commands, |\see|
%    and |\printindex|.
%
% \DescribeMacro\see
%    The command |\see| can used in the index to cross reference to
%    other items.
%
% \DescribeMacro\printindex
%    This command can be used to include the sorted and formatted
%    index in the document.
%
% \subsection{Showidx}
%
%    The package \texttt{showidx} changes a number of internal
%    \LaTeXe\ commands in such a way that each index entry is shown in
%    the margin on the page where the entry appears. This package was
%    originally meant to be used with the \texttt{report} and
%    \texttt{book} document classes, but works with other classes as
%    well. It makes |\flushbottom| the default.
%
% \StopEventually{}
%
% \section{The {\sc docstrip} modules}
%
% The following modules are used in the implementation to direct
% {\sc docstrip} in generating the external files:
% \begin{center}
% \begin{tabular}{ll}
%   makeidx & produce makeidx.sty\\
%   showidx & produce showidx.sty\\
%   driver  & produce a documentation driver file \\
% \end{tabular}
% \end{center}
%
% \section{The documentation driver file}
%
%    The next bit of code contains the documentation driver file for
%    \TeX{}, i.e., the file that will produce the documentation you are
%    currently reading. It can be extracted from this file by the
%    {\sc docstrip} program.
%    \begin{macrocode}
%<*driver>
\documentclass{ltxdoc}
\begin{document}
\DocInput{makeindx.dtx}
\end{document}
%</driver>
%    \end{macrocode}
%
% \section{Implementation}
%
% \subsection{Identification}
%
%    Announce the package and its version:
% \changes{v1.0h}{1994/05/01}{Removed use of variables}
%    \begin{macrocode}
%<makeidx>\ProvidesPackage{makeidx}
%<showidx>\ProvidesPackage{showidx}
                [2014/09/29 v1.0m Standard LaTeX package]
%    \end{macrocode}
%
% \subsection{Makeidx}
%
% \begin{macro}{\see}
%    This macro discards its second argument (typically a page number)
%    and just prints |\seename| together with the entry the reader is
%    pointed to.
% \changes{v1.0j}{1995/04/19}{Use \cs{emph} instead of \cs{em}}
% \changes{v1.0j}{1995/04/19}{Disallow \cs{par} in argument}
%    \begin{macrocode}
%<*makeidx>
\newcommand*\see[2]{\emph{\seename} #1}
%    \end{macrocode}
% \end{macro}
%
% \begin{macro}{\seealso}
% \changes{v1.0m}{2000/03/29}{Macro added (see PR 3133).}
%    This macro discards its second argument (typically a page number)
%    and just prints |\alsoname| together with the entry the reader is
%    pointed to. We use |\providecommand| to retain compatibility with
%    existing files that define this macro.
%    \begin{macrocode}
\providecommand*\seealso[2]{\emph{\alsoname} #1}
%    \end{macrocode}
% \end{macro}
%
% \begin{macro}{\printindex}
%    This command simply inputs the (formatted) index if it exists,
%    otherwise a warning is issued.
% \changes{1.0c}{1994/02/08}{Use \cs{@input@} instead of \cs{@input} to
%                        get the index listed by \cs{listfiles}}
%    \begin{macrocode}
\newcommand\printindex{\@input@{\jobname.ind}}
%    \end{macrocode}
% \end{macro}
%
% \begin{macro}{\seename}
%    This package is for documents prepared in the English language.
%    To prepare a version for another language, various English words
%    must be replaced.  All the English words that require replacement
%    are% defined below in command names.
% \changes{1.0g}{1994/03/07}{Replaced \cs{newcommand} by \cs{providecommand}.}
%    \begin{macrocode}
\providecommand\seename{see}
%    \end{macrocode}
%    We used |\providecommand| in case the command is already defined
%    by a package loaded earlier.
% \end{macro}
%
% \begin{macro}{\alsoname}
% \changes{v1.0m}{2000/03/29}{Macro added (see PR 3133).}
%    This macro discards its second argument (typically a page number)
%    and just prints |\alsoname| together with the entry the reader is
%    pointed to. We use |\providecommand| to retain compatibility with
%    existing files that define this macro.
%    \begin{macrocode}
\providecommand*\alsoname{see also}
%</makeidx>
%    \end{macrocode}
% \end{macro}
%
% \subsection{showidx}
%
% \begin{macro}{\indexbox}
%    This package uses \TeX's insert mechanism, therefore it needs to
%    allocate an insert register.
%    \begin{macrocode}
%<*showidx>
\newinsert\indexbox
\dimen\indexbox=\maxdimen
%    \end{macrocode}
% \end{macro}
%
% \begin{macro}{\index}
%    This is a modified default definition for the user level |\index|
%    command. The difference is the change of the category code of the
%    space character.
%    \begin{macrocode}
\renewcommand\index{\@bsphack\begingroup
                    \@sanitize\catcode32=10\relax\@index}
%    \end{macrocode}
% \end{macro}
%
% \begin{macro}{\makeindex}
%    The same change has to be included in the user level |\makeindex|
%    command, which changes the definition of |\index| to actually
%    write index entries to an external file.
%    \begin{macrocode}
\renewcommand\makeindex{\if@filesw \newwrite\@indexfile
  \immediate\openout\@indexfile=\jobname.idx
  \def\index{\@bsphack\begingroup
    \def\protect####1{\string####1\space}\@sanitize
    \catcode32=10 \@wrindex\@indexfile}\typeout
   {Writing index file \jobname.idx }\fi}
%    \end{macrocode}
% \end{macro}
%
% \begin{macro}{\@wrindex}
%    This macro takes care of writing the index entries to a file. The
%    definition is modified to call |\@showidx|.
%    \begin{macrocode}
\def\@wrindex#1#2{\let\thepage\relax
   \xdef\@gtempa{\write#1{\string
      \indexentry{#2}{\thepage}}}\endgroup\@gtempa
   \@showidx{#2}\if@nobreak \ifvmode\nobreak\fi\fi\@esphack}
%    \end{macrocode}
% \end{macro}
%
% \begin{macro}{\@index}
%    When the user didn't use the |\makeindex| command, the |\index|
%    macro calls |\@index|, which normally does basically nothing.
%    This package changes the definition to call |\@showidx|, which
%    includes the entry in the list of indexentries on the current page.
%    \begin{macrocode}
\def\@index#1{\@showidx{#1}\endgroup\@esphack}
%    \end{macrocode}
% \end{macro}
%
% \begin{macro}{\@showidx}
%    This macro adds the current index entry to the insert
%    |\indexbox|. The |\indexbox| is typeset as a flushleft paragraph.
%    \begin{macrocode}
\def\@showidx#1{%
  \insert\indexbox{\small
    \hsize\marginparwidth
    \hangindent\marginparsep \parindent\z@
    \everypar{}\let\par\@@par \parfillskip\@flushglue
    \lineskip\normallineskip
    \baselineskip .8\normalbaselineskip\sloppy
    \raggedright \leavevmode
    \vrule \@height .7\normalbaselineskip \@width \z@\relax
        #1\relax
    \vrule \@height \z@ \@depth .3\normalbaselineskip \@width \z@}}
%    \end{macrocode}
% \end{macro}
%
% \begin{macro}{\raggedbottom}
% \begin{macro}{\flushbottom}
%    The definition of these macros from \texttt{latex.dtx} is changed
%    here to add the execution of |\@mkidx| to |\@texttop|, which is
%    executed at the top of each page.
%    \begin{macrocode}
\renewcommand\raggedbottom{\def\@textbottom{\vskip
      \z@ plus.0001fil}\let\@texttop\@mkidx}
\renewcommand\flushbottom{\let\@textbottom\relax
                          \let\@texttop\@mkidx}
%    \end{macrocode}
% \end{macro}
% \end{macro}
%
% \begin{macro}{\@mkidx}
%    This macro actually typesets the box containing all the index
%    entries on the current page. They will occur on the left or the
%    right side of the text, or both, depending on the setting of the
%    switches |\if@twocolumn| and |\if@twoside|.
%    \begin{macrocode}
\def\@mkidx{\vbox to \z@{\hbox{\if@twocolumn
    \if@firstcolumn \@leftidx \else \@rightidx \fi
  \else \if@twoside \ifodd\c@page \@rightidx
                    \else \@leftidx \fi
        \else \@rightidx \fi
  \fi
  \box\indexbox}\vss}}
%    \end{macrocode}
% \end{macro}
%
% \begin{macro}{\@leftidx}
% \begin{macro}{\@rightidx}
%    These macros give the amount of displacement for the |\indexbox|.
%    \begin{macrocode}
\def\@leftidx{\hskip-\marginparsep \hskip-\marginparwidth}
\def\@rightidx{\hskip\columnwidth \hskip\marginparsep}
%    \end{macrocode}
% \end{macro}
% \end{macro}
%
%    To make this work we have to execute either |\raggedbottom| or
%    |\flushbottom|. Assuming this package is used most often with the
%    document classes \texttt{report} and \texttt{book}, we execute
%    |\flushbottom|.
%    \begin{macrocode}
\flushbottom
%</showidx>
%    \end{macrocode}
%
% \Finale
%
\endinput
