% \iffalse meta-comment
%
% Copyright (C) 1990-2004 Nederlandstalige TeX Gebruikersgroep.
% All rights reserved.
% 
% This file is part of the NTG document classes distribution
% ----------------------------------------------------------
% 
% It may be distributed and/or modified under the
% conditions of the LaTeX Project Public License, either version 1.3
% of this license or (at your option) any later version.
% The latest version of this license is in
%   http://www.latex-project.org/lppl.txt
% and version 1.3 or later is part of all distributions of LaTeX
% version 2003/12/01 or later.
% 
% This work has the LPPL maintenance status "maintained".
% 
% The Current Maintainer of this work is Johannes Braams.
% 
% The list of all files belonging to the NTG document classes
% distribution is given in the file `manifest.txt.
% 
% The list of derived (unpacked) files belonging to the distribution
% and covered by LPPL is defined by the unpacking scripts (with
% extension .ins) which are part of the distribution.
% \fi
% \CheckSum{153}
%
%\iffalse
%% Style option `A4' to be used with LaTeX V2.09
%% Copyright (C) 1990-2004 Johannes Braams, Nico Poppelier. 
%%               All rights reserved.
%%
% We begin by identifying the version of this file on the terminal and
% in the transcript file.
%<*package>
\ifx\ProvidesPackage\undefined
  \def\ProvidesPackage#1[#2 #3 #4]{%
    \wlog{LaTeX Package `#1' #3 <#2> #4}}
\fi
%</package>
%<*dtx>
\ProvidesFile{a4.dtx}
%</dtx>
%<package>\ProvidesPackage{a4}
       [2004/04/15 v1.2g A4 based page layout]
%
% Copying of this file is authorized only if either
% (1) you make absolutely no changes to your copy, including name, or
% (2) if you do make changes, you name it something other than
%     A4.doc or A4.sty.
% This restriction helps ensure that all files developed By NTG WG13
% are identical.
%
% Error reports for UNCHANGED versions please to:
%                          J.L. Braams
%                          TeXniek at braams.cistron.nl
%                      or: N.A.F.M. Poppelier
%                          N.Poppelier at elsevier.nl
%
%
%\fi
%
% \GetFileInfo{a4.dtx}
%
% \changes{1.0}{10 Jan 90}{Changed the margins for one-sided printing}
% \changes{1.1}{23 Jan 90}{Added \cs{WideMargins} macros}
% \changes{1.2}{26 Feb 90}{Consolidated setting of
%    \cs{marginparwidth}}
% \changes{1.2a}{30 Nov 90}{replaced \cs{typeout} with \cs{wlog}}
% \changes{1.2b}{1994/02/28}{Update for \LaTeXe}
% \changes{1.2b}{1994/02/28}{Removed some ltugboat remnants}
% \changes{1.2c}{1994/03/15}{Provide package a4 instead of A4}
% \changes{1.2c}{1994/03/15}{Added \cs{ProcessOptions}}
% \changes{1.2c}{1994/03/15}{Moved \cs{wlog} to the \LaTeX2.09 case}
% \changes{1.2c}{1994/03/15}{Moved execution of \cs{ProcessOptions} to
%    end of file}
% \changes{v1.2d}{1995/08/11}{Moved identification part to the top of
%    the file}
%
% \title{A style option to adapt the standard \LaTeX\ document
%        styles to A4 paper\thanks{This file
%        has version number \fileversion --
%        last revision \filedate.}}
%
% \author{Nico Poppelier\\
%          \TeX nique         \\
%          Washingtondreef 153\\
%          3564 KD Utrecht\\
%          \texttt{Poppelier@elsevier.nl}\\
%          \and
%          Johannes Braams\\
%          \TeX niek\\
%          Kooienswater 62\\
%          2715 AJ Zoetermeer\\
%          \texttt{JLBraams@cistron.nl}}
%
% \date{Printed \today}
%
% \maketitle
%
% \begin{abstract}
%   This article describes a new style option that can be used with the
%   document styles that are distributed with the \LaTeX\ distributions.
%   It modifies the page layout to conform to the paper format most
%   commonly used in Europe, portrait A4.
% \end{abstract}
%
% \section{Introduction}
%
%    This file is based on the document style options \file{A4.sty}
%    and \file{A4wide.sty}, which can be found in the Rochester style
%    archive. The original style option \file{A4.sty} we started from
%    was written by John Pavel, and is dated May 1987. This option
%    only changes the vertical size of the text somewhat, by
%    increasing the number of lines on a page. The style option
%    \file{A4wide.sty} was written by Jean-Francois Lamy, and is dated
%    July 1986. This option only increases the width of the text.
%
% \section{Goals and design decisions}
%
%    As many people before us, we found the page layout as implemented
%    in the standard \LaTeX\ document styles too much geared towards
%    the American-sized paper, which is somewhat wider than A4 paper,
%    but also noticeably less high.
%
%    Our goal was to get a page layout that was suitable for A4 paper,
%    and produced legible texts. There are a number of layout
%    parameters that influence the legibility of a text. A parameter
%    of major importance is the number of words (or characters) on a
%    line. The maximum number of words per line is ten to twelve for
%    optimal legibility, a rule-of-thumb that can be found in
%    typographic literature (we used~\cite{treebus}). This results in
%    a number of characters per line which lies somewhere between
%    sixty and seventy.
%
%    Another important parameter is the amount of white space
%    surrounding the text. Here we have to distinguish between texts
%    that are printed one-sided and texts that are printed two-sided
%    (back to back). In the first case the margins on odd and even
%    pages should be equal; in the latter case care should be taken
%    that the texts on both sides of the paper overlap. Also a printed
%    document is likely to be bound some way or another, so there
%    should be enough white space in the `inner' margin\footnote{For
%    two-sided printing, this is the left margin on odd-numbered pages
%    and the right margin on even-numbered ones; for one-sided
%    printing, this is always the left margin.} of the text to allow
%    this.
%
%    There is yet one more thing to take into account when designing a
%    page layout. \LaTeX\ offers the possibility of using marginal
%    notes and if someone wants to use marginal notes, they should of
%    course fit on the paper.
%
%    So, we have the following goals:
%    \begin{enumerate}
%    \item Choose the text width such that there will be sixty to
%    seventy characters on a line;
%    \item See to it that in documents that are printed two-sided, the
%    texts which end up on two sides of one sheet of paper overlap;
%    \item Leave enough white space in the `inner' margin to allow for
%    the binding of the document; 
%    \item Leave enough white space in the `outer' margin for marginal
%    notes if they are going to be used.
% \end{enumerate}
%
% \section{Update for \LaTeXe}
%
%    With the new document classes this package may not be needed much
%    longer, because A4 paper is now supported through the standard
%    option \Lopt{a4paper}. Yet this package additional functionality,
%    not available in the standard document classes. With \LaTeXe, you
%    can now use this package with the command:
% \begin{verbatim}
%     \usepacke[widemargins]{a4}
% \end{verbatim}
%    The option \Lopt{widemargins} executes the |\WideMargins|
%    command. 
%
% \section{The implementation}
%
% \subsection{The starting point}
%
%    Thus we set out to modify some of the design decisions in the
%    standard document styles. Because we knew that we were not the
%    first to tackle the problem, we started by having a look at what
%    was already available. We came up with the two options mentioned
%    earlier, which are publicly available. Undoubtedly there will
%    exist many more such files, some of them maybe modifications of
%    those two files.
%
%    We had a look at the layout produced by both options and were not
%    satisfied with it. For one thing, both of the original options
%    \Lopt{a4} and \Lopt{A4wide} modify only one aspect of the page
%    layout. The first thing to do was to put these two files
%    together. This resulted in a layout which was still
%    unsatisfactory, since for the 10-point and 11-point options lines
%    in the text contained on the average eighty characters or more.
%
% \subsection{What else?}
%
% \DescribeMacro\textwidth
% \DescribeMacro\marginparwidth
%    Because the result so far gave us lines that contained too many
%    characters, we decreased the |\textwidth| to get lines that
%    contain about sixty to seventy characters for all three size
%    options. Still more work had to be done. As it turned out, using
%    our new \file{A4.sty} together with the option |twosided| had a
%    drawback: when the document was printed two-sided the texts on
%    both side of one piece of paper overlapped only partly, which
%    does not look good. We solved this by modifying the width of the
%    margins for two-sided printing. At the same time we modified the
%    |\marginparwidth| so that if someone uses a marginal note it
%    would completely fit on the paper instead of falling off the
%    page, which obviously would render the note unreadable.
%
% \DescribeMacro\WideMargins
% The decisions described above allow for marginal notes to be printed
%    along with the normal text, but if someone makes heavy use of
%    marginal notes, the resultant layout will not be very
%    satisfactory, because if the full width of the marginal notes is
%    used, they will take up too much space in the `outer' margin. For
%    this case we provide the macro |\WideMargins|. This macro
%    modifies the page-layout parameters in such a way that the width
%    reserved for marginal notes becomes 1.5 inches. To achieve this
%    the width of the main body of the text is decreased. This macro
%    is meant to be used only in the preamble of the document.
%
%\StopEventually{
% \section{Conclusion}
%
%    We have presented a new approach to adapt the page layout of the
%    document styles that are part of the standard \LaTeX\
%    distributions to the dimensions of A4 paper. The width of
%    marginal notes has been taken into account and a means to get
%    wider marginal notes at the cost of shorter lines in the main
%    body of the text has been provided.
%
% \begin{thebibliography}{9}
%   \bibitem{treebus} K.F. Treebus.
%   \emph{Tekstwijzer, een gids voor het grafisch verwerken van tekst.}
%   SDU Uitgeverij ('s-Gravenhage, 1988). A Dutch book on layout
%   design and typography.
% \end{thebibliography}
% }
%
%
% \subsection{The \textsc{docstrip} modules}
%
%    The following modules are used in the implementation to direct
%    \textsc{docstrip} in generating the external files:
%    \begin{center}
%      \begin{tabular}{ll}
%        driver   & produce a documentation driver file \\
%        package  & produce a package file \\
%      \end{tabular}
%    \end{center}
%
% \subsection{Producing the documentation}
%
%    A short driver is provided that can be extracted if necessary by
%    the \textsc{docstrip} program provided with \LaTeXe.
%    \begin{macrocode}
%<*driver>
\NeedsTeXFormat{LaTeX2e}
\documentclass{ltxdoc}
\DisableCrossrefs

\CodelineIndex

\MakeShortVerb{\|}
\newcommand\Lopt[1]{\textsf{#1}}
\newcommand\file[1]{\texttt{#1}}
\begin{document}

\DocInput{a4.dtx}

\end{document}
%</driver>
%    \end{macrocode}
%
% \subsection{The code}
%
%    Declare the option \Lopt{widemargins}.
%    \begin{macrocode}
%<*package>
\ifx\documentclass\undefined
\else
  \DeclareOption{widemargins}{\WideMargins}
\fi
%    \end{macrocode}
%
% \begin{macro}{\topmargin}

%    First, we redefine the |\textheight| and |\topmargin|. The
%    |\topmargin| is the distance from the reference point on the page
%    to the top of the page of text. In most cases extra white space
%    is not necessary since one inch of white space at the top of the
%    page suffices.
%    \begin{macrocode}
\topmargin 0pt
%    \end{macrocode}
% \end{macro}
%
% \begin{macro}{\textheight}
%    The dimension parameter |\textheight| gives the total height of
%    the text, including footnotes and figures, excluding the running
%    head and foot. This height is given as an integral number times
%    the |\baselineskip|, which results in an integral number of lines
%    on a page.
%
%    We have to include definitions of all relevant dimension
%    parameters for each of the cases 10-point, 11-point and
%    12-point. We do this with a case statement:
%
%    \begin{macrocode}
\ifcase \@ptsize
    \textheight 53\baselineskip
%    \end{macrocode}
%    which modifies the height of the text for texts to be produced
%    with the ten-point typeface:
%    \begin{macrocode}
\or
    \textheight 46\baselineskip
%    \end{macrocode}
%    the same for eleven point:
%    \begin{macrocode}
\or
    \textheight 42\baselineskip
%    \end{macrocode}
%     and for twelve point. Finally we close the |\ifcase| statement:
%    \begin{macrocode}
\fi
%    \end{macrocode}
%
%    The only thing left to be done is to add the |\topskip| to the
%    |\textheight|. The value of |\topskip| appears always to be 10pt.
%
%    \begin{macrocode}
\advance\textheight by \topskip
%    \end{macrocode}
% \end{macro}
%
% \begin{macro}{\textwidth}
% \begin{macro}{\oddsidemargin}
% \begin{macro}{\evensidemargin}
%    That was the `vertical part' of the work. Now we have some work
%    to do to get things right horizontally. Again we have to
%    distinguish between the various character sizes because sixty
%    eleven-point characters take up more space than sixty ten-point
%    characters. But there's more to take into account. If documents
%    are printed two-sided, the texts on both sides of the paper
%    should overlap completely. This can be done be assigning
%    appropriate values to |\oddsidemargin| and |\evensidemargin|, the
%    parameters that define the left margins on odd and even pages
%    respectively.
%
%    First we start a case statement to distinguish between the
%    various typeface sizes.
%    \begin{macrocode}
\ifcase \@ptsize
%    \end{macrocode}
%    Then we specify the width of the text.
%    \begin{macrocode}
    \textwidth      5.00in
%    \end{macrocode}
%    Also specify the width of marginal notes. They must have a
%    reasonable width to be of any use, and this should be the same
%    for either one-sided or two-sided printing.
%    \begin{macrocode}
    \marginparwidth 1.00in
%    \end{macrocode}
%    Here we need an if statement to test whether the option
%    |twosided| has been specified.
%    \begin{macrocode}
  \if@twoside
%    \end{macrocode}
%    If it was, assign appropriate values to the margin parameters
%    \begin{macrocode}
    \oddsidemargin  0.55in
    \evensidemargin 0.75in
  \else
%    \end{macrocode}
%    If the option |twosided| was not used, both margin parameters
%    must have the same value, for texts on consecutive pages have to
%    be put in the same place on the paper.
%    \begin{macrocode}
    \oddsidemargin  0.55in
    \evensidemargin 0.55in
%    \end{macrocode}
%    Now we close the if statement.
%    \begin{macrocode}
  \fi
%    \end{macrocode}
%    We are ready with the modifications for the ten-point typeface
%    size, so now we do something similar for the eleven-point
%    typeface.
%    \begin{macrocode}
\or
    \textwidth      5.20in
    \marginparwidth 1.00in
  \if@twoside
    \oddsidemargin  0.45in
    \evensidemargin 0.65in
  \else
    \oddsidemargin  0.45in
    \evensidemargin 0.45in
  \fi
%    \end{macrocode}
%    One more time, now for the twelve-point typeface.
%    \begin{macrocode}
\or
    \textwidth      5.70in
    \marginparwidth 0.80in
  \if@twoside
    \oddsidemargin  0.20in
    \evensidemargin 0.40in
  \else
    \oddsidemargin  0.20in
    \evensidemargin 0.20in
  \fi
%    \end{macrocode}
%    Finally we close the case statement.
%    \begin{macrocode}
\fi
%    \end{macrocode}
% \end{macro}
% \end{macro}
% \end{macro}
%
% \begin{macro}{\WideMargins}
%    This macro is somewhat tricky: it has to find out
%    which typeface size is used, whether the document should be printed
%    two-sided, and whether the |\reversemarginpar| is in effect.
%    |\reversemarginpar| makes the marginal notes appear in the
%    margin on the opposite side of the normal placement.
%    \begin{macrocode}
\def\WideMargins{%
%    \end{macrocode}
% \noindent
%    Because for each typeface size the changes to the parameters that
%    need to be made are similar, the macro |\WideMargins|
%    uses an internal macro |\@widemargins|.
%
% \begin{macro}{\ExtraWidth}
%    In order to store the amount of extra width needed for the marginal
%    notes an extra dimension parameter is defined.
%
%    \begin{macrocode}
  \newdimen\ExtraWidth
%    \end{macrocode}
% \end{macro}
%
% \noindent
%    First find out about the point size, then call |\@widemargins| to
%    modify the margin widths by the amount given in |\ExtraWidth|.
%    \begin{macrocode}
  \ifcase \@ptsize
%    \end{macrocode}
%
% \noindent
%    For both 10-point and 11-point texts the width for marginal notes is
%    already 1 inch, so we increase it by half an inch. We subtract half an
%    inch from the text width and modify the margins appropriately.
%
%    \begin{macrocode}
    \ExtraWidth = 0.5in
    \@widemargins
  \or
    \ExtraWidth = 0.5in
    \@widemargins
  \or
%    \end{macrocode}
%
% \noindent
%    For 12-point texts the marginal notes are only 0.8 inch wide, so
%    now we have to add 0.7 inch to get them 1.5 inch wide.
%
%    \begin{macrocode}
    \ExtraWidth = 0.7in
    \@widemargins
%    \end{macrocode}
%
% \noindent
%    This macro should only be called once, during the preamble of a
%    document, so we |\let| it be equal to |\relax| as soon
%    as the work is done.
%    \begin{macrocode}
  \fi\let\WideMargins\relax\let\@widemargins\relax}
%    \end{macrocode}
% \end{macro}
%
% \begin{macro}{\@preamblecmds}
%    We add |\WideMargins| to |\@preamblecmds|, which is a
%    list of commands to be used only in the preamble of a document.
%
%    \begin{macrocode}
{\def\do{\noexpand\do\noexpand}
 \xdef\@preamblecmds{\@preamblecmds \do\WideMargins}
}
%    \end{macrocode}
% \end{macro}
%
% \begin{macro}{\@widemargins}
%    This macro modifies the margin parameters. To do this it uses
%    the dimension variable |\ExtraWidth|, which was defined by
%    |\WideMargins|.
%
% \noindent
%    First the |\ExtraWidth| is subtracted from the |\textwidth|
%    and added to the |\marginparwidth|.
%    \begin{macrocode}
\def\@widemargins{%
    \global\advance\textwidth by -\ExtraWidth
    \global\advance\marginparwidth by \ExtraWidth
%    \end{macrocode}
%
% \noindent
%    Then we modify the margins, but the value of the switch
%    |\if@twoside| has to be taken into account. Because
%    we have to test another switch (|\if@reversemargin|)
%    we add another level of macros to modify the margin parameters%
%
%    \begin{macrocode}
    \if@twoside
      \tw@sidedwidemargins
    \else
      \@nesidedwidemargins
    \fi}
%    \end{macrocode}
% \end{macro}
%
% \begin{macro}{\tw@sidedwidemargins}
%    Normally the marginal notes are printed in the `outer' margins, so
%    we have to increase the |\evensidemargin| to keep the text
%    balanced on both sides of the paper, but if |\reversemarginpar|
%    is in effect we have to increase the |\oddsidemargin| and
%    decrease the |\evensidemargin| accordingly.
%
%    \begin{macrocode}
\def\tw@sidedwidemargins{%
    \if@reversemargin
%    \end{macrocode}
%
% \noindent
%    Notice that for documents printed two-sided, the
%    |\evensidemargin| is wider than the |\oddsidemargin|;
%    this difference in width is transferred to the other margin.
%
%    \begin{macrocode}
      \@tempdima=\evensidemargin
      \advance\@tempdima by -\oddsidemargin
      \advance\oddsidemargin by \ExtraWidth
      \advance\oddsidemargin by \@tempdima
      \advance\evensidemargin by -\@tempdima
    \else
%    \end{macrocode}
%
% \noindent
%    If the marginal notes go on the normal side of the paper, only the
%    |\evensidemargin| has to be increased.
%
%    \begin{macrocode}
      \advance\evensidemargin by \ExtraWidth
    \fi}
%    \end{macrocode}
% \end{macro}
%
% \begin{macro}{\@nesidedwidemargins}
%    For documents that are printed one-sided, both margins have the same
%    width. The default placement for the marginal notes is in the right
%    margin, so if |\reversemarginpar| is \emph{not} in effect the
%    margin parameters need not be modified. If it is in effect, both the
%    |\oddsidemargin| and the |\evensidemargin| need to be
%    increased.
%
%    \begin{macrocode}
\def\@nesidedwidemargins{%
    \if@reversemargin
      \advance\oddsidemargin by \ExtraWidth
      \advance\evensidemargin by \ExtraWidth
    \fi}
%    \end{macrocode}
%    \end{macro}
%
%    The command |\ProcessOptions| can only be executed \emph{after}
%    |\WideMargins| has been defined. Defering the execution of
%    |\WideMargins| with |\AtBeginDocument| doesn't work, the changing
%    of |\textwidth| then comes too late because of the time when the
%    |\@begindocumenthook| gets executed by |\begin{document|.
%    \begin{macrocode}
\ifx\documentclass\undefined
\else
  \ProcessOptions
\fi
%</package>
%    \end{macrocode}
%
% \Finale
%
%% \CharacterTable
%%  {Upper-case    \A\B\C\D\E\F\G\H\I\J\K\L\M\N\O\P\Q\R\S\T\U\V\W\X\Y\Z
%%   Lower-case    \a\b\c\d\e\f\g\h\i\j\k\l\m\n\o\p\q\r\s\t\u\v\w\x\y\z
%%   Digits        \0\1\2\3\4\5\6\7\8\9
%%   Exclamation   \!     Double quote  \"     Hash (number) \#
%%   Dollar        \$     Percent       \%     Ampersand     \&
%%   Acute accent  \'     Left paren    \(     Right paren   \)
%%   Asterisk      \*     Plus          \+     Comma         \,
%%   Minus         \-     Point         \.     Solidus       \/
%%   Colon         \:     Semicolon     \;     Less than     \<
%%   Equals        \=     Greater than  \>     Question mark \?
%%   Commercial at \@     Left bracket  \[     Backslash     \\
%%   Right bracket \]     Circumflex    \^     Underscore    \_
%%   Grave accent  \`     Left brace    \{     Vertical bar  \|
%%   Right brace   \}     Tilde         \~}
%%
\endinput
