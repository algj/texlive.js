%\iffalse
% proof-at-the-end.dtx generated using makedtx version 1.2 (c) Nicola Talbot
% Command line args:
%   -doc "../doc/proof-at-the-end_doc.tex"
%   -author "Léo Colisson"
%   -dir "/tmp/pat/proof-at-the-end/"
%   -src "proof-at-the-end\.sty=>proof-at-the-end.sty"
%   proof-at-the-end
% Created on 2019/5/22 12:52
%\fi
%\iffalse
%<*package>
%% \CharacterTable
%%  {Upper-case    \A\B\C\D\E\F\G\H\I\J\K\L\M\N\O\P\Q\R\S\T\U\V\W\X\Y\Z
%%   Lower-case    \a\b\c\d\e\f\g\h\i\j\k\l\m\n\o\p\q\r\s\t\u\v\w\x\y\z
%%   Digits        \0\1\2\3\4\5\6\7\8\9
%%   Exclamation   \!     Double quote  \"     Hash (number) \#
%%   Dollar        \$     Percent       \%     Ampersand     \&
%%   Acute accent  \'     Left paren    \(     Right paren   \)
%%   Asterisk      \*     Plus          \+     Comma         \,
%%   Minus         \-     Point         \.     Solidus       \/
%%   Colon         \:     Semicolon     \;     Less than     \<
%%   Equals        \=     Greater than  \>     Question mark \?
%%   Commercial at \@     Left bracket  \[     Backslash     \\
%%   Right bracket \]     Circumflex    \^     Underscore    \_
%%   Grave accent  \`     Left brace    \{     Vertical bar  \|
%%   Right brace   \}     Tilde         \~}
%</package>
%\fi
% \iffalse
% Doc-Source file to use with LaTeX2e
% Copyright (C) 2019 Léo Colisson, all rights reserved.
% \fi
% \iffalse
%<*driver>
\PassOptionsToPackage{unicode=true}{hyperref} % options for packages loaded elsewhere
\PassOptionsToPackage{hyphens}{url}
%
\documentclass[
]{article}
\usepackage{lmodern}
\usepackage{amssymb,amsmath}
\usepackage{ifxetex,ifluatex}
\ifnum 0\ifxetex 1\fi\ifluatex 1\fi=0 % if pdftex
  \usepackage[T1]{fontenc}
  \usepackage[utf8]{inputenc}
  \usepackage{textcomp} % provides euro and other symbols
\else % if luatex or xelatex
  \usepackage{unicode-math}
  \defaultfontfeatures{Scale=MatchLowercase}
  \defaultfontfeatures[\rmfamily]{Ligatures=TeX,Scale=1}
\fi
% use upquote if available, for straight quotes in verbatim environments
\IfFileExists{upquote.sty}{\usepackage{upquote}}{}
\IfFileExists{microtype.sty}{% use microtype if available
  \usepackage[]{microtype}
  \UseMicrotypeSet[protrusion]{basicmath} % disable protrusion for tt fonts
}{}
\makeatletter
\@ifundefined{KOMAClassName}{% if non-KOMA class
  \IfFileExists{parskip.sty}{%
    \usepackage{parskip}
  }{% else
    \setlength{\parindent}{0pt}
    \setlength{\parskip}{6pt plus 2pt minus 1pt}}
}{% if KOMA class
  \KOMAoptions{parskip=half}}
\makeatother
\usepackage{xcolor}
\IfFileExists{xurl.sty}{\usepackage{xurl}}{} % add URL line breaks if available
\IfFileExists{bookmark.sty}{\usepackage{bookmark}}{\usepackage{hyperref}}
\hypersetup{
  pdftitle={Proof-at-the-end, or how to move proofs in appendix in LaTeX},
  pdfauthor={Léo Colisson},
  pdfborder={0 0 0},
  breaklinks=true}
\urlstyle{same}  % don't use monospace font for urls
\usepackage{color}
\usepackage{fancyvrb}
\newcommand{\VerbBar}{|}
\newcommand{\VERB}{\Verb[commandchars=\\\{\}]}
\DefineVerbatimEnvironment{Highlighting}{Verbatim}{commandchars=\\\{\}}
% Add ',fontsize=\small' for more characters per line
\newenvironment{Shaded}{}{}
\newcommand{\AlertTok}[1]{\textcolor[rgb]{1.00,0.00,0.00}{\textbf{#1}}}
\newcommand{\AnnotationTok}[1]{\textcolor[rgb]{0.38,0.63,0.69}{\textbf{\textit{#1}}}}
\newcommand{\AttributeTok}[1]{\textcolor[rgb]{0.49,0.56,0.16}{#1}}
\newcommand{\BaseNTok}[1]{\textcolor[rgb]{0.25,0.63,0.44}{#1}}
\newcommand{\BuiltInTok}[1]{#1}
\newcommand{\CharTok}[1]{\textcolor[rgb]{0.25,0.44,0.63}{#1}}
\newcommand{\CommentTok}[1]{\textcolor[rgb]{0.38,0.63,0.69}{\textit{#1}}}
\newcommand{\CommentVarTok}[1]{\textcolor[rgb]{0.38,0.63,0.69}{\textbf{\textit{#1}}}}
\newcommand{\ConstantTok}[1]{\textcolor[rgb]{0.53,0.00,0.00}{#1}}
\newcommand{\ControlFlowTok}[1]{\textcolor[rgb]{0.00,0.44,0.13}{\textbf{#1}}}
\newcommand{\DataTypeTok}[1]{\textcolor[rgb]{0.56,0.13,0.00}{#1}}
\newcommand{\DecValTok}[1]{\textcolor[rgb]{0.25,0.63,0.44}{#1}}
\newcommand{\DocumentationTok}[1]{\textcolor[rgb]{0.73,0.13,0.13}{\textit{#1}}}
\newcommand{\ErrorTok}[1]{\textcolor[rgb]{1.00,0.00,0.00}{\textbf{#1}}}
\newcommand{\ExtensionTok}[1]{#1}
\newcommand{\FloatTok}[1]{\textcolor[rgb]{0.25,0.63,0.44}{#1}}
\newcommand{\FunctionTok}[1]{\textcolor[rgb]{0.02,0.16,0.49}{#1}}
\newcommand{\ImportTok}[1]{#1}
\newcommand{\InformationTok}[1]{\textcolor[rgb]{0.38,0.63,0.69}{\textbf{\textit{#1}}}}
\newcommand{\KeywordTok}[1]{\textcolor[rgb]{0.00,0.44,0.13}{\textbf{#1}}}
\newcommand{\NormalTok}[1]{#1}
\newcommand{\OperatorTok}[1]{\textcolor[rgb]{0.40,0.40,0.40}{#1}}
\newcommand{\OtherTok}[1]{\textcolor[rgb]{0.00,0.44,0.13}{#1}}
\newcommand{\PreprocessorTok}[1]{\textcolor[rgb]{0.74,0.48,0.00}{#1}}
\newcommand{\RegionMarkerTok}[1]{#1}
\newcommand{\SpecialCharTok}[1]{\textcolor[rgb]{0.25,0.44,0.63}{#1}}
\newcommand{\SpecialStringTok}[1]{\textcolor[rgb]{0.73,0.40,0.53}{#1}}
\newcommand{\StringTok}[1]{\textcolor[rgb]{0.25,0.44,0.63}{#1}}
\newcommand{\VariableTok}[1]{\textcolor[rgb]{0.10,0.09,0.49}{#1}}
\newcommand{\VerbatimStringTok}[1]{\textcolor[rgb]{0.25,0.44,0.63}{#1}}
\newcommand{\WarningTok}[1]{\textcolor[rgb]{0.38,0.63,0.69}{\textbf{\textit{#1}}}}
\setlength{\emergencystretch}{3em}  % prevent overfull lines
\providecommand{\tightlist}{%
  \setlength{\itemsep}{0pt}\setlength{\parskip}{0pt}}
\setcounter{secnumdepth}{5}
% Redefines (sub)paragraphs to behave more like sections
\ifx\paragraph\undefined\else
  \let\oldparagraph\paragraph
  \renewcommand{\paragraph}[1]{\oldparagraph{#1}\mbox{}}
\fi
\ifx\subparagraph\undefined\else
  \let\oldsubparagraph\subparagraph
  \renewcommand{\subparagraph}[1]{\oldsubparagraph{#1}\mbox{}}
\fi

% set default figure placement to htbp
\makeatletter
\def\fps@figure{htbp}
\makeatother


\title{Proof-at-the-end, or how to move proofs in appendix in LaTeX}
\author{Léo Colisson}
\date{}

\begin{document}
\DocInput{proof-at-the-end.dtx}
\end{document}
%</driver>
%\fi
%\maketitle
%
%{
%\setcounter{tocdepth}{3}
%\tableofcontents
%}
%\hypertarget{introduction}{%
%\section{Introduction}\label{introduction}}
%
%This package aims to provide a way to easily move proofs in the
%appendix. You can:
%
%\begin{itemize}
%\tightlist
%\item
%  Move proofs in different places/sections by giving different
%  ``categories'' to the theorems
%\item
%  Create links from theorem to proof, and from proof to theorem
%\item
%  Restate the theorem in appendix (or before)
%\item
%  Keep the proof in the main body like normal theorems by just adding
%  with just one keyword
%\item
%  Duplicate the proof in appendix and in the current section, practical
%  to use synctex during the proof writting
%\item
%  Add comments that would appear only in the appendix (or in both body
%  and appendix)
%\item
%  Move both the theorem and the proof completely in appendix
%\item
%  Easily change the defaults, and create your own styles/environments
%\item
%  Include sketch of proof in the main text, and full proof in appendix
%\item
%  Change the text of the link, for example to translate into another
%  language
%\item
%  Have a nice environment-based commands in order to mimic the usual
%  theorem/proof structure.
%\end{itemize}
%
%NB: This project is hosted on github at
%\url{https://github.com/leo-colisson/proof-at-the-end} . Feel free to
%contribute, report bugs, or read/copy-paste the documentation/examples
%from there.
%
%Disclaimer: This package is still in beta and not considered as stable.
%
%This package is licensed under LPPL v1.3, and the last version of this
%package on CTAN is 2019/05/21.
%
%\hypertarget{demo}{%
%\section{Demo}\label{demo}}
%
%If you just want to see an example of what you can do, you can directly
%open the file \texttt{demo.pdf} (also available online at
%\url{https://github.com/leo-colisson/proof-at-the-end/raw/master/demo.pdf})
%to see what is possible, or generate it with
%
%\begin{Shaded}
%\begin{Highlighting}[]
%\FunctionTok{git}\NormalTok{ clone https://github.com/leo-colisson/proof-at-the-end.git}
%\ExtensionTok{pdflatex}\NormalTok{ demo.tex }\KeywordTok{&&} \ExtensionTok{pdflatex}\NormalTok{ demo.tex}
%\end{Highlighting}
%\end{Shaded}
%
%\hypertarget{quickstart}{%
%\section{Quickstart}\label{quickstart}}
%
%\hypertarget{install}{%
%\subsection{Install}\label{install}}
%
%If your CTAN distribution is recent enough, you have nothing to do.
%Otherwise if it's not yet in your CTAN distribution, first download the
%\texttt{proof-at-the-end.sty} file and insert it in the root of your
%project with the following commands on unix (you can also clone this
%repository if you prefer, or just manually download or copy/paste the
%files on Windows). It also requires a recent version of xparse, so for
%simplicity we included the sty file of xparse in this repository as
%well:
%
%\begin{Shaded}
%\begin{Highlighting}[]
%\BuiltInTok{cd} \OperatorTok{<}\NormalTok{your project}\OperatorTok{>}
%\VariableTok{repopratend=}\StringTok{"https://raw.githubusercontent.com/leo-colisson/proof-at-the-end"}
%\FunctionTok{wget} \VariableTok{$\{repopratend\}}\NormalTok{/master/proof-at-the-end.sty}
%\FunctionTok{wget} \VariableTok{$\{repopratend\}}\NormalTok{/master/xparse.sty}
%\end{Highlighting}
%\end{Shaded}
%
%If you have an old distribution of LaTeX (before 2018 basically, which
%is the case of Overleaf), you may also need a
%\href{https://tex.stackexchange.com/questions/489646/expl3-and-recent-xparse-on-overleaf-no-expl3-loader-detected/489649?noredirect=1\#comment1236409_489649}{more
%recent expl3}. It is also very easy to install, just download the zip
%file
%\url{http://mirrors.ctan.org/install/macros/latex/contrib/l3kernel.tds.zip},
%unzip, and copy the content of the directory
%\texttt{tex/latex/l3kernel/} into your project. On linux it's a matter
%of two commands in your project:
%
%\begin{Shaded}
%\begin{Highlighting}[]
%\FunctionTok{wget}\NormalTok{ http://mirrors.ctan.org/install/macros/latex/contrib/l3kernel.tds.zip}
%\FunctionTok{unzip}\NormalTok{ -d . -j l3kernel.tds.zip }\StringTok{'tex/latex/l3kernel/*'}
%\FunctionTok{rm}\NormalTok{ l3kernel.tds.zip}
%\end{Highlighting}
%\end{Shaded}
%
%If you don't want to pollute your main project with all these files, you
%may be interested to put them in a subfolder and update the environment
%variable \texttt{TEXINPUTS} or, if you use latexmk or overleaf, you can
%write instead a \texttt{latexmkrc} file as explained
%\href{https://www.overleaf.com/learn/latex/Questions/I_have_a_lot_of_.cls,_.sty,_.bst_files,_and_I_want_to_put_them_in_a_folder_to_keep_my_project_uncluttered._But_my_project_is_not_finding_them_to_compile_correctly}{here}.
%
%That's all!
%
%\hypertarget{use-in-your-project}{%
%\subsection{Use in your project}\label{use-in-your-project}}
%
%Now, you can load the library in your project by simply using:
%
%\begin{Shaded}
%\begin{Highlighting}[]
%\BuiltInTok{\textbackslash{}usepackage}\NormalTok{\{}\ExtensionTok{proof-at-the-end}\NormalTok{\}}
%\end{Highlighting}
%\end{Shaded}
%
%Then, you can configure your theorem/lemma/\ldots{} environments as
%usual, by using any counter you like\ldots{}:
%
%\begin{Shaded}
%\begin{Highlighting}[]
%\BuiltInTok{\textbackslash{}usepackage}\NormalTok{\{}\ExtensionTok{amssymb, amsthm, amsmath}\NormalTok{\}}
%\CommentTok{% Theorems}
%\FunctionTok{\textbackslash{}newtheorem}\NormalTok{\{thm\}\{Theorem\}[section]}
%\FunctionTok{\textbackslash{}newtheorem}\NormalTok{*\{thm*\}\{Theorem\}}
%\FunctionTok{\textbackslash{}providecommand}\NormalTok{*}\FunctionTok{\textbackslash{}thmautorefname}\NormalTok{\{Theorem\}}
%\CommentTok{% Lemmata}
%\FunctionTok{\textbackslash{}newtheorem}\NormalTok{\{lemma\}[thm]\{Lemma\}}
%\FunctionTok{\textbackslash{}newtheorem}\NormalTok{*\{lemma*\}\{Lemma\}}
%\FunctionTok{\textbackslash{}providecommand}\NormalTok{*}\FunctionTok{\textbackslash{}lemmaautorefname}\NormalTok{\{Lemma\}}
%\end{Highlighting}
%\end{Shaded}
%
%And inside your document, you can use the following syntax to create a
%new theorem:
%
%\begin{Shaded}
%\begin{Highlighting}[]
%\KeywordTok{\textbackslash{}begin}\NormalTok{\{}\ExtensionTok{theoremEnd}\NormalTok{\}[OPTIONS]\{THEOREM ENVIRONMENT\}[OPTIONAL TITLE]}
%\NormalTok{    YOUR THEOREM, with eventually labels like }\KeywordTok{\textbackslash{}label}\NormalTok{\{}\ExtensionTok{thm:OPTIONAL LABEL}\NormalTok{\}}
%\KeywordTok{\textbackslash{}end}\NormalTok{\{}\ExtensionTok{theoremEnd}\NormalTok{\}}
%\KeywordTok{\textbackslash{}begin}\NormalTok{\{}\ExtensionTok{proofEnd}\NormalTok{\} }\CommentTok{%% Optional environment}
%\NormalTok{    YOUR (OPTIONAL) PROOF}
%\KeywordTok{\textbackslash{}end}\NormalTok{\{}\ExtensionTok{proofEnd}\NormalTok{\}}
%\end{Highlighting}
%\end{Shaded}
%
%For example:
%
%\begin{Shaded}
%\begin{Highlighting}[]
%\KeywordTok{\textbackslash{}begin}\NormalTok{\{}\ExtensionTok{theoremEnd}\NormalTok{\}\{thm\}[Yes I can have a title]}
%  \KeywordTok{\textbackslash{}label}\NormalTok{\{}\ExtensionTok{thm:ilikelabels}\NormalTok{\}}
%\NormalTok{  Creating a new theorem is easy}
%\KeywordTok{\textbackslash{}end}\NormalTok{\{}\ExtensionTok{theoremEnd}\NormalTok{\}}
%\KeywordTok{\textbackslash{}begin}\NormalTok{\{}\ExtensionTok{proofEnd}\NormalTok{\}}
%\NormalTok{  You want a proof? Here is it!}
%\KeywordTok{\textbackslash{}end}\NormalTok{\{}\ExtensionTok{proofEnd}\NormalTok{\}}
%\end{Highlighting}
%\end{Shaded}
%
%And put in the place where you would like to display the theorem the
%following code:
%
%\begin{Shaded}
%\begin{Highlighting}[]
%\FunctionTok{\textbackslash{}printProofs}
%\end{Highlighting}
%\end{Shaded}
%
%If you would like to display a lemma instead, just change
%\texttt{\{thm\}} into \texttt{\{lemma\}}, or into any other theorem
%environment you defined! You can now compile safely your document ;)
%
%NB: if you want to make sure all the references are linked correctly,
%make sure to compile twice the document!
%
%Isn't it simple ?
%
%\hypertarget{use-cases}{%
%\section{Use cases}\label{use-cases}}
%
%\hypertarget{configuration-and-how-to-use-and-create-styles}{%
%\subsection{Configuration and how to use and create
%styles}\label{configuration-and-how-to-use-and-create-styles}}
%
%You can very easily configure this package, and choose how each
%theorem/proof must be displayed by providing a value in
%\texttt{OPTIONS}. For example, if you would like to keep the proof of a
%theorem in the main text like any normal theorem, use the
%\texttt{normal} option:
%
%\begin{Shaded}
%\begin{Highlighting}[]
%\KeywordTok{\textbackslash{}begin}\NormalTok{\{}\ExtensionTok{theoremEnd}\NormalTok{\}[normal]\{thm\}[A title]}
%\NormalTok{  You can easily turn a theorem back into a normal theorem!}
%\KeywordTok{\textbackslash{}end}\NormalTok{\{}\ExtensionTok{theoremEnd}\NormalTok{\}}
%\KeywordTok{\textbackslash{}begin}\NormalTok{\{}\ExtensionTok{proofEnd}\NormalTok{\}}
%\NormalTok{  And keep the proof with you!}
%\KeywordTok{\textbackslash{}end}\NormalTok{\{}\ExtensionTok{proofEnd}\NormalTok{\}}
%\end{Highlighting}
%\end{Shaded}
%
%The options are in fact a set of keys/values, thanks to
%\texttt{pgfkeys}. So you can combine them with comma separated list like
%that (order matters, as the right-most values may overwrite
%configuration set by left-most values):
%
%\begin{Shaded}
%\begin{Highlighting}[]
%\KeywordTok{\textbackslash{}begin}\NormalTok{\{}\ExtensionTok{theoremEnd}\NormalTok{\}[proof at the end,}
%\NormalTok{                   no link to proof,}
%\NormalTok{                   text proof=\{Difficult proof\}}
%\NormalTok{                  ]\{thm\}[A title]}
%\NormalTok{  Each theorem can have a custom configuration!}
%\KeywordTok{\textbackslash{}end}\NormalTok{\{}\ExtensionTok{theoremEnd}\NormalTok{\}}
%\KeywordTok{\textbackslash{}begin}\NormalTok{\{}\ExtensionTok{proofEnd}\NormalTok{\}}
%\NormalTok{  Quite practical, isn't it?}
%\KeywordTok{\textbackslash{}end}\NormalTok{\{}\ExtensionTok{proofEnd}\NormalTok{\}}
%\end{Highlighting}
%\end{Shaded}
%
%You can easily create your own styles like that:
%
%\begin{Shaded}
%\begin{Highlighting}[]
%\FunctionTok{\textbackslash{}pgfkeys}\NormalTok{\{/prAtEnd/my great style/.style=\{}
%\NormalTok{    proot at the end,}
%\NormalTok{    no link to proof,}
%\NormalTok{    text proof=\{Difficult proof\},}
%\NormalTok{  \}}
%\NormalTok{\}}
%\end{Highlighting}
%\end{Shaded}
%
%You can also change the default configuration when you load the package
%by nesting the configuration into a \texttt{conf} key:
%
%\begin{Shaded}
%\begin{Highlighting}[]
%\BuiltInTok{\textbackslash{}usepackage}\NormalTok{[conf=\{normal, one big link\}]\{}\ExtensionTok{proof-at-the-end}\NormalTok{\}}
%\end{Highlighting}
%\end{Shaded}
%
%Note however that for now it is \emph{not} possible to use macros
%directly inside the options when you load the package, so if you need to
%use more complicated configuration, you can overwrite the
%\texttt{global\ custom\ defaults} style for global configuration, and
%the \texttt{local\ custom\ defaults} style for local configuration
%(useful for example if you want to define a category for a single
%section):
%
%\begin{Shaded}
%\begin{Highlighting}[]
%\FunctionTok{\textbackslash{}pgfkeys}\NormalTok{\{/prAtEnd/global custom defaults/.style=\{}
%\NormalTok{    one big link=\{Go to proof on page~}\KeywordTok{\textbackslash{}pageref}\ErrorTok{*}\NormalTok{\{}\ExtensionTok{proof:prAtEnd\textbackslash{}pratendcountercurrent}\NormalTok{\}\}}
%\NormalTok{  \}}
%\NormalTok{\}}
%\end{Highlighting}
%\end{Shaded}
%
%and for local configuration:
%
%\begin{Shaded}
%\begin{Highlighting}[]
%\FunctionTok{\textbackslash{}pgfkeys}\NormalTok{\{/prAtEnd/local custom defaults/.style=\{}
%\NormalTok{    category=greattheorem}
%\NormalTok{  \}}
%\NormalTok{\}}
%\end{Highlighting}
%\end{Shaded}
%
%Finally, it can be practical to define custom environments to avoid
%typing always \texttt{theoremEnd}:
%
%\begin{Shaded}
%\begin{Highlighting}[]
%\FunctionTok{\textbackslash{}NewDocumentEnvironment}\NormalTok{\{thmE\}\{O\{\}O\{\}+b\}\{}\CommentTok{%}
%  \KeywordTok{\textbackslash{}begin}\NormalTok{\{}\ExtensionTok{theoremEnd}\NormalTok{\}[normal,#2]\{thm\}[#1]}\CommentTok{%}
%\NormalTok{    #3}\CommentTok{%}
%  \KeywordTok{\textbackslash{}end}\NormalTok{\{}\ExtensionTok{theoremEnd}\NormalTok{\}}\CommentTok{%}
%\NormalTok{\}\{\}}
%\CommentTok{% Do not forget the second parameter or you might get Missing \textbackslash{}begin\{document\} error}
%\FunctionTok{\textbackslash{}NewDocumentEnvironment}\NormalTok{\{proofE\}\{O\{\}+b\}\{}\CommentTok{%}
%  \KeywordTok{\textbackslash{}begin}\NormalTok{\{}\ExtensionTok{proofEnd}\NormalTok{\}[#1]}\CommentTok{%}
%\NormalTok{    #2}\CommentTok{%}
%  \KeywordTok{\textbackslash{}end}\NormalTok{\{}\ExtensionTok{proofEnd}\NormalTok{\}}\CommentTok{%}
%\NormalTok{\}\{\}}
%\end{Highlighting}
%\end{Shaded}
%
%That you could use like that:
%
%\begin{Shaded}
%\begin{Highlighting}[]
%\KeywordTok{\textbackslash{}begin}\NormalTok{\{}\ExtensionTok{thmE}\NormalTok{\}[Title]}
%\NormalTok{  Here is a normal theorem with the proof in the main text.}
%\KeywordTok{\textbackslash{}end}\NormalTok{\{}\ExtensionTok{thmE}\NormalTok{\}}
%\KeywordTok{\textbackslash{}begin}\NormalTok{\{}\ExtensionTok{proofE}\NormalTok{\}}
%\NormalTok{  The (optional) proof}
%\KeywordTok{\textbackslash{}end}\NormalTok{\{}\ExtensionTok{proofE}\NormalTok{\}}
%  
%\KeywordTok{\textbackslash{}begin}\NormalTok{\{}\ExtensionTok{thmE}\NormalTok{\}[Title][end]}
%\NormalTok{  Here is a theorem whose proof goes to the end.}
%\KeywordTok{\textbackslash{}end}\NormalTok{\{}\ExtensionTok{thmE}\NormalTok{\}}
%\KeywordTok{\textbackslash{}begin}\NormalTok{\{}\ExtensionTok{proofE}\NormalTok{\}}
%\NormalTok{  The proof}
%\KeywordTok{\textbackslash{}end}\NormalTok{\{}\ExtensionTok{proofE}\NormalTok{\}}
%
%\KeywordTok{\textbackslash{}begin}\NormalTok{\{}\ExtensionTok{thmE}\NormalTok{\}[Title][all end]}
%\NormalTok{  Here is a theorem that goes with the proof at the end.}
%\KeywordTok{\textbackslash{}end}\NormalTok{\{}\ExtensionTok{thmE}\NormalTok{\}}
%\KeywordTok{\textbackslash{}begin}\NormalTok{\{}\ExtensionTok{proofE}\NormalTok{\}}
%\NormalTok{  The proof}
%\KeywordTok{\textbackslash{}end}\NormalTok{\{}\ExtensionTok{proofE}\NormalTok{\}}
%\end{Highlighting}
%\end{Shaded}
%
%Note also that it is also possible to give options to the
%\texttt{proofEnd} environment, but it is usually useless, as it will
%automatically pick the parameters from the last \texttt{theoremEnd}
%environment. However, if for some reasons you want to change the options
%of the proof only, you can do it, but do it as your own risks ;)
%
%\hypertarget{usual-styles}{%
%\subsection{Usual styles}\label{usual-styles}}
%
%We predefined some pretty common styles/options. The full list is at the
%end of this document, but here is a list of the most practical ones:
%
%\begin{itemize}
%\tightlist
%\item
%  \texttt{normal}: turn the theorem into a ``normal'' theorem, with the
%  proof in the main text and nothing in appendix
%\item
%  \texttt{category=yourowncategory}: change the category of the theorem
%  (see next sub-section)
%\item
%  \texttt{end}: put the proof in appendix
%\item
%  \texttt{all\ end}: put both the theorem and the proof in appendix
%\item
%  \texttt{debug}: make sure the proof is written in the main text as
%  well. Practical when you write the proof to be able to use synctex (if
%  you use synctex with the proof in appendix, your will be unfortunately
%  moved to a temporary file that this library is using\ldots{} so
%  \textbf{make sure you don't modify the files named like
%  \texttt{prattheenddefaultcategory.tex} or all your changes will be
%  lost at the next compilation}!).
%\item
%  \texttt{one\ big\ link}: if you prefer to have a single big link
%  instead of two links (one for the proof, one for the page)
%\item
%  \texttt{one\ big\ link\ translated=Your\ translation}: to
%  change/translate the text of the link easily
%\item
%  \texttt{text\ link\ section}: put a link looking like ``See proof in
%  section XX.''
%\item
%  \texttt{text\ link\ section\ full\ proof}: put a link looking like
%  ``See full proof in section XX.''
%\item
%  \texttt{text\ proof\ translated=Your\ translation}: to
%  change/translate the text of the proof at the end easily
%\item
%  \texttt{global\ custom\ defaults}: empty style that you can modify to
%  change the configuration (globally)
%\item
%  \texttt{local\ custom\ defaults}: empty style that you can modify to
%  change the configuration (locally). Practical to set a category for a
%  single section.
%\end{itemize}
%
%\hypertarget{categories-or-how-to-move-proofs-in-different-sections}{%
%\subsection{Categories, or how to move proofs in different
%sections}\label{categories-or-how-to-move-proofs-in-different-sections}}
%
%Let's imagine that you have some proofs that are easy to do, and some
%proofs that are long but interesting. You may want to put the easy
%proofs in a different place that the long proofs. It is super easy to
%do, you just need to give a category name to the option
%\texttt{category} like here:
%
%\begin{Shaded}
%\begin{Highlighting}[]
%\KeywordTok{\textbackslash{}begin}\NormalTok{\{}\ExtensionTok{theoremEnd}\NormalTok{\}[category=mylongproofs]\{thm\}[A title]}
%\NormalTok{  You can easily change the place of the proofs}
%\KeywordTok{\textbackslash{}end}\NormalTok{\{}\ExtensionTok{theoremEnd}\NormalTok{\}}
%\KeywordTok{\textbackslash{}begin}\NormalTok{\{}\ExtensionTok{proofEnd}\NormalTok{\}}
%\NormalTok{  Just use a different category name!}
%\KeywordTok{\textbackslash{}end}\NormalTok{\{}\ExtensionTok{proofEnd}\NormalTok{\}}
%\end{Highlighting}
%\end{Shaded}
%
%and give this category name to \texttt{\textbackslash{}printProofs} in
%the section where you would like to display the proofs:
%
%\begin{Shaded}
%\begin{Highlighting}[]
%\FunctionTok{\textbackslash{}printProofs}\NormalTok{[mylongproofs]}
%\end{Highlighting}
%\end{Shaded}
%
%\hypertarget{comments}{%
%\subsection{Comments}\label{comments}}
%
%You can also move some text in the appendix by using:
%
%\begin{Shaded}
%\begin{Highlighting}[]
%\FunctionTok{\textbackslash{}textEnd}\NormalTok{\{Your text that should go in appendix\}}
%\end{Highlighting}
%\end{Shaded}
%
%You can also give it a category as explained above, or configure it to
%be displayed in both the main text and at the end of the file with:
%
%\begin{Shaded}
%\begin{Highlighting}[]
%\FunctionTok{\textbackslash{}textEnd}\NormalTok{[both]\{I am a comment that is written in both the main text}
%\NormalTok{and the appendix\}}
%\end{Highlighting}
%\end{Shaded}
%
%You can also use the environment notation like that:
%
%\begin{Shaded}
%\begin{Highlighting}[]
%\KeywordTok{\textbackslash{}begin}\NormalTok{\{}\ExtensionTok{textAtEnd}\NormalTok{\}[options]}
%\NormalTok{  You can also use the environment syntax.}
%\KeywordTok{\textbackslash{}end}\NormalTok{\{}\ExtensionTok{textAtEnd}\NormalTok{\}}
%\end{Highlighting}
%\end{Shaded}
%
%\hypertarget{restate-a-theorem}{%
%\subsection{Restate a theorem}\label{restate-a-theorem}}
%
%It is easy to restate a theorem in the appendix, to have both the
%theorem in the main text and in the appendix: just use the option
%\texttt{restate}:
%
%\begin{Shaded}
%\begin{Highlighting}[]
%\KeywordTok{\textbackslash{}begin}\NormalTok{\{}\ExtensionTok{theoremEnd}\NormalTok{\}[end, restate]\{thm\}[A title]}
%\NormalTok{  This theorem will be displayed both in main text and appendix.}
%\KeywordTok{\textbackslash{}end}\NormalTok{\{}\ExtensionTok{theoremEnd}\NormalTok{\}}
%\KeywordTok{\textbackslash{}begin}\NormalTok{\{}\ExtensionTok{proofEnd}\NormalTok{\}}
%\NormalTok{  Just use restate option.}
%\KeywordTok{\textbackslash{}end}\NormalTok{\{}\ExtensionTok{proofEnd}\NormalTok{\}}
%\end{Highlighting}
%\end{Shaded}
%
%You can also use the option \texttt{restate\ command=yourcustomcommand}
%in order to create a macro \texttt{\textbackslash{}yourcustomcommand}
%that will restate the theorem wherever you want (but after the
%definition).
%
%If you want to (re)state a theorem \emph{before} its definition (say in
%the introduction), there is also a special environment
%\texttt{theoremEndRestateBefore} that requires a (unique) custom name
%that you need to provide also later on in place of the real theorem with
%the option \texttt{restated\ before}:
%
%\begin{Shaded}
%\begin{Highlighting}[]
%\KeywordTok{\textbackslash{}section}\NormalTok{\{Introduction\}}
%\KeywordTok{\textbackslash{}begin}\NormalTok{\{}\ExtensionTok{theoremEndRestateBefore}\NormalTok{\}\{thm\}[Title]\{anamethatisusedtorestate\}}
%\NormalTok{  It is possible to state the theorem before}
%\NormalTok{  in the introduction, and restate it later}
%\KeywordTok{\textbackslash{}end}\NormalTok{\{}\ExtensionTok{theoremEndRestateBefore}\NormalTok{\}}
%
%\KeywordTok{\textbackslash{}section}\NormalTok{\{Real definition\}}
%\KeywordTok{\textbackslash{}begin}\NormalTok{\{}\ExtensionTok{theoremEnd}\NormalTok{\}[restated before]\{thm\}}
%\NormalTok{  anamethatisusedtorestate}
%\KeywordTok{\textbackslash{}end}\NormalTok{\{}\ExtensionTok{theoremEnd}\NormalTok{\}}
%\KeywordTok{\textbackslash{}begin}\NormalTok{\{}\ExtensionTok{proofEnd}\NormalTok{\}}
%\NormalTok{  Proof of the theorem, put in place of the theorem the unique name}
%\KeywordTok{\textbackslash{}end}\NormalTok{\{}\ExtensionTok{proofEnd}\NormalTok{\}}
%\end{Highlighting}
%\end{Shaded}
%
%\hypertarget{translate-the-links}{%
%\subsection{Translate the links}\label{translate-the-links}}
%
%The more powerful way to change the text of the links is to redefine
%\texttt{text\ link} and \texttt{text\ proof} (see section List of
%options for more details). However we defined also some easy way to
%redefine the text using \texttt{one\ big\ link\ translated} and
%\texttt{text\ proof\ translated}. For example, to create your
%\texttt{french} style you can do:
%
%\begin{Shaded}
%\begin{Highlighting}[]
%\FunctionTok{\textbackslash{}pgfkeys}\NormalTok{\{/prAtEnd/french/.style=\{}
%\NormalTok{    one big link translated=\{Voir preuve page\},}
%\NormalTok{    text proof translated=\{Preuve du\}}
%\NormalTok{  \}}
%\NormalTok{\}}
%\end{Highlighting}
%\end{Shaded}
%
%\hypertarget{write-a-sketch-of-proof-in-the-main-text}{%
%\subsection{Write a sketch of proof in the main
%text}\label{write-a-sketch-of-proof-in-the-main-text}}
%
%You can include a sketch of proof in the main text by simply adding a
%proof in between \texttt{theoremEnd} and \texttt{proofEnd}. An alias
%option \texttt{see\ full\ proof} can also be used to change the link
%into ``See full proof on page X.'':
%
%\begin{Shaded}
%\begin{Highlighting}[]
%\KeywordTok{\textbackslash{}begin}\NormalTok{\{}\ExtensionTok{theoremEnd}\NormalTok{\}[see full proof]\{thm\}}
%\NormalTok{  I can also write a sketch of proof, and put the full proof in appendix.}
%\KeywordTok{\textbackslash{}end}\NormalTok{\{}\ExtensionTok{theoremEnd}\NormalTok{\}}
%\KeywordTok{\textbackslash{}begin}\NormalTok{\{}\ExtensionTok{proof}\NormalTok{\}}
%\NormalTok{  Hint: look at the alias options.}
%\KeywordTok{\textbackslash{}end}\NormalTok{\{}\ExtensionTok{proof}\NormalTok{\}}
%\KeywordTok{\textbackslash{}begin}\NormalTok{\{}\ExtensionTok{proofEnd}\NormalTok{\}}
%\NormalTok{  You just use ``see full proof'' as an option}
%\KeywordTok{\textbackslash{}end}\NormalTok{\{}\ExtensionTok{proofEnd}\NormalTok{\}}
%\end{Highlighting}
%\end{Shaded}
%
%\hypertarget{list-of-options}{%
%\section{List of options}\label{list-of-options}}
%
%Here is the list of fundamental options supported. Most options have a
%\texttt{no} version, with \texttt{no} written before. Note that you may
%prefer to use directly the alias/styles (see next paragraph) as the
%options listed here are quite fundamental and atomic.
%
%\begin{itemize}
%\item
%  \texttt{category}: category of the proof (if you want to put proofs at
%  several places), can be anything
%\item
%  \texttt{proof\ here}/\texttt{no\ proof\ here}: put (or not) the proof
%  in the main text
%\item
%  \texttt{proof\ end}/\texttt{no\ proof\ end}: display the proof in
%  appendix
%\item
%  \texttt{restate}/\texttt{no\ restate}: restate the theorem in appendix
%\item
%  \texttt{link\ to\ proof}/\texttt{no\ link\ to\ proof}: Display a link
%  to the proof in the main text
%\item
%  \texttt{opt\ all\ end}/\texttt{no\ opt\ all\ end}: put the theorem and
%  proof only in appendix. You may prefer the alias \texttt{all\ end},
%  that also makes sure that the proof is indeed displayed in appendix.
%\item
%  \texttt{text\ link}: text of the link to the proof, defaults to
%
%  \texttt{\{See\ \textbackslash{}hyperref{[}proof:prAtEnd\textbackslash{}pratendcountercurrent{]}\{proof\}\ on\ page\textasciitilde{}\textbackslash{}pageref\{proof:prAtEnd\textbackslash{}pratendcountercurrent\}.\}}
%\item
%  \texttt{text\ proof}: text displayed in place of ``Proof'' in the
%  appendix. Defaults to
%  \texttt{\{Proof\ of\ \textbackslash{}string\textbackslash{}autoref\{thm:prAtEnd\textbackslash{}pratendcountercurrent\}\}}
%\item
%  \texttt{restate\ command}: name of a unique macro (without backslash)
%  that will be defined as an alias to restate the theorem wherever you
%  want
%\item
%  \texttt{restated\ before}: if the theorems has been stated before
%  (with \texttt{\textbackslash{}theoremProofEndRestateBefore}), then we
%  just need to put the restate command in place of the theorem, and
%  enable this option
%\item
%  \texttt{both}/\texttt{no\ both}: only for
%  \texttt{\textbackslash{}textInAppendix}, specifies that the text must
%  be present in both the main text and the appendix.
%\end{itemize}
%
%Here are all the alias/styles (you can create you own as well), they are
%practical to quickly define a behaviours, but are made of the basic
%options listed above:
%
%\begin{itemize}
%\tightlist
%\item
%  \texttt{normal}: like a `normal' theorem, without any proof in the
%  appendix, and with a proof displayed in the main text. Shortcut for
%  \texttt{proof\ here,\ no\ all\ end,\ no\ proof\ end,\ no\ link\ to\ proof,\ no\ restate,\ no\ both}.
%\item
%  \texttt{end}: theorems whose proof need to go in the appendix. Shorcut
%  for \texttt{proof\ at\ the\ end,\ link\ to\ proof}.
%\item
%  \texttt{all\ end}: makes sure both the theorem and the proof are in
%  appendix. Alias of \texttt{end,\ opt\ all\ end}.
%\item
%  \texttt{proof\ at\ the\ end}: theorems whose proof need to go in the
%  appendix contrary to \texttt{end} it does not make sure that there is
%  a link to the proof. Shorcut for
%  \texttt{no\ proof\ here,\ no\ all\ end,\ proof\ end,\ no\ both}.
%\item
%  \texttt{debug}: make sure the proof is written in the main text as
%  well (alias of \texttt{proof\ here,\ no\ opt\ all\ end}), it is quite
%  practical to use when you write a proof to be able to use synctex
%  features to move between the pdf and the file.
%\item
%  \texttt{no\ link\ to\ theorem}: Remove the link from the proof to the
%  theorem, alias of \texttt{text\ proof=\{\textbackslash{}proofname\}}
%\item
%  \texttt{stared} (or \texttt{no\ number}): when you use the stared
%  version of a theorem you don't have any number, so autoref fails to
%  write a nice link to the theorem. This option changes the text of
%  ``Proof'', by keeping the link but writting only \texttt{Proof}.
%  Equivalent to
%  \texttt{text\ proof=\{\textbackslash{}string\textbackslash{}mbox\{\textbackslash{}string\textbackslash{}hyperref{[}thm:prAtEnd\textbackslash{}pratendcountercurrent{]}\{\textbackslash{}proofname\}\}\}}
%\item
%  \texttt{see\ full\ proof}: useful when you want to write in the main
%  text only a sketch of proof, this alias writes a link
%  \texttt{See\ full\ proof\ on\ page\ X.}. Equivalent to
%  \texttt{text\ link=\{See\ \textbackslash{}hyperref{[}proof:prAtEnd\textbackslash{}pratendcountercurrent{]}\{full\ proof\}\ on\ page\textasciitilde{}\textbackslash{}pageref\{proof:prAtEnd\textbackslash{}pratendcountercurrent\}.\}}
%\item
%  \texttt{one\ big\ link}: instead of two links, one for page, one for
%  proof, put just one link around everything. It can also accept an
%  optional argument which will be the text of the link, like
%  \texttt{one\ big\ link=Go\ to\ the\ proof}. The default value is
%  \texttt{See\ proof\ on\ page\textasciitilde{}\textbackslash{}pageref*\{proof:prAtEnd\textbackslash{}pratendcountercurrent.\}}.
%\item
%  \texttt{one\ big\ link\ translated}: This is like
%  \texttt{one\ big\ link}, but automatically add the page at the end
%  (and a big link around). Practical to quickly define a translation
%  like \texttt{one\ big\ link\ translated=Voir\ preuve\ page}. See also
%  \texttt{text\ proof\ translated}.
%\item
%  \texttt{text\ link\ section}: Put a link to the proof looking like
%  ``See proof in section X''. Defaults to
%  \texttt{text\ link=\{See\ \textbackslash{}hyperref{[}proof:prAtEnd\textbackslash{}pratendcountercurrent{]}\{proof\}\ in\ \textbackslash{}autoref\{proofsection:prAtEnd\textbackslash{}pratendcountercurrent\}.\}}
%\item
%  \texttt{text\ link\ section\ full\ proof}: Put a link to the proof
%  looking like ``See full proof in section X''. Defaults to
%  \texttt{text\ link=\{See\ \textbackslash{}hyperref{[}proof:prAtEnd\textbackslash{}pratendcountercurrent{]}\{full\ proof\}\ in\ \textbackslash{}autoref\{proofsection:prAtEnd\textbackslash{}pratendcountercurrent\}.\}}
%\item
%  \texttt{default\ text\ link}: default text for the link to the proof,
%  equivalent of
%  \texttt{text\ link=\{See\ \textbackslash{}hyperref{[}proof:prAtEnd\textbackslash{}pratendcountercurrent{]}\{proof\}\ on\ page\textasciitilde{}\textbackslash{}pageref\{proof:prAtEnd\textbackslash{}pratendcountercurrent\}.\}}
%\item
%  \texttt{default\ text\ proof}: default text for the proof in appendix,
%  equivalent of
%  \texttt{text\ proof=\{Proof\ of\ \textbackslash{}string\textbackslash{}autoref\{thm:prAtEnd\textbackslash{}pratendcountercurrent\}\}}
%\item
%  \texttt{text\ proof\ translated}: like \texttt{default\ text\ proof},
%  but takes one argument and use it instead of \texttt{Proof\ of}.
%  Example: \texttt{text\ proof\ translated=\{Preuve\ du\}}
%\item
%  \texttt{bare\ defaults}: default style that is loaded before anything
%  else that configure by default a link to the proof, put the proof in
%  appendix, use the category \texttt{defaultcategory}. It is an alias of
%  \texttt{end,\ link\ to\ proof,\ no\ restate,category=defaultcategory,\ default\ text\ link,default\ text\ proof,restate\ command=pratenddummymacro}.
%\item
%  \texttt{configuration\ options}: style that contains the options used
%  to load the package. It is called right after \texttt{bare\ defaults}.
%  Note that you cannot insert macro in the options, overwrite
%  \texttt{global\ custom\ defaults} instead
%\item
%  \texttt{global\ custom\ defaults}: empty style that you can overwrite
%  to change the global defaults
%\item
%  \texttt{local\ custom\ defaults}: empty style that you can overwrite
%  to change the ``local'' defaults, like category
%\item
%  \texttt{all\ defaults}: all the defaults, equivalent of
%  \texttt{bare\ defaults,\ configuration\ options,\ global\ custom\ defaults,\ local\ custom\ defaults}
%\end{itemize}
%
%\hypertarget{contributions}{%
%\section{Contributions}\label{contributions}}
%
%Feel free to contribute, report bugs, and send pull requests on the
%github repository \url{https://github.com/leo-colisson/proof-at-the-end}
%!
%
%NB: the documentation is generated from the Markdown file
%\texttt{README.md} thanks to pandoc. These commands may help you:
%
%\begin{Shaded}
%\begin{Highlighting}[]
%\ExtensionTok\NormalTok{ Compile the demo}
%\FunctionTok{make}\NormalTok{ demo}
%\ExtensionTok\NormalTok{ Clean the project}
%\FunctionTok{make}\NormalTok{ clean}
%\ExtensionTok\NormalTok{ Generate the documentation}
%\FunctionTok{make}\NormalTok{ doc}
%\ExtensionTok\NormalTok{ Generate a package for CTAN}
%\FunctionTok{make}\NormalTok{ package}
%\end{Highlighting}
%\end{Shaded}
%
%
%
%\StopEventually{}
%\section{The Code}
%\iffalse
%    \begin{macrocode}
%<*proof-at-the-end.sty>
%    \end{macrocode}
%\fi
\NeedsTeXFormat{LaTeX2e}
\ProvidesPackage{proof-at-the-end}[2019/05/21 A package to move proofs in appendix]
 
\RequirePackage{etoolbox}
\RequirePackage{hyperref}
\RequirePackage{thmtools}
\RequirePackage{thm-restate}
\RequirePackage{catchfile}
\RequirePackage{pgfkeys}
\RequirePackage{xparse}

\RequirePackage{kvoptions}

%% https://tex.stackexchange.com/questions/109747/put-all-package-options-into-one-command
%% Forward the options list to the command \pratendOptdefconf
%% in order to use:
%% \usepackage[conf={normal}]{proof-at-the-end}
\SetupKeyvalOptions{
  family=pratendOpt,
  prefix=pratendOpt,
}
\DeclareStringOption{conf}
\DeclareLocalOptions{conf}% \pratendOptdefconf contains the proof
\ProcessKeyvalOptions*

% \DeclareOption*{\PackageWarning{proof-at-the-end}{Unknown}}
% \ProcessOptions\relax

\newwrite\appendwrite

% The first argument is the file name
% The second argument is the text to write
\NewDocumentCommand\appendtofile{m+m}{%
  \begingroup
  \IfFileExists{#1}%
  {\CatchFileDef{\filecontent}{#1}{\catcode`\\=12 \endlinechar=`^^J\catcode\endlinechar=12\relax}}% keep existing end-of-lines
  {\let\filecontent\empty}%
  \immediate\openout\appendwrite=#1\relax
  \immediate\write\appendwrite{\detokenize\expandafter{\filecontent}#2}%
  \immediate\closeout\appendwrite
  \endgroup
}

%% This functions takes one input: the category (without .tex),
%% If it's the first time we write in this
%% category file, it "blanks" it.
\def\prefixPrAtEndFiles{pratend}
\newcommand*\eraseIfNeeded[1]{%
  % A macro 'pratendmacrocat{category}' is created to check
  % if it's the first time we write in this category file.
  \protected@edef\macroname{pratendmacrocat#1}%
  \ifcsdef{\macroname}{% The macro exists, nothing to do
  }{ % The macro does not exists, create it, and empty the file
    \global\expandafter\def\csname \macroname\endcsname{true}%
    \immediate\openout\appendwrite=\prefixPrAtEndFiles#1.tex%
    \immediate\write\appendwrite{}%
    \immediate\closeout\appendwrite%
  }%
}

\newif\ifproofhere
\newif\ifproofend
\newif\ifrestatethm
\newif\iflinktoproof
\newif\ifboth
\newif\ifallattheend
\newif\ifrestatedbefore
\pgfkeys{
  /prAtEnd/.cd, %% Proof at end will be the main path
  %% Category of the proof (if you want to put proofs
  %% at several places), can be anything
  category/.initial=defaultcategory,
  category/.store in=\category,
  category/.get=\category,
  %% Display the proof in the main part
  proof here/.is if=proofhere,
  no proof here/.style={proof here=false}, % alias
  %% Display the proof when using \printProofs
  proof end/.is if=proofend,
  no proof end/.style={proof end=false},
  %% Restate the theorem when using \printProofs
  restate/.is if=restatethm,
  no restate/.style={restate=false},
  %% Put a link to the proof after the theorem
  link to proof/.is if=linktoproof,
  no link to proof/.style={link to proof=false},
  %% Put the theorem and proof only in appendix
  opt all end/.is if=allattheend,
  no opt all end/.style={opt all end=false},
  %% Text of link
  text link/.code={\def\pratendtextlink{#1}},
  %% Text of proof. Make sure also to "\renewcommand*{\proofname}{Name of the proof}"
  %% to make sure the proof for normal theorems are changed
  text proof/.code={\def\pratendtextproof{#1}},
  %% Custom restate command
  restate command/.code={\protected@edef\pratendcustomrestate{#1}},
  %% (Re)stated before
  %% If the theorems has been stated before, then we just need to put the restate command in
  %% place of the argument, and we set this value to true:
  restated before/.is if=restatedbefore,
  no restated before/.style={restated before=false},
  %% In star version, we don't want 
  %% Put the text (defined only for \textInAppendix) in both the
  %% current location and in appendix
  both/.is if=both,
  no both/.style={both=false},
  %%%% Alias and styles
  normal/.style={
    proof here,
    no opt all end,
    no proof end,
    no link to proof,
    no restate,
    no both,
  },
  proof at the end/.style={
    no proof here,
    no opt all end,
    proof end,
    no both,
  },
  end/.style={
    proof at the end,
    link to proof,
  },
  all end/.style={
    end,
    opt all end,
  },
  debug/.style={
    no opt all end,
    proof here
  },
  no link to theorem/.style={ % Remove the link to the theorem
    text proof={\proofname},
  },
  stared/.style={ % Remove 
    text proof={\string\mbox{\string\hyperref[thm:prAtEnd\pratendcountercurrent]{\proofname}}},
  },
  no number/.style={
    stared
  },
  see full proof/.style={
    text link={See \hyperref[proof:prAtEnd\pratendcountercurrent]{full proof} on page~\pageref{proof:prAtEnd\pratendcountercurrent}.}
  },
  one big link/.style={
    text link={\hyperref[proof:prAtEnd\pratendcountercurrent] {#1}}
  },
  one big link/.default={%
    See proof on page~\pageref*{proof:prAtEnd\pratendcountercurrent}.
  },
  one big link translated/.style={
    one big link={#1~\pageref*{proof:prAtEnd\pratendcountercurrent}.}
  },
  text link section/.style={%
      text link={See \hyperref[proof:prAtEnd\pratendcountercurrent]{proof} in \autoref{proofsection:prAtEnd\pratendcountercurrent}.}
  },
  text link section full proof/.style={%
      text link={See \hyperref[proof:prAtEnd\pratendcountercurrent]{full proof} in \autoref{proofsection:prAtEnd\pratendcountercurrent}.}
  },
  default text link/.style={
    text link={See \hyperref[proof:prAtEnd\pratendcountercurrent]{proof} on page~\pageref{proof:prAtEnd\pratendcountercurrent}.},
  },
  text proof translated/.style={
    text proof={#1 \string\autoref{thm:prAtEnd\pratendcountercurrent}},
  },
  default text proof/.style={
    text proof={Proof of \string\autoref{thm:prAtEnd\pratendcountercurrent}},
  },
  %%%% Defaults
  bare defaults/.style={
    end,
    link to proof,
    no restate,
    category=defaultcategory,
    default text link,
    default text proof,
    restate command=pratenddummymacro,
  },
  configuration options/.style/.expand once={
    % This styles will contain the configuration
    % given as options of the package like:
    % \usepackage[conf={normal, no link to proof}]{proof-at-the-end}
    % The package options does not accept macros and valued keys
    % due to some fundamental issues:
    % https://tex.stackexchange.com/questions/489564/use-unexpanded-macro-in-package-options/489570#489570
    % so if you need to write macro/valued key, edit instead the
    % style "/prAtEnd/global custom defaults" or
    % "local /prAtEnd/custom defaults" for local changes instead.
    \pratendOptconf%
  },
  global custom defaults/.style={
    %% you can put in this style any global defaults
    %% that should overwrite the usual defaults.
  },
  local custom defaults/.style={
    %% you can put in this style any overwrite of the defaults
    %% that should be "local" and changed over the file, like
    % the category for a given section.
  },
  all defaults/.style={
    %% Load all the style that sets the default values
    bare defaults,
    configuration options,
    global custom defaults,
    local custom defaults,
  },
}

\newcounter{counterAllProofEnd}
\stepcounter{counterAllProofEnd}

\NewDocumentEnvironment{theoremEndRestateBefore}{mO{}m+b}{
  \stepcounter{counterAllProofEnd}%
  \protected@edef\currcounterval{\roman{counterAllProofEnd}}
  \protected@edef\temprest{\noexpand\begin{restatable*}[#2]{#1}{prAtEndRestate\currcounterval}\noexpand\label{thm:prAtEnd\currcounterval}}%
  \expandafter\protected@xdef\csname #3\endcsname{\currcounterval}%
  \temprest%
    #4%
  \end{restatable*}%
}{}

\NewDocumentEnvironment{theoremEnd}{O{}mO{}+b}{
  % The first facultative argument will be the options: type of proof you want, the file to which you want to write...
  % The first mandatory option is the type of the theorem (thm,lemma,...)
  % The second facultative argument will be the title
  % the second mandatory option is the theorem (will \label inside eventually)
  % the last mandatory option is the proof
  \global\def\pratendlastoptions{#1}%
  \pgfkeys{%
    /prAtEnd/.cd,
    all defaults,
    #1
  }%
  \stepcounter{counterAllProofEnd}%
  %% Pre-expand the restatable environment. Need protected
  %% otherwise can't have $\mathtt{G}$ in the title 
  \protected@edef\temprest{\noexpand\begin{restatable}[#3]{#2}{prAtEndRestate\roman{counterAllProofEnd}}}%
  %% Create the file if it's the first time
  \eraseIfNeeded{\category}%
  %% If the theorem must be written here:
  \unless\ifallattheend%
    %% Restate the theorem if it was stated before:
    \ifrestatedbefore%
      \protected@xdef\pratendcountercurrent{\csname #4\endcsname} % Store the current (alpha value of the) counter
      \csname prAtEndRestate\pratendcountercurrent\endcsname % Restate the theorem
    \fi%
    %%  Otherwise just state the theorem in a restatable environment
    \unless\ifrestatedbefore%
      \temprest%
        \label{thm:prAtEnd\roman{counterAllProofEnd}}%
        #4%
      \end{restatable}%
      %% Store the current (alpha value of the) counter
      %% in \pratendcountercurrent
      \protected@xdef\pratendcountercurrent{\roman{counterAllProofEnd}} %
    \fi%
    %% Create a custom alias to restate the theorem
    \expandafter\protected@xdef\csname \pratendcustomrestate\endcsname{\noexpand\csname prAtEndRestate\pratendcountercurrent\endcsname}%
    %% Restate the theorem if needed in appendix
    \ifrestatethm
      \appendtofile{\prefixPrAtEndFiles\category.tex}{\string\prAtEndRestate\pratendcountercurrent*}
    \fi%
  \fi%  
  %% If the theorem is not stated in the main text,
  %% write it at the end 
  \ifallattheend%
    %% Store the current (alpha value of the) counter
    %% in \pratendcountercurrent
    \protected@xdef\pratendcountercurrent{\roman{counterAllProofEnd}}
    %% Create a custom alias to restate the theorem
    \expandafter\protected@xdef\csname \pratendcustomrestate\endcsname{\noexpand\csname prAtEndRestate\pratendcountercurrent\endcsname}%
    \appendtofile{\prefixPrAtEndFiles\category.tex}{\string\begin{restatable}[#3]{#2}{prAtEndRestate\pratendcountercurrent}\string\label{thm:prAtEnd\pratendcountercurrent}\detokenize{#4}\string\end{restatable}}%
  \fi%
}{}

\NewDocumentEnvironment{proofEnd}{O{}+b}{
  \pgfkeys{%
    /prAtEnd/.cd,
    all defaults,
    prAtEndTmpStyle/.style/.expand once={\pratendlastoptions},
    prAtEndTmpStyle,
    #1
  }%
  \unless\ifallattheend
    %% Write eventually a link to the proof
    \iflinktoproof%
      \pratendtextlink{}%
    \fi%
    %% And eventually the proof
    \ifproofhere%
      \begin{proof}%
        #2%
      \end{proof}%
    \fi%
  \fi%
  %% Write the proof at the end
  \ifproofend%
    \appendtofile{\prefixPrAtEndFiles\category.tex}{\string\label{proofsection:prAtEnd\pratendcountercurrent}\string\begin{proof}[\pratendtextproof]\string\phantomsection\string\label{proof:prAtEnd\pratendcountercurrent}\detokenize{#2}\string\end{proof}}%
  \fi%
}{}


%%%%% Text in appendix

\NewDocumentEnvironment{textAtEnd}{O{}+b}{
  % Use it to put normal text in Appendix.
  \pgfkeys{
    /prAtEnd/.cd,
    all defaults,
    #1
  }%
  \ifboth%
  #2%
  \fi%
  \eraseIfNeeded{\category}%
  \appendtofile{\prefixPrAtEndFiles\category.tex}{\detokenize{#2}}%
}{}

\NewDocumentCommand\textEnd{O{}+m}{%
  \begin{textAtEnd}[#1]%
    #2%
  \end{textAtEnd}%
}

\NewDocumentCommand\printProofs{O{defaultcategory}}{
  \input{\prefixPrAtEndFiles#1.tex}
}


%%% You can easily modify the defaults:
% \pgfkeys{/prAtEnd/custom defaults/.style={
%     category=greattheorem
%   }
% }
%%% Or create new styles to apply:
% \pgfkeys{/prAtEnd/great category/.style={
%     category=greattheorem
%   }
% }


\endinput
%\iffalse
%    \begin{macrocode}
%</proof-at-the-end.sty>
%    \end{macrocode}
%\fi
%\Finale
\endinput
