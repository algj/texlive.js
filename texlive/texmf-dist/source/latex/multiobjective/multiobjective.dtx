%\iffalse meta-comment
% File: multiobjective.dtx
% Copyright (C) 2008 by Luis Marti (luis dot marti at uc3m dot es)
% http://www.giaa.inf.uc3m.es/miembros/lmarti
%
% Version of: $Date: 2008-08-18 15:45:14 +0200 (lun, 18 ago 2008) $
% 
% This file may be distributed and/or modified under the 
% conditions of the LaTeX Project Public License, either 
% version 1.2 of this license or (at your option) any later 
% version. The latest version of this license is in: 
% 
% http://www.latex-project.org/lppl.txt 
% 
% and version 1.2 or later is part of all distributions of 
% LaTeX version 1999/12/01 or later. 
% 
% This class is distributed in the hope that it will be useful, but WITHOUT 
% ANY WARRANTY; without even the implied warranty of MERCHANTABILITY or 
% FITNESS FOR A PARTICULAR PURPOSE.
%
%\fi 
% 
%\iffalse 
%<package>\NeedsTeXFormat{LaTeX2e}[1999/12/01]
%<package>\ProvidesPackage{multiobjective}[2008/08/19 v1.0 multiobjective optimization symbols]
%<package>\RequirePackage{amssymb}
%<*driver>
\documentclass{ltxdoc}
\usepackage{multiobjective}
\usepackage{hyperref}
\usepackage[utf8]{inputenc}
\EnableCrossrefs
\PageIndex
\begin{document}
\DocInput{multiobjective.dtx}
\end{document}
%</driver>
% \fi
% \CheckSum{0}
%
% \changes{v1.0}{2008/08/19}{Initial version} 
% \GetFileInfo{multiobjective.sty} 

%\title{\texttt{multiobjective.sty}\\A \LaTeX package for multiobjective optimization 
%and multicriteria decision making\thanks{This document 
% corresponds to \textsf{multiobjective}~\fileversion, 
% dated~\filedate.}} 
%
% \author{Luis Mart\'\i\\
%       \small Universidad Carlos III de Madrid\\
%       \small \href{mailto:luis dot marti at uc3m dot es}{luis dot marti at uc3m dot es}\\
%       \small \href{http://www.giaa.inf.uc3m.es/miembros/lmarti}{http://www.giaa.inf.uc3m.es/miembros/lmarti}}
% \date{Version of $ $Date: 2008-08-18 15:45:14 +0200 (lun, 18 ago 2008) $ $}
%\maketitle
%
%\begin{abstract}
%	This package provides a series of operators commonly used in papers related to multiobjective optimisation, 
%   multiobjective evolutionary algorithms, multicriteria decision making and similar fields.
%\end{abstract}
%
%\section{Motivation}
% \LaTeX does not explicitly includes the operators used in the fields related to multiobjective optimisation. Therefore,
% the appearance of those operators changes from paper to paper, sometimes leading to misunderstandings. The spirit behind 
% this small package is to eliminate those variations.
%
%\section{Dominance operators}
% The operators contained in the package are summarized on table \ref{tbl:operators}. Their formal definition can 
% be found in \cite{coello-2007:eas-solving-mops,Deb01e,ehrgott-2005:multicrit-opt}.
%\begin{table}[h]
%\caption{Dominance operators}
%\begin{center}
%\begin{tabular}{rlc}
%\hline
%\textbf{Operator} & \textbf{Command} & \textbf{Sample} \\
%\hline
%dominance & |\dom| & $\vec{x}\dom\vec{y}$ \\
%neg. dominance & |\negdom| & $\vec{x}\negdom\vec{y}$ \\
%\hline
%weak dominance & |\weakdom| & $\vec{x}\weakdom\vec{y}$ \\
%neg. weak dominance & |\negweakdom| & $\vec{x}\negweakdom\vec{y}$ \\
%\hline
%strict dominance & |\strictdom| & $\vec{x}\strictdom\vec{y}$ \\
%%neg. strict dominance & |\negstrictdom| & $\vec{x}\negstrictdom\vec{y}$ \\
%\hline
%multiplicative $\epsilon$--dominance & |\multepsilondom| & $\vec{x}\multepsilondom\vec{y}$ \\
%additive $\epsilon$--dominance & |\addiepsilondom| & $\vec{x}\addiepsilondom\vec{y}$ \\
%\hline
%better & |\better| & $\set{A}\better\set{B}$ \\
%\hline
%\end{tabular}
%\end{center}
%\label{tbl:operators}
%\end{table}%
%
%\section{Support definitions}
%
% Some extra features are included in addition to the previous operators, as shown on table \ref{tbl:extras}. The |\argmin| and |\argmax|
% operators are babel--ready.
% 
%\begin{table}[ht]
%\caption{Extra definitions}
%\begin{center}
%\begin{tabular}{rlc}
%\hline
%\textbf{Description}   & \textbf{Command} & \textbf{Sample} \\
%\hline
%|\vec| redefinition      & |\vec|\marg{x} & $\vec{x}$ \\
%sets                             & |\set|\marg{A} & $\set{A}$ \\
%$\argmin$ operator  & |\argmin_{sub=1\dots n} {F_i}| & $\argmin_{sub=1\dots n} {F_i}$ \\
%$\argmax$ operator  & |\argmax_{sub=1\dots n} {F_i}| & $\argmax_{sub=1\dots n} {F_i}$ \\
%\hline
%\end{tabular}
%\end{center}
%\label{tbl:extras}
%\end{table}
%
%\begin{thebibliography}{3}
%\bibitem[{1}]{coello-2007:eas-solving-mops}
%Coello~Coello, C.~A., Lamont, G.~B., \& Van~Veldhuizen, D.~A. (2007).
%\newblock {\em Evolutionary Algorithms for Solving Multi-Objective Problems\/},
%   (2nd ed.).
%\newblock Genetic and Evolutionary Computation. New York: Springer.
%
%\bibitem[{2}]{Deb01e}
%Deb, K. (2001).
%\newblock {\em Multi-{O}bjective {O}ptimization using {E}volutionary
%  {A}lgorithms\/}.
%\newblock Chichester, UK: John Wiley \& Sons.
%\newblock {ISBN} 0-471-87339-X.
%
%\bibitem[{3}]{ehrgott-2005:multicrit-opt}
%Ehrgott, M. (2005).
%\newblock {\em Multicriteria Optimization\/}.
%\newblock vol. 491 of {\em Lecture Notes in Economics and Mathematical
%  Systems\/}.
%\newblock Springer.
%\end{thebibliography}
%\StopEventually {\PrintIndex}

\providecommand{\dom}{\prec}
\providecommand{\negdom}{\not\prec}
\providecommand{\weakdom}{\preccurlyeq}
\providecommand{\negweakdom}{\not\preccurlyeq}
\providecommand{\strictdom}{\prec\!\!\!\prec}
\providecommand{\negstrictdom}{\not\prec\!\!\!\prec}
\providecommand{\multepsilondom}{\preccurlyeq_{\epsilon\cdot}}
\providecommand{\addiepsilondom}{\preccurlyeq_{\epsilon +}}
\providecommand{\better}{\triangleleft}

\def\vec#1{\mathchoice{\mbox{\boldmath$\displaystyle#1$}} {\mbox{\boldmath$\textstyle#1$}}
{\mbox{\boldmath$\scriptstyle#1$}} {\mbox{\boldmath$\scriptscriptstyle#1$}}}

\providecommand{\set}[1]{
\mathchoice{\mbox{$\displaystyle\mathcal{#1}$}}
{\mbox{$\textstyle\mathcal{#1}$}} {\mbox{$\scriptstyle\mathcal{#1}$}}
{\mbox{$\scriptscriptstyle\mathcal{#1}$}}
}

\def\argmax{\mathop{{\rm arg}\,\max}}
\def\argmin{\mathop{{\rm arg}\,\min}}
%\Finale 
\endinput 