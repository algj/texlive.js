% \iffalse meta-comment
%
%% File: l3backend-basics.dtx
%
% Copyright (C) 2019 The LaTeX3 Project
%
% It may be distributed and/or modified under the conditions of the
% LaTeX Project Public License (LPPL), either version 1.3c of this
% license or (at your option) any later version.  The latest version
% of this license is in the file
%
%    https://www.latex-project.org/lppl.txt
%
% This file is part of the "l3backend bundle" (The Work in LPPL)
% and all files in that bundle must be distributed together.
%
% -----------------------------------------------------------------------
%
% The development version of the bundle can be found at
%
%    https://github.com/latex3/latex3
%
% for those people who are interested.
%
%<*driver>
\documentclass[full,kernel]{l3doc}
\begin{document}
  \DocInput{\jobname.dtx}
\end{document}
%</driver>
% \fi
%
% \title{^^A
%   The \textsf{l3backend-basics} package\\ Backend basics^^A
% }
%
% \author{^^A
%  The \LaTeX3 Project\thanks
%    {^^A
%      E-mail:
%        \href{mailto:latex-team@latex-project.org}
%          {latex-team@latex-project.org}^^A
%    }^^A
% }
%
% \date{Released 2019-11-25}
%
% \maketitle
%
% \begin{documentation}
%
% \end{documentation}
%
% \begin{implementation}
%
% \section{\pkg{l3backend-basics} Implementation}
%
%    \begin{macrocode}
%<*initex|package>
%    \end{macrocode}
%
% Whilst there is a reasonable amount of code overlap between backends,
% it is much clearer to have the blocks more-or-less separated than run
% in together and DocStripped out in parts. As such, most of the following
% is set up on a per-backend basis, though there is some common code (again
% given in blocks not interspersed with other material).
%
% All the file identifiers are up-front so that they come out in the right
% place in the files.
%    \begin{macrocode}
%<*package>
\ProvidesExplFile
%<*dvipdfmx>
  {l3backend-dvipdfmx.def}{2019-04-06}{}
  {L3 backend support: dvipdfmx}
%</dvipdfmx>
%<*dvips>
  {l3backend-dvips.def}{2019-04-06}{}
  {L3 backend support: dvips}
%</dvips>
%<*dvisvgm>
  {l3backend-dvisvgm.def}{2019-04-06}{}
  {L3 backend support: dvisvgm}
%</dvisvgm>
%<*pdfmode>
  {l3backend-pdfmode.def}{2019-04-06}{}
  {L3 backend support: PDF mode}
%</pdfmode>
%<*xdvipdfmx>
  {l3backend-xdvipdfmx.def}{2019-04-06}{}
  {L3 backend support: xdvipdfmx}
%</xdvipdfmx>
%</package>
%    \end{macrocode}
%
% The order of the backend code here is such that we get somewhat logical
% outcomes in terms of code sharing whilst keeping things readable. (Trying to
% mix all of the code by concept is almost unmanageable.) The key parts which
% are shared are
% \begin{itemize}
%   \item Color support is either \texttt{dvips}-like or \texttt{pdfmode}-like.
%   \item \texttt{pdfmode} and \texttt{(x)dvipdfmx} share drawing routines.
%   \item \texttt{xdvipdfmx} is largely the same as \texttt{dvipdfmx} so
%     takes most of the same code.
% \end{itemize}
%
% \begin{macro}
%   {
%     \__kernel_backend_literal:e,
%     \__kernel_backend_literal:n,
%     \__kernel_backend_literal:x
%   }
%  The one shared function for all backends is access to the basic
%  \tn{special} primitive: it has slightly odd expansion behaviour
%  so a wrapper is provided.
%    \begin{macrocode}
\cs_new_eq:NN \__kernel_backend_literal:e \tex_special:D
\cs_new_protected:Npn \__kernel_backend_literal:n #1
  { \__kernel_backend_literal:e { \exp_not:n {#1} } }
\cs_generate_variant:Nn \__kernel_backend_literal:n { x }
%    \end{macrocode}
% \end{macro}
%
% \subsection{\texttt{dvips} backend}
%
%    \begin{macrocode}
%<*dvips>
%    \end{macrocode}
%
% \begin{macro}
%   {\__kernel_backend_literal_postscript:n, \__kernel_backend_literal_postscript:x}
%   Literal PostScript can be included using a few low-level formats. Here,
%   we use the form with no positioning: this is overall more convenient as
%   a wrapper. Note that this does require that where position is important,
%   an appropriate wrapper is included.
%    \begin{macrocode}
\cs_new_protected:Npn \__kernel_backend_literal_postscript:n #1
  { \__kernel_backend_literal:n { ps:: #1 } }
\cs_generate_variant:Nn \__kernel_backend_literal_postscript:n { x }
%    \end{macrocode}
% \end{macro}
%
% \begin{macro}{\__kernel_backend_postscript:n, \__kernel_backend_postscript:x}
%   PostScript data that does have positioning, and also applying
%   a shift to |SDict| (which is not done automatically by
%   |ps:| or |ps::|, in contrast to |!| or |"|).
%    \begin{macrocode}
\cs_new_protected:Npn \__kernel_backend_postscript:n #1
  { \__kernel_backend_literal:n { ps: SDict ~ begin ~ #1 ~ end } }
\cs_generate_variant:Nn \__kernel_backend_postscript:n { x }
%    \end{macrocode}
% \end{macro}
%
% PostScript for the header: a small saving but makes the code clearer.
% This is held until the start of shipout such that a document with no
% actual output does not write anything.
%    \begin{macrocode}
\cs_if_exist:NTF \AtBeginDvi
  { \exp_not:N \AtBeginDvi }
  { \use:n }
    { \__kernel_backend_literal:n { header = l3backend-dvips.pro } }
%    \end{macrocode}
%
% \begin{macro}
%   {
%     \__kernel_backend_align_begin:,
%     \__kernel_backend_align_end:
%   }
%   In \texttt{dvips} there is no built-in saving of the current
%   position, and so some additional PostScript is required to set up the
%   transformation matrix and also to restore it afterwards. Notice the use
%   of the stack to save the current position \enquote{up front} and to
%   move back to it at the end of the process. Notice that the |[begin]|/^^A
%   |[end]| pair here mean that we can use a run of PostScript statements
%   in separate lines: not \emph{required} but does make the code and
%   output more clear.
%    \begin{macrocode}
\cs_new_protected:Npn \__kernel_backend_align_begin:
  {
    \__kernel_backend_literal:n { ps::[begin] }
    \__kernel_backend_literal_postscript:n { currentpoint }
    \__kernel_backend_literal_postscript:n { currentpoint~translate }
  }
\cs_new_protected:Npn \__kernel_backend_align_end:
  {
    \__kernel_backend_literal_postscript:n { neg~exch~neg~exch~translate }
    \__kernel_backend_literal:n { ps::[end] }
  }
%    \end{macrocode}
% \end{macro}
%
% \begin{macro}{\__kernel_backend_scope_begin:, \__kernel_backend_scope_end:}
%   Saving/restoring scope for general operations needs to be done with
%   \texttt{dvips} positioning (try without to see this!). Thus we need the
%   |ps:| version of the special here. As only the graphics state is ever
%   altered within this pairing, we use the lower-cost |g|-versions.
%    \begin{macrocode}
\cs_new_protected:Npn \__kernel_backend_scope_begin:
  { \__kernel_backend_literal:n { ps:gsave } }
\cs_new_protected:Npn \__kernel_backend_scope_end:
  { \__kernel_backend_literal:n { ps:grestore } }
%    \end{macrocode}
% \end{macro}
%
%    \begin{macrocode}
%</dvips>
%    \end{macrocode}
%
% \subsection{\texttt{pdfmode} backend}
%
%    \begin{macrocode}
%<*pdfmode>
%    \end{macrocode}
%
% The direct PDF backend covers both \pdfTeX{} and \LuaTeX{}. The latter
% renames and restructures the backend primitives but this can be handled
% at one level of abstraction. As such, we avoid using two separate backends
% for this material at the cost of some \texttt{x}-type definitions to get
% everything expanded up-front.
%
% \begin{macro}{\__kernel_backend_literal_pdf:n, \__kernel_backend_literal_pdf:x}
%   This is equivalent to \verb|\special{pdf:}| but the engine can
%   track it. Without the \texttt{direct} keyword everything is kept in
%   sync: the transformation matrix is set to the current point automatically.
%   Note that this is still inside the text (\texttt{BT} \dots \texttt{ET}
%   block).
%    \begin{macrocode}
\cs_new_protected:Npx \__kernel_backend_literal_pdf:n #1
  {
    \cs_if_exist:NTF \tex_pdfextension:D
      { \tex_pdfextension:D literal }
      { \tex_pdfliteral:D }
        { \exp_not:N \exp_not:n {#1} }
  }
\cs_generate_variant:Nn \__kernel_backend_literal_pdf:n { x }
%    \end{macrocode}
% \end{macro}
%
% \begin{macro}{\__kernel_backend_literal_page:n}
%  Page literals are pretty simple. To avoid an expansion, we write out
%  by hand.
%    \begin{macrocode}
\cs_new_protected:Npx \__kernel_backend_literal_page:n #1
  {
    \cs_if_exist:NTF \tex_pdfextension:D
      { \tex_pdfextension:D literal ~ }
      { \tex_pdfliteral:D }
        page
        { \exp_not:N \exp_not:n {#1} }
  }
%    \end{macrocode}
% \end{macro}
%
% \begin{macro}{\__kernel_backend_scope_begin:, \__kernel_backend_scope_end:}
%  Higher-level interfaces for saving and restoring the graphic state.
%    \begin{macrocode}
\cs_new_protected:Npx \__kernel_backend_scope_begin:
  {
    \cs_if_exist:NTF \tex_pdfextension:D
      { \tex_pdfextension:D save \scan_stop: }
      { \tex_pdfsave:D }
  }
\cs_new_protected:Npx \__kernel_backend_scope_end:
  {
    \cs_if_exist:NTF \tex_pdfextension:D
      { \tex_pdfextension:D restore \scan_stop: }
      { \tex_pdfrestore:D }
  }
%    \end{macrocode}
% \end{macro}
%
% \begin{macro}{\__kernel_backend_matrix:n, \__kernel_backend_matrix:x}
%   Here the appropriate function is set up to insert an affine matrix
%   into the PDF. With \pdfTeX{} and \LuaTeX{} in direct PDF output mode there
%   is a primitive for this, which only needs the rotation/scaling/skew part.
%    \begin{macrocode}
\cs_new_protected:Npx \__kernel_backend_matrix:n #1
  {
    \cs_if_exist:NTF \tex_pdfextension:D
      { \tex_pdfextension:D setmatrix }
      { \tex_pdfsetmatrix:D }
        { \exp_not:N \exp_not:n {#1} }
  }
\cs_generate_variant:Nn \__kernel_backend_matrix:n { x }
%    \end{macrocode}
% \end{macro}
%
%    \begin{macrocode}
%</pdfmode>
%    \end{macrocode}
%
% \subsection{\texttt{dvipdfmx} backend}
%
%    \begin{macrocode}
%<*dvipdfmx|xdvipdfmx>
%    \end{macrocode}
%
% The \texttt{dvipdfmx} shares code with the PDF mode one (using the common
% section to this file) but also with \texttt{xdvipdfmx}. The latter is close
% to identical to \texttt{dvipdfmx} and so all of the code here is extracted
% for both backends, with some \texttt{clean up} for \texttt{xdvipdfmx} as
% required.
%
% \begin{macro}{\__kernel_backend_literal_pdf:n, \__kernel_backend_literal_pdf:x}
%   Equivalent to \texttt{pdf:content} but favored as the link to
%   the \pdfTeX{} primitive approach is clearer.
%    \begin{macrocode}
\cs_new_protected:Npn \__kernel_backend_literal_pdf:n #1
  { \__kernel_backend_literal:n { pdf:literal~ #1 } }
\cs_generate_variant:Nn \__kernel_backend_literal_pdf:n { x }
%    \end{macrocode}
% \end{macro}
%
% \begin{macro}{\__kernel_backend_literal_page:n}
%  Whilst the manual says this is like |literal direct| in \pdfTeX{},
%  it closes the |BT| block!
%    \begin{macrocode}
\cs_new_protected:Npn \__kernel_backend_literal_page:n #1
  { \__kernel_backend_literal:n { pdf:literal~direct~ #1 } }
%    \end{macrocode}
% \end{macro}
%
% \begin{macro}{\__kernel_backend_scope_begin:, \__kernel_backend_scope_end:}
%   Scoping is done using the backend-specific specials.
%    \begin{macrocode}
\cs_new_protected:Npn \__kernel_backend_scope_begin:
  { \__kernel_backend_literal:n { x:gsave } }
\cs_new_protected:Npn \__kernel_backend_scope_end:
  { \__kernel_backend_literal:n { x:grestore } }
%    \end{macrocode}
% \end{macro}
%
%    \begin{macrocode}
%</dvipdfmx|xdvipdfmx>
%    \end{macrocode}
%
% \subsection{\texttt{dvisvgm} backend}
%
%    \begin{macrocode}
%<*dvisvgm>
%    \end{macrocode}
%
% \begin{macro}{\__kernel_backend_literal_svg:n, \__kernel_backend_literal_svg:x}
%   Unlike the other backends, the requirements for making SVG files mean
%   that we can't conveniently transform all operations to the current point.
%   That makes life a bit more tricky later as that needs to be accounted for.
%   A new line is added after each call to help to keep the output readable
%   for debugging.
%    \begin{macrocode}
\cs_new_protected:Npn \__kernel_backend_literal_svg:n #1
  { \__kernel_backend_literal:n { dvisvgm:raw~ #1 { ?nl } } }
\cs_generate_variant:Nn \__kernel_backend_literal_svg:n { x }
%    \end{macrocode}
% \end{macro}
%
% \begin{macro}{\__kernel_backend_scope_begin:, \__kernel_backend_scope_end:}
%   A scope in SVG terms is slightly different to the other backends as
%   operations have to be \enquote{tied} to these not simply inside them.
%    \begin{macrocode}
\cs_new_protected:Npn \__kernel_backend_scope_begin:
  { \__kernel_backend_literal_svg:n { <g> } }
\cs_new_protected:Npn \__kernel_backend_scope_end:
  { \__kernel_backend_literal_svg:n { </g> } }
%    \end{macrocode}
% \end{macro}
%
% \begin{macro}{\__kernel_backend_scope_begin:n, \__kernel_backend_scope_begin:x}
%   In SVG transformations, clips and so on are attached directly to scopes so
%   we need a way or allowing for that. This is rather more useful than
%   \cs{__kernel_backend_scope_begin:} as a result.
%   No assumptions are made about the nature
%   of the scoped operation(s).
%    \begin{macrocode}
\cs_new_protected:Npn \__kernel_backend_scope_begin:n #1
  { \__kernel_backend_literal_svg:n { <g~ #1 > } }
\cs_generate_variant:Nn \__kernel_backend_scope_begin:n { x }
%    \end{macrocode}
% \end{macro}
%
%    \begin{macrocode}
%</dvisvgm>
%    \end{macrocode}
%
%    \begin{macrocode}
%</initex|package>
%    \end{macrocode}
%
% \end{implementation}
%
% \PrintIndex
