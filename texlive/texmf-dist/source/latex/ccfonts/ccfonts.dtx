%\CheckSum{144}
%
% \iffalse meta comment
%
% Copyright (c) Walter Schmidt 1997--2000
%
% This program may be distributed and/or modified under the
% conditions of the LaTeX Project Public License, either version 1.2
% of this license or (at your option) any later version.
% The latest version of this license is in
%   http://www.latex-project.org/lppl.txt
% and version 1.2 or later is part of all distributions of LaTeX
% version 1999/12/01 or later.
%
% This program consists of the files ccfonts.dtx, cc.fdd and
% ccfonts.ins.
%
% \fi
%
% \iffalse
%<package>\NeedsTeXFormat{LaTeX2e}[1995/06/01]
%<*driver>
\ProvidesFile{ccfonts.drv}
%</driver>
%<package>\ProvidesPackage{ccfonts}
           [2000/06/30 v1.1 (WaS)]     
%
%<*driver> 
\documentclass[11pt]{ltxdoc}
\usepackage{mflogo,url}
\CodelineNumbered
\begin{document}
 \DocInput{ccfonts.dtx}
\end{document}
%</driver>
% \fi
%
% \GetFileInfo{ccfonts.drv}
% \DeleteShortVerb{\|}
% \MakeShortVerb{\+}
% 
% \title{The \LaTeXe{} package \texttt{ccfonts}}
% \author{Walter Schmidt\thanks{\texttt{walter.schmidt@arcormail.de}}}
% \date{(\fileversion{} -- \filedate)}
% \maketitle
% \tableofcontents
%
% \section{Prerequisites}
% In order to make use of the package +ccfonts+, the following fonts
% and +.fd+ files are required:
% \begin{itemize}
% \item The Concrete text fonts with traditional encoding
%   (CTAN: \path{fonts/concrete/})
% \item The Concrete text fonts with European encoding
%   (CTAN: \path{fonts/ecc/})
% \item The mathematical Concrete fonts
%   (CTAN: \path{fonts/concmath/})
% \item The +.fd+ files for the traditional and mathematical
%   Concrete fonts (CTAN: \path{macros/latex/contrib/supported/concmath/})
% \item The +.fd+ files for the European Concrete fonts, 
%   which are distributed and installed in conjunction
%   with the +ccfonts+ package
% \end{itemize}
% On CTAN the fonts are available in \MF{} format.
% The Concrete typefaces are also provided in Type1 format from
% Micropress Inc, see
% \path{<http://www.micropress-inc.com>}.
%
%
% \section{Using the package}
% The \LaTeX{} macro package +ccfonts+ 
% supports typesetting with the font family `Concrete'.
% Loading this package through
% \begin{verse}
% +\usepackage{ccfonts}+
% \end{verse}
% will effect the following:
% \begin{itemize}
% \item The default roman font family is changed to \texttt{ccr},
% i.e.\ Concrete.
% \item The default leading (+\baselineskip+) for the font sizes 8--12\,pt
% is increased slightly.
% \item The `Concrete' fonts are used in math mode, too.
% \item The packages +amsfonts+ or +amssymb+, if loaded  additionally,
% will use the Concrete versions of the AMS symbol fonts.
%
% Notice that you may still have to specify the option +psamsfonts+
% for these packages, so as to prevent them from using design sizes
% of the Euler Fraktur fonts, which may be 
% unavailable within your TeX system; this works flawlessly with
% version 1.1 of the +ccfonts+ package now. (You need not care for this
% subject, unless Euler Fraktur is actually used.)
% \end{itemize}
%
% \subsection{Package options}
% \newcommand{\oitem}[1]{\item[\texttt{#1}]}
% \begin{description}
%   \oitem{boldsans} The semibold series of CM Sans is used
%   as a replacement for the missing bold series of Concrete.
%   (The default behaviour is to use the bold extended version of 
%   CM Roman.)
%   \oitem{standard-baselineskips} disables the increased leading.
%   This can be useful, e.g., when typesetting in narrow columns.
%   \oitem{exscale} implements scaling of the math extension font.
%   For a discussion of this feature see the file +exscale.dtx+. 
%   \oitem{slantedGreek} makes uppercase Greek letters slanted by default.
%   Regardless of this option, the new commands +\upDelta+ and +\upOmega+
%   will always produce an upright \( \Delta \) and \( \Omega \).
% \end{description}
%
% \subsection{Font encoding}
% The package does \emph{not} change the default output font
% encoding from OT1.  Switching to the extended T1 and TS1 encodings
% needs the following additional commands:
% \begin{verse}
% +\usepackage[T1]{fontenc}+\\
% +\usepackage{textcomp}+
% \end{verse}
%
% \section{Known problems}
% \begin{itemize}
% \item
% There are no bold math fonts available.
% \item
% In order to enlarge the default +\baselineskip+, the size-changing
% macros have been redefined, and they are no longer as robust as the 
% original definitions.  This may result in \LaTeX\ errors with
% `moving arguments'.  As a workaround, you may protect any font-related
% commands in moving arguments with a +\protect+ command.  In case this
% does not help, the package should be loaded with the option
% \textsf{standard-baselineskips} which will prevent the commands from
% being redefined;  you will, however, have to care for an appropriate
% line spacing by other means then.
% \end{itemize}
%
% \section{NFSS classification of the Concrete typefaces}
% \begin{center}
%   \begin{tabular}{|l|l|l|l|}
%   \hline
%   \textbf{encoding} & \textbf{family} & \textbf{series} & \textbf{shape(s)}\\
%   \hline\hline
%   \multicolumn{4}{|c|}{\textit{Concrete}}\\ \hline
%   OT1, T1, TS1 & ccr & m & n, sl, it, sc \\ \hline
%   OT1          & ccr & c & sl\\ \hline \hline
%   \multicolumn{4}{|c|}{\textit{Concrete Math}}\\ \hline
%   OML          & ccm  & m & it \\ \hline \hline
%   OMS          & ccsy & m & n  \\ \hline
%   OMX          & ccex & m & n  \\ \hline \hline
%   \multicolumn{4}{|c|}{\textit{Concrete AMS A, B}}\\ \hline
%   U          & msa & m & n \\ \hline
%   U          & msb & m & n \\ \hline 
%   \end{tabular}
% \end{center}
% Notice, that 
% \begin{itemize}
%   \item the series c (condensed) is available as slanted
%   and with a font size of 9\,pt only;
%   \item the Concrete AMS fonts are only defined 
%   through the package \textsf{ccfonts},
%   i.e., there are no related \texttt{.fd} files.
% \end{itemize}
%
% \StopEventually{}
%
% \section{Implementation}
% \subsection{Font setup for text mode}
% We make \texttt{ccr} the default font family:
%    \begin{macrocode}
%<*package>
\renewcommand{\rmdefault}{ccr}
%    \end{macrocode}
%
% The +\baselineskip+ should be larger than with CM Roman.
% In order to overwrite the +\baselineskip+ defined in the commands
% like +\normalsize+, +\small+, etc., we use a trick from Frank Jensen's
% package +beton+.
% First we set up a table containing our +\baselineskip+ values:
%    \begin{macrocode}
\def\cc@baselineskip@table
   {<\@viiipt>10<\@ixpt>11.5<\@xpt>13<\@xipt>14.5<\@xiipt>16}
%    \end{macrocode}
% All the standard \LaTeX\ size-changing commands (+\small+, +\large+,
% etc.)\ are defined in terms of the +\@setfontsize+ macro.  This
% macro is called with the following three arguments: +#1+~is the
% size-changing command; +#2+~is the font size; +#3+~is the
% +\baselineskip+ value.  We modify this macro to check
% the above +\cc@baselineskip@table+ for an alternative +\baselineskip+
% value:
%    \begin{macrocode}
\def\cc@setfontsize#1#2#3%
   {\edef\@tempa{\def\noexpand\@tempb####1<#2}%
    \@tempa>##2<##3\@nil{\def\cc@baselineskip@value{##2}}%
    \edef\@tempa{\noexpand\@tempb\cc@baselineskip@table<#2}%
    \@tempa><\@nil
    \ifx\cc@baselineskip@value\@empty
       \def\cc@baselineskip@value{#3}%
    \fi
    \old@setfontsize{#1}{#2}\cc@baselineskip@value}
%    \end{macrocode}
% Now we redefine +\@setfontsize+:
%    \begin{macrocode}
\let\old@setfontsize=\@setfontsize
\let\@setfontsize=\cc@setfontsize
%    \end{macrocode}
%
% \subsection{Options}
% \subsubsection{Standard leading}
% The +\baselineskip+ values specified in the above table should be
% appropriate for most purposes, i.\,e., for one-column material in the
% normal article/report/book formats.  However, it is sometimes
% desirable to use a smaller value for +\baselineskip+, e.\,g. in two-column
% material.  We therefore provide an option
% to turn off the above automatic mechanism for +\baselineskip+ settings:
%    \begin{macrocode}
\DeclareOption{standard-baselineskips}{%
 \let\@setfontsize=\old@setfontsize}
%    \end{macrocode}
%
% \subsubsection{The option \texttt{exscale}}
% The code is simply copied from \texttt{exscale.sty}, 
% with \texttt{xccex} instead of \texttt{cmex}.
%    \begin{macrocode}
\DeclareOption{exscale}{
\DeclareFontFamily{OMX}{ccex}{}
\DeclareFontShape{OMX}{ccex}{m}{n}{%
<-8>sfixed*xccex7%
<8>xccex8%
<9>xccex9%
<10><10.95><12><14.4><17.28><20.74><24.88>xccex10%
}{}
\newdimen\big@size
\addto@hook\every@math@size{\setbox\z@\vbox{\hbox{$($}\kern\z@}%
   \global\big@size 1.2\ht\z@}
\def\bBigg@#1#2{%
   {\hbox{$\left#2\vcenter to#1\big@size{}\right.\n@space$}}}
\def\big{\bBigg@\@ne}
\def\Big{\bBigg@{1.5}}
\def\bigg{\bBigg@\tw@}
\def\Bigg{\bBigg@{2.5}}
}
%    \end{macrocode}
%
% \subsubsection{The option \texttt{slantedGreek}}
%    \begin{macrocode}
\let\upDelta\Delta
\let\upOmega\Omega
\DeclareOption{slantedGreek}{%
 \DeclareMathSymbol{\Gamma}{\mathalpha}{letters}{0}
 \DeclareMathSymbol{\Delta}{\mathalpha}{letters}{1}
 \DeclareMathSymbol{\Theta}{\mathalpha}{letters}{2}
 \DeclareMathSymbol{\Lambda}{\mathalpha}{letters}{3}
 \DeclareMathSymbol{\Xi}{\mathalpha}{letters}{4}
 \DeclareMathSymbol{\Pi}{\mathalpha}{letters}{5}
 \DeclareMathSymbol{\Sigma}{\mathalpha}{letters}{6}
 \DeclareMathSymbol{\Upsilon}{\mathalpha}{letters}{7}
 \DeclareMathSymbol{\Phi}{\mathalpha}{letters}{8}
 \DeclareMathSymbol{\Psi}{\mathalpha}{letters}{9}
 \DeclareMathSymbol{\Omega}{\mathalpha}{letters}{10}
}
%    \end{macrocode}
%
% \subsection{The option \texttt{boldsans}}
%    \begin{macrocode}
\DeclareOption{boldsans}{%
  \renewcommand{\bfdefault}{sbc}}
%    \end{macrocode}
%
% \subsubsection{Processing options}
% Note that +\old@setfontsize+ must have been defined before!
%    \begin{macrocode}
\ProcessOptions\relax
%    \end{macrocode}
%
% \subsection{Font setup for math mode}
%    \begin{macrocode}
\DeclareSymbolFont  {operators}   {OT1}{ccr} {m}{n}
\DeclareSymbolFont  {letters}     {OML}{ccm} {m}{it}
\DeclareSymbolFont  {symbols}     {OMS}{ccsy}{m}{n}
\DeclareSymbolFont  {largesymbols}{OMX}{ccex}{m}{n}
\DeclareMathAlphabet{\mathbf}     {OT1}{ccr}{bx}{n}
\DeclareMathAlphabet{\mathit}     {OT1}{ccr} {m}{it}
%    \end{macrocode}
%
% In case the package +amsfonts+ is loaded additionally,
% we must ensure that the Concrete versions of the 
% AMS symbol fonts are used.  We execute the font definitions
% AtBeginDocument, so that loading +amsfonts+ with the 
% option +psamsfonts+ cannot do any harm.  Notice that the option may 
% be required for getting the Euler Fraktur fonts right.
%    \begin{macrocode}
\AtBeginDocument{
   \DeclareFontFamily{U}{msa}{}
   \DeclareFontShape{U}{msa}{m}{n}{%
   <5><6><7><8><9><10>gen*xccam%
   <10.95><12><14.4><17.28><20.74><24.88>xccam10}{}
   \DeclareFontFamily{U}{msb}{}
   \DeclareFontShape{U}{msb}{m}{n}{%  
   <5><6><7><8><9><10>gen*xccbm%
   <10.95><12><14.4><17.28><20.74><24.88>xccbm10}{}
}
%    \end{macrocode}
%
% 
% \subsection{Initialization}
% We ensure that any package loaded after \texttt{ccfonts}
% will find the new value of +\baselineskip+.
%    \begin{macrocode}
\normalsize
%</package>
%    \end{macrocode}
%
% The next line of code prevents DocStrip from adding the
% character table to all modules:
%    \begin{macrocode}
\endinput
%    \end{macrocode}
% \Finale
%% \CharacterTable
%%  {Upper-case    \A\B\C\D\E\F\G\H\I\J\K\L\M\N\O\P\Q\R\S\T\U\V\W\X\Y\Z
%%   Lower-case    \a\b\c\d\e\f\g\h\i\j\k\l\m\n\o\p\q\r\s\t\u\v\w\x\y\z
%%   Digits        \0\1\2\3\4\5\6\7\8\9
%%   Exclamation   \!     Double quote  \"     Hash (number) \#
%%   Dollar        \$     Percent       \%     Ampersand     \&
%%   Acute accent  \'     Left paren    \(     Right paren   \)
%%   Asterisk      \*     Plus          \+     Comma         \,
%%   Minus         \-     Point         \.     Solidus       \/
%%   Colon         \:     Semicolon     \;     Less than     \<
%%   Equals        \=     Greater than  \>     Question mark \?
%%   Commercial at \@     Left bracket  \[     Backslash     \\
%%   Right bracket \]     Circumflex    \^     Underscore    \_
%%   Grave accent  \`     Left brace    \{     Vertical bar  \|
%%   Right brace   \}     Tilde         \~}
%%

