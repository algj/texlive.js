% \iffalse 
%
% chmst-pdf.dtx
%
% Copyright (C) 2010 by Shinsaku Fujita  All rights reserved.
%
% ========================================
%
% Modified from the following previous packages: 
% 
% chemist.dtx 
% Copyright (C) 1996, 1999, 2000 by Shinsaku Fujita  All rights reserved
% and 
% chmst-ps.dtx
% Copyright (C) 2002, 2004, 2009, 2010 by Shinsaku Fujita All rights reserved
%
% in order to utilize pgf. 
%
% This file is part of the ChemSci system
% ========================================
%
% This style file is created for submitting a manuscript to 
% scientific journals. This style file is to be contained in the 
% ``chemist'' directory which is an input directory for TeX. 
%
% For using this file, see 
%  Shinsaku Fujita, ``LaTeX for Chemists and Biochemists'' 
%  Tokyo Kagaku Dojin, Tokyo (1993), Chapter 16. 
%
% This work may be distributed and/or modified under the
% conditions of the LaTeX Project Public License, either version 1.3
% of this license or (at your option) any later version.
% The latest version of this license is in
%   http://www.latex-project.org/lppl.txt
% and version 1.3 or later is part of all distributions of LaTeX
% version 2005/12/01 or later.
%
% This work has the LPPL maintenance status `maintained'. 
% The Current Maintainer of this work is Shinsaku Fujita.
%
% This work consists of the files chmst-ps.dtx and chmst-ps.ins
% and the derived file chmst-ps.sty.
%
% Please report any bugs, comments, suggestions, etc. to:
%   Shinsaku Fujita, 
%   Shonan Institute of Chemoinformatics and 
%   Mathematical Chemistry
%   fujita-sicimc@nifmail.jp
%
%=======================================
%
% This file is a successor to:
%
% *********************************************************************
% * chemist.sty <Sept 29 1991> by Shinsaku Fujita                     *
% *  counters and related matters for compounds.                      *
% *  ffboxit: November 2 1991 by S. Fujita                            *
% *  bslskipchange: November 6 1991 by S. Fujita (deleted)            *
% *  chapinitial: 1991 November 7 by S. Fujita                        *
% *  variable arrows: 1992 May 24 by S. Fujita                        *
% *  reaction and scheme arrows (revised): 1992 May 24  by S. Fujita  *
% *  changespace <=== bslskipchange: Dec 31 1992 by S. Fujita         *
% *  (g)rshfboxit and (g)lshfboxit: January 3, 1993 by S. Fujita      *
% *********************************************************************
% %%%%%%%%%%%%%%%%%%%%%%%%%%%%%%%%%%%%%%%%%%%%%%%%%%%%%%%%%%%%%%%%%%%%
% Released on January 3, 1993 
% Copyright (C) 1993 by Shinsaku Fujita, all rights reserved.
% This style file is created for making manuscripts of chemical fields.  
% This option style file is to be contained in the ``chemist'' directory 
% which is an input directory for TeX. 
%
%  %%%%%%%%%%%%%%%%%%%%%%%%%%%%%%%%%%%%%%%%%%%%%%%%%%%%%%%%%%%%%%%%%%%
% \def\j@urnalname{chemist}
% \def\versi@ndate{January 03, 1993}
% \def\versi@nno{ver1.00}
%  %%%%%%%%%%%%%%%%%%%%%%%%%%%%%%%%%%%%%%%%%%%%%%%%%%%%%%%%%%%%%%%%%%%
%
% Version 1.01 
%  --- added the end commands of the \verbatim environment and the 
%      \verb command contained in LaTeX 2.09
%  --- \LaTeX, \BibTeX logos etc improved
% \typeout{verbatim for LaTeX2.09 (and LaTeX2e) in chemist.sty}
% \typeout{logos for LaTeX2.09 (and LaTeX2e) in chemist.sty}
% \def\versi@ndate{April 20, 1996}
% \def\j@urnalname{chemist}
% \def\versi@ndate{April 20, 1996}
% \def\versi@nno{ver1.01}
% \def\copyrighth@lder{SF} % Shinsaku Fujita
% %%%%%%%%%%%%%%%%%%%%%%%%%%%%%%%%%%%%%%%%%%%%%%%%%%%%%%%%%%%%%%%%%%%%
% \def\j@urnalname{chemist} \def\journalID{chemist}
% \def\versi@ndate{June 15, 1996}
% \def\versi@nno{ver1.01a}
% \def\copyrighth@lder{SF} % Shinsaku Fujita
% \typeout{Package `\j@urnalname' (\versi@nno) <\versi@ndate>\space 
% [\copyrighth@lder]}
% \typeout{chemist.sty ver1.01a (for LaTeX2e) 1996/06/15[SF]}
% %%%%%%%%%%%%%%%%%%%%%%%%%%%%%%%%%%%%%%%%%%%%%%%%%%%%%%%%%%%%%%%%%%%%
% \def\j@urnalname{chemist} \def\journalID{chemist}
% \def\versi@ndate{March 16, 1999}
% \def\versi@nno{ver2.00}
% \def\copyrighth@lder{SF} % Shinsaku Fujita
% %%%%%%%%%%%%%%%%%%%%%%%%%%%%%%%%%%%%%%%%%%%%%%%%%%%%%%%%%%%%%%%%%%%%
% \def\j@urnalname{chemist} \def\journalID{chemist}
% \def\versi@ndate{November 3, 2000}
% \def\versi@nno{ver2.00a}
% \def\copyrighth@lder{SF} % Shinsaku Fujita
% %%%%%%%%%%%%%%%%%%%%%%%%%%%%%%%%%%%%%%%%%%%%%%%%%%%%%%%%%%%%%%%%%%%%
% \def\j@urnalname{chmst-ps} \def\journalID{chmst-ps}
% \def\versi@ndate{May 30, 2002}
% \def\versi@nno{ver1.00}
% \def\copyrighth@lder{SF} % Shinsaku Fujita
% %%%%%%%%%%%%%%%%%%%%%%%%%%%%%%%%%%%%%%%%%%%%%%%%%%%%%%%%%%%%%%%%%%%%
% \def\j@urnalname{chmst-ps} \def\journalID{chmst-ps}
% \def\versi@ndate{August 30, 2004}
% \def\versi@nno{ver1.01}
% \def\copyrighth@lder{SF} % Shinsaku Fujita
% %%%%%%%%%%%%%%%%%%%%%%%%%%%%%%%%%%%%%%%%%%%%%%%%%%%%%%%%%%%%%%%%%%%%
% \def\j@urnalname{chmst-ps} \def\journalID{chmst-ps}
% \def\versi@ndate{November 08, 2009}
% \def\versi@nno{ver1.02}
% \def\copyrighth@lder{SF} % Shinsaku Fujita
% %%%%%%%%%%%%%%%%%%%%%%%%%%%%%%%%%%%%%%%%%%%%%%%%%%%%%%%%%%%%%%%%%%%%
% \def\j@urnalname{chmst-ps} \def\journalID{chmst-ps}
% \def\versi@ndate{November 26, 2009}
% \def\versi@nno{ver1.03}
% \def\copyrighth@lder{SF} % Shinsaku Fujita
% %%%%%%%%%%%%%%%%%%%%%%%%%%%%%%%%%%%%%%%%%%%%%%%%%%%%%%%%%%%%%%%%%%%%
%
% 
% \fi
% \CheckSum{1429}
%% \CharacterTable
%%  {Upper-case    \A\B\C\D\E\F\G\H\I\J\K\L\M\N\O\P\Q\R\S\T\U\V\W\X\Y\Z
%%   Lower-case    \a\b\c\d\e\f\g\h\i\j\k\l\m\n\o\p\q\r\s\t\u\v\w\x\y\z
%%   Digits        \0\1\2\3\4\5\6\7\8\9
%%   Exclamation   \!     Double quote  \"     Hash (number) \#
%%   Dollar        \$     Percent       \%     Ampersand     \&
%%   Acute accent  \'     Left paren    \(     Right paren   \)
%%   Asterisk      \*     Plus          \+     Comma         \,
%%   Minus         \-     Point         \.     Solidus       \/
%%   Colon         \:     Semicolon     \;     Less than     \<
%%   Equals        \=     Greater than  \>     Question mark \?
%%   Commercial at \@     Left bracket  \[     Backslash     \\
%%   Right bracket \]     Circumflex    \^     Underscore    \_
%%   Grave accent  \`     Left brace    \{     Vertical bar  \|
%%   Right brace   \}     Tilde         \~}
%
% \iffalse
%%
%% `chmst-ps.dtx' ver1.00 (for LaTeX2e) 2002/05/30
%% `chmst-ps.dtx' ver1.01 (for LaTeX2e) 2004/08/30
%% `chmst-ps.dtx' ver1.02 (for LaTeX2e) 2009/11/08
%% `chmst-ps.dtx' ver1.03 (for LaTeX2e) 2009/11/26
%% `chmst-ps.dtx' ver5.00 (for LaTeX2e) 2010/10/01
%%  by Shinsaku Fujita
%%
%
% \section{Initial declaration}
%
% \changes{v1.00}{2002/05/30}{First Version}
% \changes{v1.01}{2004/08/30}{2nd Version}
% \changes{v1.02}{2009/11/08}{3rd Version}
% \changes{v1.03}{2009/11/26}{4th Version, harpoons}
% \changes{v5.00}{2010/10/01}{For LaTeX Project Public License}
%
%    \begin{macrocode}
% \NeedsTeXFormat{LaTeX2e}
%    \end{macrocode}
%
%    \begin{macrocode}
%    <chmstpdf>\ProvidesFile{chmst-ps.sty}
%<driver>\ProvidesFile{chmst-ps.drv}
%<driver>         [2010/10/01 v5.00
%    <chmstpdf>     ChemSci package ]
%<driver>     ChemSci package driver (English)
%    \end{macrocode}
%
%\setcounter{StandardModuleDepth}{1}
%
% \section{Driver file for this dtx file}
%
%    \begin{macrocode}
%<*driver>
 ]
\documentclass{ltxdoc}
\GetFileInfo{chmst-pdf.drv}
\title{The chmst-pdf Package \space\fileversion \\
(Part of the XyMTeX/ChemSci package v5.00)}
\author{Shinsaku Fujita \\ 
Shonan Institute of Chemoinformatics and 
Mathematical Chemistry \\
Ashigara-Kami-Gun, Kanagawa, 258-0019 Japan 
%% Department of Chemistry and Materials Technology \\
%% Kyoto Institute of Technology \\
%% Matsugasaki, Sakyoku, Kyoto, 606-8585 Japan 
%% (old address)
%% Ashigara Research Laboratories \\
%% Fuji Photo Film Co. Ltd.
}
\date{\filedate}
\begin{document}
\maketitle
\tableofcontents
\DocInput{chmst-pdf.dtx}
\end{document}
%</driver>
%    \end{macrocode}
%
% \fi
%
% \StopEventually{}
%
% \section{{\sc docstrip} options}
%
% This dtx file provides the {\sffamily chmst-pdf} package 
% by docstrip processing. 
%
% \DeleteShortVerb{\|}
% \begin{center}
% \begin{tabular}{|l|l|}
%  \hline
%   argument & package name \\
%  \hline
%   chmstpdf   & chmst-pdf.sty \\
%   driver  & a documentation driver file \\
%  \hline
% \end{tabular}
% \end{center}
% \MakeShortVerb{\|}
% 
% \section{Version information}
%
% The old process for typing out version information remains active. 
%    \begin{macrocode}
%<chmstpdf>\def\j@urnalname{chmst-pdf} \def\journalID{chmst-pdf}
\def\versi@ndate{October 01, 2010}
\def\versi@nno{ver5.00}
\def\copyrighth@lder{SF} % Shinsaku Fujita
%<*chmstpdf>
\typeout{Package `\j@urnalname' (\versi@nno) <\versi@ndate>\space 
[\copyrighth@lder]}
\typeout{chmst-pdf.sty ver5.00 (for LaTeX2e) 2010/10/01[SF]}
%</chmstpdf>
%    \end{macrocode}
%
%    \begin{macrocode}
%<*chmstpdf>
\@ifundefined{if@@@chemtimes}{\newif\if@@@chemtimes}{}
\@ifundefined{if@@chemtimes}{\newif\if@@chemtimes}{}%from chemtimes
\DeclareOption{chemtimes}{\@@@chemtimestrue}
\@@@chemtimesfalse
\ProcessOptions
\if@@@chemtimes\if@@chemtimes\else
  \PackageError{chmst-pdf}
    {The option ``chemtimes'' is not permitted}
    {You should use this option after the chemtimes package is loaded.}
\fi\fi
\if@@@chemtimes
\RequirePackage[chemtimes]{chemist}%
\else
\RequirePackage{chemist}%
\fi
\RequirePackage{xymtx-pdf}%
%    \end{macrocode}
%
% \section{Cross-references of compounds}
%  (Deleted)
% \section{Boxes for placing compounds}
%  (Deleted)
% \section{Arrows}
% \subsection{Arrows with variable length}
%  (Deleted)
% \subsection{Reaction arrows with variable length}
%
% The {\sffamily chmst-pdf} package defines arrows with variable length 
% for drawing chemical equations, where additional information 
% can be written on the upper and/or downward sides of each arrow.  
% \changes{v1.00}{2002/05/30}{First Version for PostScript mode}
% \changes{v5.00}{2010/10/01}{For PDF mode}
%
% Several elements for drawing horizontal reaction arrows within the scope of 
% pstricks package are defined first.  
%
%  \begin{macro}{\rArrow@Element}
%  \begin{macro}{\lArrow@Element}
%  \begin{macro}{\lrArrow@Element}
%  \begin{macro}{\eqArrow@Element}
% \changes{v5.00}{2010/10/01}{Redefined for PDF mode}
%    \begin{macrocode}
\newdimen\@tempdimX
\newdimen\@tempdimY
\def\rArrow@Element#1{\begingroup\hbox to#1{\hss%
\tikz[baseline=(X.base)]{%
\draw[>=stealth,->,line width=0.4pt]%
(0pt,0pt) coordinate (X) (0pt,0.28em)  -- (#1,0.28em);}%
\hss}\endgroup}
\let\Arrow@Element=\rArrow@Element
\def\lArrow@Element#1{\begingroup\hbox to#1{\hss%
\tikz[baseline=(X.base)]{%
\draw[stealth-,line width=0.4pt]%
(0pt,0pt) coordinate (X) (0pt,0.28em)  -- (#1,0.28em);}%
\hss}\endgroup}
\def\lrArrow@Element#1{\begingroup\hbox to#1{\hss%
\tikz[baseline=(X.base)]{%
\draw[stealth-stealth,line width=0.4pt]%
(0pt,0pt) coordinate (X) (0pt,0.28em)  -- (#1,0.28em);}%
\hss}\endgroup}
\def\eqArrow@Element#1{\begingroup\hbox to#1{\hss%
\tikz[baseline=(X.base)]{%
\draw[-stealth,line width=0.4pt]%
(0pt,0.43em)  -- (#1,0.43em);
\draw[stealth-,line width=0.4pt]%
(0pt,0pt) coordinate (X) 
(0pt,0.13em)  -- (#1,0.13em);}%
\hss}\endgroup}
%    \end{macrocode}
%  \end{macro}
%  \end{macro}
%  \end{macro}
%  \end{macro}
%
%  \begin{macro}{\eqHarpoon@Element}
% \changes{v1.03}{2009/11/26}{Harpoons added as PostScript procedures}
% \changes{v5.00}{2010/10/01}{Redefined for PDF mode}
%    \begin{macrocode}
\def\eqHarpoon@Element#1{\begingroup\hbox to#1{\hss%
\tikz[baseline=(X.base)]{%
\draw[-left to,line width=0.4pt]%
(0pt,0.43em)  -- (#1,0.43em);
\draw[left to-,line width=0.4pt]%
(0pt,0pt) coordinate (X) 
(0pt,0.13em)  -- (#1,0.13em);}%
\hss}\endgroup}
%    \end{macrocode}
%  \end{macro}
%
% Equilibirum arrows with uneven lengths are defined by using the \verb/\draw/ command 
% of the pgf package.
% \changes{v1.02}{2009/11/08}{2nd Version fro PostScript mode (due to PSTricks)}
% \changes{v5.00}{2010/10/01}{Redefined for PDF mode (due to pgf)}
%
%  \begin{macro}{\reqArrow@Element}
%  \begin{macro}{\leqArrow@Element}
%    \begin{macrocode}
\def\reqArrow@Element#1{\begingroup%
\dimen1=#1%
\dimen0=\eqlbarrowstretch\dimen1%
\dimen1=#1 \advance\dimen1 by-\dimen0 \divide\dimen1 by2\relax
\dimen2=\dimen1 \advance\dimen2 by\dimen0
\tikz[baseline=(X.base)]{%
\draw[-stealth,line width=0.4pt]%
(0pt,0.43em)  -- (#1,0.43em);
\draw[stealth-,line width=0.4pt]%
(0pt,0pt) coordinate (X) 
(\the\dimen1,0.13em)  -- (\the\dimen2,0.13em);}%
\endgroup}
\def\leqArrow@Element#1{\begingroup%
\dimen1=#1%
\dimen0=\eqlbarrowstretch\dimen1%
\dimen1=#1 \advance\dimen1 by-\dimen0 \divide\dimen1 by2\relax
\dimen2=\dimen1 \advance\dimen2 by\dimen0
\tikz[baseline=(X.base)]{%
\draw[-stealth,line width=0.4pt]%
(\the\dimen1,0.43em)  -- (\the\dimen2,0.43em);
\draw[stealth-,line width=0.4pt]%
(0pt,0pt) coordinate (X) (0pt,0.13em)  -- (#1,0.13em);}%
\endgroup}
%    \end{macrocode}
%  \end{macro}
%  \end{macro}
%
%  \begin{macro}{\reqHarpoon@Element}
%  \begin{macro}{\leqHarpoon@Element}
% \changes{v1.02}{2009/11/19}{Added}
% \changes{v1.03}{2009/11/26}{Harpoons added as PostScript procedures}
% \changes{v5.00}{2010/10/01}{Redefined for PDF mode}
%    \begin{macrocode}
\def\reqHarpoon@Element#1{\begingroup%
\dimen1=#1%
\dimen0=\eqlbarrowstretch\dimen1%
\dimen1=#1 \advance\dimen1 by-\dimen0 \divide\dimen1 by2\relax
\dimen2=\dimen1 \advance\dimen2 by\dimen0
\tikz[baseline=(X.base)]{%
\draw[-left to,line width=0.4pt]%
(0pt,0.43em)  -- (#1,0.43em);
\draw[left to-,line width=0.4pt]%
(0pt,0pt) coordinate (X) 
(\the\dimen1,0.13em)  -- (\the\dimen2,0.13em);}%
\endgroup}
\def\leqHarpoon@Element#1{\begingroup%
\dimen1=#1%
\dimen0=\eqlbarrowstretch\dimen1%
\dimen1=#1 \advance\dimen1 by-\dimen0 \divide\dimen1 by2\relax
\dimen2=\dimen1 \advance\dimen2 by\dimen0
\tikz[baseline=(X.base)]{%
\draw[-left to,line width=0.4pt]%
(\the\dimen1,0.43em)  -- (\the\dimen2,0.43em);
\draw[left to-,line width=0.4pt]%
(0pt,0pt) coordinate (X) (0pt,0.13em)  -- (#1,0.13em);
}%
\endgroup}
%    \end{macrocode}
%  \end{macro}
%  \end{macro}
%
% Each arrow element is used in the following common command for drawing 
% horizontal reaction arrows. 
% \changes{v1.02}{2009/11/08}{changed definition; 
% added \cs{reactarrowsep} and \cs{reactarrowseprate}}
% \changes{v5.00}{2010/10/01}{Maintained for PDF mode}
%
% \begin{verbatim}
%  \reactarrowsep (dimension) for adjusting the two vertical spaces (default value 0pt)
%  \reactarrowseprate (string) for adjusting the lower vertical space (default 1.0) 
% \end{verbatim}
%
%  \begin{macro}{\react@@rlarrow}
%    \begin{macrocode}
\newdimen\@tempdimX
\def\react@@rlarrow[#1]#2#3#4#5{\ensuremath{\mathrel{%
\hskip#1\raisebox{#2}{\begingroup\@tempdimX=#3\relax%
  \parbox{#3}{\centering
    \baselineskip=.8\normalbaselineskip \ChemStrut{#4}\par%
    \vskip-0.2\normalbaselineskip\vskip\reactarrowsep%
    \noindent
    \Arrow@Element{\the\@tempdimX}
    \par%
    \vskip\reactarrowseprate\reactarrowsep%
    \vskip-0.1\normalbaselineskip\ChemStrut{#5}}\endgroup}}}}
%    \end{macrocode}
%  \end{macro}
%
%  \begin{macro}{\reactrarrow}
%  \begin{macro}{\reactlarrow}
%  \begin{macro}{\reactlrarrow}
%  \begin{macro}{\reactEqarrow}%2009/11/19 renamed from \verb/\reacteqarrow/
%  \begin{macro}{\reacteqarrow}%2009/11/19
% \changes{v5.00}{2010/10/01}{Maintained for PDF mode}
%    \begin{macrocode}
% ******************* 1991 Oct 4 S. Fujita
% * reaction arrows * Revised 1992 May 24 S.Fujita%1999/02/02 SF
% ******************* Redifined 2002 May 30 S. Fujita for pstricks
% After Chapter 9 of ``LaTeX for (Bio)Chemists'' by Shinsaku Fujita
%
% #1 yoko #2 ue, #3 haba, #4 ya no ue, #5 ya no shita 
% 
\def\newreactrarrow{%
\@ifnextchar[{\newreact@rarrow}{\newreact@rarrow[0pt]}}
\def\newreact@rarrow[#1]#2#3#4#5{%
\let\Arrow@Element=\rArrow@Element%
\react@@rlarrow[#1]{#2}{#3}{#4}{#5}}
\def\newreactlarrow{%
\@ifnextchar[{\newreact@larrow}{\newreact@larrow[0pt]}}
\def\newreact@larrow[#1]#2#3#4#5{%
\let\Arrow@Element=\lArrow@Element%
\react@@rlarrow[#1]{#2}{#3}{#4}{#5}}
\def\newreactlrarrow{%
\@ifnextchar[{\newreact@lrarrow}{\newreact@lrarrow[0pt]}}
\def\newreact@lrarrow[#1]#2#3#4#5{%
\let\Arrow@Element=\lrArrow@Element%
\react@@rlarrow[#1]{#2}{#3}{#4}{#5}}
%%%%2009/11/19
\def\newreactEqarrow{%
\@ifnextchar[{\newreact@Eqarrow}{\newreact@Eqarrow[0pt]}}
\def\newreact@Eqarrow[#1]#2#3#4#5{%
\let\Arrow@Element=\eqArrow@Element%
\react@@rlarrow[#1]{#2}{#3}{#4}{#5}}
\def\newreacteqarrow{%
\@ifnextchar[{\newreact@eqarrow}{\newreact@eqarrow[0pt]}}
\def\newreact@eqarrow[#1]#2#3#4#5{%
\let\Arrow@Element=\eqHarpoon@Element%
\react@@rlarrow[#1]{#2}{#3}{#4}{#5}}
%%%%%2009/11/19
%\let\reactrarrow=\newreactrarrow
%\let\reactlarrow=\newreactlarrow
%\let\reactlrarrow=\newreactlrarrow
%\let\reactEqarrow=\newreactEqarrow
%\let\reacteqarrow=\newreacteqarrow
%    \end{macrocode}
%  \end{macro}
%  \end{macro}
%  \end{macro}
%  \end{macro}
%  \end{macro}
%
%  \begin{macro}{\reactREqarrow}%%renamed 2009/11/19
%  \begin{macro}{\reactLEqarrow}%%renamed 2009/11/19
%  \begin{macro}{\reactreqarrow}
%  \begin{macro}{\reactleqarrow}
%
% These commands have been added for the XyMTeX version 4.05. 
% The {\sffamily chemist} package additionally defines arrows with variable length 
% for drawing chemical equilibrium, where additional information 
% can be written on the upper and/or downward sides of each arrow.  
% The length of either one of arrows can be reduced by \verb/\eqlbarrowstretch/.
% 
% \changes{v4.05}{2009/11/08}{added \cs{reactreqarrow} and \cs{reactleqarrow}}
% \changes{v5.00}{2010/10/01}{Maintained for PDF mode}
%
%    \begin{macrocode}
\def\newreactREqarrow{%
\@ifnextchar[{\newreact@REqarrow}{\newreact@REqarrow[0pt]}}
\def\newreact@REqarrow[#1]#2#3#4#5{%
\let\Arrow@Element=\reqArrow@Element
\react@@rlarrow[#1]{#2}{#3}{#4}{#5}}
\def\newreactLEqarrow{%
\@ifnextchar[{\newreact@LEqarrow}{\newreact@LEqarrow[0pt]}}
\def\newreact@LEqarrow[#1]#2#3#4#5{%
\let\Arrow@Element=\leqArrow@Element
\react@@rlarrow[#1]{#2}{#3}{#4}{#5}}
%%%%%2009/11/19
\def\newreactreqarrow{%
\@ifnextchar[{\newreact@reqarrow}{\newreact@reqarrow[0pt]}}
\def\newreact@reqarrow[#1]#2#3#4#5{%
\let\Arrow@Element=\reqHarpoon@Element
\react@@rlarrow[#1]{#2}{#3}{#4}{#5}}
\def\newreactleqarrow{%
\@ifnextchar[{\newreact@leqarrow}{\newreact@leqarrow[0pt]}}
\def\newreact@leqarrow[#1]#2#3#4#5{%
\let\Arrow@Element=\leqHarpoon@Element
\react@@rlarrow[#1]{#2}{#3}{#4}{#5}}
%%%%%2009/11/19
%\let\reactREqarrow=\newreactREqarrow
%\let\reactLEqarrow=\newreactLEqarrow
%\let\reactreqarrow=\newreactreqarrow
%\let\reactleqarrow=\newreactleqarrow
%    \end{macrocode}
%  \end{macro}
%  \end{macro}
%  \end{macro}
%  \end{macro}
%
% Several elements for drawing vertical reaction arrows within the scope of 
% pstricks package are defined first.  
% \changes{v5.00}{2010/10/01}{Redefined for PDF mode}
%
%  \begin{macro}{\dArrow@Element}
%  \begin{macro}{\uArrow@Element}
%  \begin{macro}{\veqArrow@Element}
%  \begin{macro}{\duArrow@Element}
%    \begin{macrocode}
\def\dArrow@Element#1{%
\tikz[baseline=(X.base)]{%
\draw[-stealth,line width=0.4pt]%
(0pt,#1) coordinate (X)  -- (0pt,0pt);}}
\def\uArrow@Element#1{%
\tikz[baseline=(X.base)]{%
\draw[stealth-,line width=0.4pt]%
(0pt,#1) coordinate (X) -- (0pt,0pt);}}
\def\veqArrow@Element#1{%
\tikz[baseline=(X.base)]{%
\draw[-stealth,line width=0.4pt]%
(0pt,0pt) coordinate (X)
(-0.15em,#1)  -- (-0.15em,0pt);
\draw[stealth-,line width=0.4pt]%
(0.15em,#1) -- (0.15em,0pt);}}
\def\duArrow@Element#1{%
\tikz[baseline=(X.base)]{%
\draw[stealth-stealth,line width=0.4pt]%
(0pt,#1) coordinate (X) -- (0pt,0pt);}}
%    \end{macrocode}
%  \end{macro}
%  \end{macro}
%  \end{macro}
%  \end{macro}
%
%  \begin{macro}{\veqHarpoon@Element}
% \changes{v1.02}{2009/11/19}{Added}
% \changes{v1.03}{2009/11/26}{Harpoons added as PostScript procedures}
% \changes{v5.00}{2010/10/01}{Redefined for PDF mode}
%    \begin{macrocode}
\def\veqHarpoon@Element#1{%
\tikz[baseline=(X.base)]{%
\draw[-left to,line width=0.4pt]%
(0pt,0pt) coordinate (X)
(-0.15em,#1)  -- (-0.15em,0pt);
\draw[left to-,line width=0.4pt]%
(0.15em,#1) -- (0.15em,0pt);}}
%    \end{macrocode}
%  \end{macro}
%
% Each arrow element is used in the following common command for drawing 
% vertical reaction arrows. 
% \changes{v5.00}{2010/10/01}{Redefined for PDF mode}
%
%  \begin{macro}{\react@@duarrow}
%    \begin{macrocode}
\def\react@@duarrow[#1]#2#3#4#5{\hskip#1\raisebox{#2}{%
\@tempdimY=#3\def\arraystretch{0.8}%
\begin{tabular}{l}#4\end{tabular}%
{\tabcolsep=0pt\begin{tabular}{c}%
\Arrow@Element{\the\@tempdimY}%
\end{tabular}}%
\begin{tabular}{l}#5\end{tabular}}}
%    \end{macrocode}
%  \end{macro}
%
%  \begin{macro}{\reactdarrow}
%  \begin{macro}{\reactuarrow}
%  \begin{macro}{\reactduarrow}
%  \begin{macro}{\reactVEqarrow}%%renamed 2009/11/19
%  \begin{macro}{\reactveqarrow}%%renamed 2009/11/19
% Vertical reaction arrows has been added. 
% \changes{v1.00}{2002/05/30}{First Version}
% \changes{v5.00}{2010/10/01}{Maintained for PDF mode}
%    \begin{macrocode}
\def\newreactdarrow{%
\@ifnextchar[{\newreact@darrow}{\newreact@darrow[0pt]}}
\def\newreact@darrow[#1]#2#3#4#5{%
\let\Arrow@Element=\dArrow@Element
\react@@duarrow[#1]{#2}{#3}{#4}{#5}}
\def\newreactuarrow{%
\@ifnextchar[{\newreact@uarrow}{\newreact@uarrow[0pt]}}
\def\newreact@uarrow[#1]#2#3#4#5{%
\let\Arrow@Element=\uArrow@Element
\react@@duarrow[#1]{#2}{#3}{#4}{#5}}
\def\newreactduarrow{%
\@ifnextchar[{\newreact@duarrow}{\newreact@duarrow[0pt]}}
\def\newreact@duarrow[#1]#2#3#4#5{%
\let\Arrow@Element=\duArrow@Element
\react@@duarrow[#1]{#2}{#3}{#4}{#5}}
%%%%%2009/11/19
\def\newreactVEqarrow{%
\@ifnextchar[{\newreact@VEqarrow}{\newreact@VEqarrow[0pt]}}
\def\newreact@VEqarrow[#1]#2#3#4#5{%
\let\Arrow@Element=\veqArrow@Element
\react@@duarrow[#1]{#2}{#3}{#4}{#5}}
\def\newreactveqarrow{%
\@ifnextchar[{\newreact@veqarrow}{\newreact@veqarrow[0pt]}}
\def\newreact@veqarrow[#1]#2#3#4#5{%
\let\Arrow@Element=\veqHarpoon@Element
\react@@duarrow[#1]{#2}{#3}{#4}{#5}}
%%%%%2009/11/19
%\let\reactdarrow=\newreactdarrow
%\let\reactuarrow=\newreactuarrow
%\let\reactduarrow=\newreactduarrow
%\let\reactVEqarrow=\newreactVEqarrow
%\let\reactveqarrow=\newreactveqarrow
%    \end{macrocode}
%  \end{macro}
%  \end{macro}
%  \end{macro}
%  \end{macro}
%  \end{macro}
%
% Several elements for drawing down-sloped reaction arrows within the scope of 
% pgf package are defined first.  
% \changes{v5.00}{2010/10/01}{Redefined for PDF mode}
%
%  \begin{macro}{\seArrow@Element}
%    \begin{macrocode}
\def\seArrow@Element#1{%
\tikz[baseline=(X.base)]{%
\draw[-stealth,line width=0.4pt]%
(0pt,#1) coordinate (X) -- (#1,0pt);}}
\def\nwArrow@Element#1{%
\tikz[baseline=(X.base)]{%
\draw[stealth-,line width=0.4pt]%
(0pt,#1) coordinate (X) -- (#1,0pt);}}
%    \end{macrocode}
%  \end{macro}
%
% Each arrow element is used in the following common command for drawing 
% down-sloped reaction arrows. 
% \changes{v5.00}{2010/10/01}{Redefined for PDF mode}
%
%  \begin{macro}{\react@@decrarrow}
%    \begin{macrocode}
\def\react@@decrarrow[#1]#2#3#4#5{\hskip#1\raisebox{#2}{%
\@tempdimX=#3\divide\@tempdimX by2\relax
\@tempdimY=#3\relax\def\arraystretch{0.8}%
\begin{tabular}{l}#4\end{tabular}\kern1em%
{\tabcolsep=0pt\begin{tabular}{c}
\Arrow@Element{\the\@tempdimY}%
\end{tabular}}\kern1em%
\begin{tabular}{l}#5\end{tabular}}}
%    \end{macrocode}
%  \end{macro}
%
%  \begin{macro}{\reactsearrow}
%  \begin{macro}{\reactnwrarrow}
% Further reaction arrows have been added. 
% \changes{v1.00}{2002/05/30}{First Version for PostScript}
% \changes{v5.00}{2010/10/01}{Maintained for PDF mode}
%    \begin{macrocode}
\def\newreactsearrow{%
\@ifnextchar[{\newreact@searrow}{\newreact@searrow[0pt]}}
\def\newreact@searrow[#1]#2#3#4#5{%
\let\Arrow@Element=\seArrow@Element
\react@@decrarrow[#1]{#2}{#3}{#4}{#5}}
\def\newreactnwarrow{%
\@ifnextchar[{\newreact@nwarrow}{\newreact@nwarrow[0pt]}}
\def\newreact@nwarrow[#1]#2#3#4#5{%
\let\Arrow@Element=\nwArrow@Element
\react@@decrarrow[#1]{#2}{#3}{#4}{#5}}
%%%%%2009/11/19
%\let\reactsearrow=\newreactsearrow
%\let\reactnwarrow=\newreactnwarrow
%    \end{macrocode}
%  \end{macro}
%  \end{macro}
%
% Several elements for drawing up-sloped reaction arrows within the scope of 
% pgf package are defined first.  
% \changes{v5.00}{2010/10/01}{Redefined for PDF mode}
%
%  \begin{macro}{\neArrow@Element}
%    \begin{macrocode}
\def\neArrow@Element#1{%
\tikz[baseline=(X.base)]{%
\draw[-stealth,line width=0.4pt]%
(0pt,0pt) coordinate (X) -- (#1,#1);}}
\def\swArrow@Element#1{%
\tikz[baseline=(X.base)]{%
\draw[stealth-,line width=0.4pt]%
(0pt,0pt) coordinate (X) -- (#1,#1);}}
%    \end{macrocode}
%  \end{macro}
%
% Each arrow element is used in the following common command for drawing 
% up-sloped reaction arrows. 
% \changes{v5.00}{2010/10/01}{Redefined for PDF mode}
%
%  \begin{macro}{\react@@incrarrow}
%    \begin{macrocode}
\def\react@@incrarrow[#1]#2#3#4#5{\hskip#1\raisebox{#2}{%
\@tempdimX=#3 \divide\@tempdimX by2
\@tempdimY=#3%
\def\arraystretch{0.8}%
\begin{tabular}{l}#4\end{tabular}\kern1em%
{\tabcolsep=0pt
\begin{tabular}{c}
\Arrow@Element{\the\@tempdimY}%
\end{tabular}}\kern1em%
\begin{tabular}{l}#5\end{tabular}}}
%    \end{macrocode}
%  \end{macro}
%
%  \begin{macro}{\reactnearrow}
%  \begin{macro}{\reactswarrow}
% \changes{v5.00}{2010/10/01}{Maintained for PDF mode}
%    \begin{macrocode}
\def\newreactnearrow{%
\@ifnextchar[{\newreact@nearrow}{\newreact@nearrow[0pt]}}
\def\newreact@nearrow[#1]#2#3#4#5{%
\let\Arrow@Element=\neArrow@Element
\react@@incrarrow[#1]{#2}{#3}{#4}{#5}}
\def\newreactswarrow{%
\@ifnextchar[{\newreact@swarrow}{\newreact@swarrow[0pt]}}
\def\newreact@swarrow[#1]#2#3#4#5{%
\let\Arrow@Element=\swArrow@Element
\react@@incrarrow[#1]{#2}{#3}{#4}{#5}}
%%%%%2009/11/19
%\let\reactnearrow=\newreactnearrow
%\let\reactswarrow=\newreactswarrow
%    \end{macrocode}
%  \end{macro}
%  \end{macro}
%
% Additional elements for drawing down- and up-sloped equilibrium arrows within the scope of 
% pgf package are defined first.  
% \changes{v5.00}{2010/10/01}{Redefined for PDF mode}
%
%  \begin{macro}{\deqArrow@Element}
%  \begin{macro}{\ueqArrow@Element}
%    \begin{macrocode}
\def\deqArrow@Element#1{%
\begingroup
\@tempdima=#1 \advance\@tempdima by1.2pt
\@tempdimb=#1 \advance\@tempdimb by-1.2pt
\tikz[baseline=(X.base)]{%
\draw[-stealth,line width=0.4pt]%
(1.2pt,\the\@tempdima)  coordinate (X) -- (\the\@tempdima,1.2pt);%
\draw[stealth-,line width=0.4pt]%
(-1.2pt,\the\@tempdimb) -- (\the\@tempdimb,-1.2pt);}
\endgroup}
\def\ueqArrow@Element#1{%
\begingroup
\@tempdima=#1 \advance\@tempdima by-1.2pt
\@tempdimb=#1 \advance\@tempdimb by1.2pt
\tikz[baseline=(X.base)]{%
\draw[-stealth,line width=0.4pt]%
(-1.2pt,1.2pt)  coordinate (X) -- (\the\@tempdima,\the\@tempdimb);%
\draw[stealth-,line width=0.4pt]%
(1.2pt,-1.2pt) -- (\the\@tempdimb,\the\@tempdima);}
\endgroup}
%    \end{macrocode}
%  \end{macro}
%  \end{macro}
%
% Additional elements for drawing down- and up-sloped equilibrium harpoons within the scope of 
% pgf package are defined.  
% \changes{v1.02}{2009/11/19}{Added}
%
%  \begin{macro}{\deqHarpoon@Element}
%  \begin{macro}{\ueqHarpoon@Element}
% \changes{v1.03}{2009/11/26}{Harpoons added as PostScript procedures}
% \changes{v5.00}{2010/10/01}{Redefined for PDF mode}
%    \begin{macrocode}
\def\deqHarpoon@Element#1{%
\begingroup
\@tempdima=#1 \advance\@tempdima by1.2pt
\@tempdimb=#1 \advance\@tempdimb by-1.2pt
\tikz[baseline=(X.base)]{%
\draw[-left to,line width=0.4pt]%
(1.2pt,\the\@tempdima)  coordinate (X) -- (\the\@tempdima,1.2pt);%
\draw[left to-,line width=0.4pt]%
(-1.2pt,\the\@tempdimb) -- (\the\@tempdimb,-1.2pt);}
\endgroup}
\def\ueqHarpoon@Element#1{%
\begingroup
\@tempdima=#1 \advance\@tempdima by-1.2pt
\@tempdimb=#1 \advance\@tempdimb by1.2pt
\tikz[baseline=(X.base)]{%
\draw[-left to,line width=0.4pt]%
(-1.2pt,1.2pt)  coordinate (X) -- (\the\@tempdima,\the\@tempdimb);%
\draw[left to-,line width=0.4pt]%
(1.2pt,-1.2pt) -- (\the\@tempdimb,\the\@tempdima);}
\endgroup}
%    \end{macrocode}
%  \end{macro}
%  \end{macro}
%
%  \begin{macro}{\reactDEqarrow}
%  \begin{macro}{\reactUEqarrow}
%  \begin{macro}{\reactdeqarrow}
%  \begin{macro}{\reactueqarrow}
% Further reaction arrows have been added. 
% \changes{v1.00}{2002/05/30}{First Version}
% \changes{v5.00}{2010/10/01}{Maintained for PDF mode}
%    \begin{macrocode}
\def\newreactDEqarrow{\@ifnextchar[%]
{\newreact@DEqarrow}{\newreact@DEqarrow[0pt]}}
\def\newreact@DEqarrow[#1]#2#3#4#5{%
\let\Arrow@Element=\deqArrow@Element
\react@@decrarrow[#1]{#2}{#3}{#4}{#5}}
\def\newreactUEqarrow{\@ifnextchar[%]
{\newreact@UEqarrow}{\newreact@UEqarrow[0pt]}}
\def\newreact@UEqarrow[#1]#2#3#4#5{%
\let\Arrow@Element=\ueqArrow@Element
\react@@decrarrow[#1]{#2}{#3}{#4}{#5}}
%%%%2009/11/19
\def\newreactdeqarrow{\@ifnextchar[%]
{\newreact@deqarrow}{\newreact@deqarrow[0pt]}}
\def\newreact@deqarrow[#1]#2#3#4#5{%
\let\Arrow@Element=\deqHarpoon@Element
\react@@decrarrow[#1]{#2}{#3}{#4}{#5}}
\def\newreactueqarrow{\@ifnextchar[%]
{\newreact@ueqarrow}{\newreact@ueqarrow[0pt]}}
\def\newreact@ueqarrow[#1]#2#3#4#5{%
\let\Arrow@Element=\ueqHarpoon@Element
\react@@decrarrow[#1]{#2}{#3}{#4}{#5}}
%%%%2009/11/19
%\let\reactDEqarrow=\newreactDEqarrow
%\let\reactUEqarrow=\newreactUEqarrow
%\let\reactdeqarrow=\newreactdeqarrow
%\let\reactueqarrow=\newreactueqarrow
%    \end{macrocode}
%  \end{macro}
%  \end{macro}
%  \end{macro}
%  \end{macro}
%
% Additional elements for drawing down- and up-sloped double-headed arrows 
% within the scope of pgf package are defined first.  
% \changes{v5.00}{2010/10/01}{Redefined for PDF mode}
%
%  \begin{macro}{\ulrArrow@Element}
%  \begin{macro}{\dlrArrow@Element}
%    \begin{macrocode}
\def\ulrArrow@Element#1{%
\tikz[baseline=(X.base)]{%
\draw[stealth-stealth,line width=0.4pt]%
(0pt,0pt) coordinate (X) -- (#1,#1);}}
\def\dlrArrow@Element#1{%
\tikz[baseline=(X.base)]{%
\draw[stealth-stealth,line width=0.4pt]%
(0pt,#1) coordinate (X) -- (#1,0pt);}}
%    \end{macrocode}
%  \end{macro}
%  \end{macro}
%
%  \begin{macro}{\reactulrarrow}
%  \begin{macro}{\reactdlrarrow}
% \changes{v5.00}{2010/10/01}{Maitained for PDF mode}
%    \begin{macrocode}
\def\newreactulrarrow{\@ifnextchar[%]
{\newreact@ulrarrow}{\newreact@ulrarrow[0pt]}}
\def\newreact@ulrarrow[#1]#2#3#4#5{%
\let\Arrow@Element=\ulrArrow@Element
\react@@incrarrow[#1]{#2}{#3}{#4}{#5}}
\def\newreactdlrarrow{\@ifnextchar[%]
{\newreact@dlrarrow}{\newreact@dlrarrow[0pt]}}
\def\newreact@dlrarrow[#1]#2#3#4#5{%
\let\Arrow@Element=\dlrArrow@Element
\react@@decrarrow[#1]{#2}{#3}{#4}{#5}}
%%%%2009/11/19
%\let\reactulrarrow=\newreactulrarrow
%\let\reactdlrarrow=\newreactdlrarrow
%    \end{macrocode}
%  \end{macro}
%  \end{macro}
%
%
% A new command \verb/\electronshiftAH/ is added to defining bent (curved) arrows 
% and harpoons.  
% \changes{v1.03}{2009/12/01}{Added}
% \changes{v5.00}{2010/10/01}{Redefined for PDF mode}
%
%  \begin{macro}{\electronshiftAH}
%    \begin{macrocode}
\def\electronshiftAH#1(#2,#3)(#4,#5)(#6,#7){%
\@ifnextchar(%
{\electr@nshiftAH{#1}(#2,#3)(#4,#5)(#6,#7)}%
{\electr@nshiftAH{#1}(#2,#3)(#4,#5)(#4,#5)(#6,#7)}}
\def\electr@nshiftAH#1(#2,#3)(#4,#5)(#6,#7)(#8,#9){%
%\tikz[baseline=(X.base)]{%
%\draw [#1,line width=0.4pt] (0pt,0pt) coordinate (X) 
%(#2\unitlength,#3\unitlength) .. 
%controls (#4\unitlength,#5\unitlength) 
%and (#6\unitlength,#7\unitlength) .. 
%(#8\unitlength,#9\unitlength);}
\tikznodimension{%
\draw [#1,line width=0.4pt]%%(0pt,0pt) coordinate (X) 
(#2\unitlength,#3\unitlength) .. 
controls (#4\unitlength,#5\unitlength) 
and (#6\unitlength,#7\unitlength) .. 
(#8\unitlength,#9\unitlength);}%
}%
%    \end{macrocode}
%  \end{macro}
%
% Several commmands for drawing curved harpoons are defined as follows: 
% \changes{v5.00}{2010/10/01}{Redefined for PDF mode}
%  \begin{macro}{\electronshiftHru}
%  \begin{macro}{\electronshiftHrd}
%  \begin{macro}{\electronshiftHlu}
%  \begin{macro}{\electronshiftHld}
%    \begin{macrocode}
\def\electronshiftHru{\electronshiftAH{-left to}}
\def\electronshiftHrd{\electronshiftAH{-right to}}
\def\electronshiftHlu{\electronshiftAH{right to-}}
\def\electronshiftHld{\electronshiftAH{left to-}}
%    \end{macrocode}
%  \end{macro}
%  \end{macro}
%  \end{macro}
%  \end{macro}
%
% Several commmands for drawing curved arrows are defined as follows: 
% \changes{v5.00}{2010/10/01}{Redefined for PDF mode}
%  \begin{macro}{\electronshiftArrowr}
%  \begin{macro}{\electronshiftArrowl}
%    \begin{macrocode}
\def\electronshiftArrowr{\electronshiftAH{-stealth}}
\def\electronshiftArrowl{\electronshiftAH{stealth-}}
%    \end{macrocode}
%  \end{macro}
%  \end{macro}
%
% The {\sffamily chmst-pdf} package defines double-line arrows with variable 
% length for drawing reaction schemes, where additional information 
% can be written on the upper and/or downward sides of each arrow.  
% \changes{v1.00}{2002/05/30}{First Version}
% \changes{v5.00}{2010/10/01}{Redefined for PDF mode}
%
% Several elements for drawing horizontal scheme arrows within the scope of 
% pgf package are defined first.  
%
%  \begin{macro}{\lSchemeArrow@Element}
%    \begin{macrocode}
\def\lSchemeArrow@Element#1{\hbox to#1{\hss%
\tikz[baseline=(X.base)]{%
\draw[stealth-,double distance=0.16em,line width=0.4pt]%
(0pt,0.28em)  coordinate (X) -- (#1,0.28em);}%
\hss}}
\def\rSchemeArrow@Element#1{\hbox to#1{\hss%
\tikz[baseline=(X.base)]{%
\draw[-stealth,double distance=0.16em,line width=0.4pt]%
(0pt,0.28em)  coordinate (X) -- (#1,0.28em);}%
\hss}}
\def\lrSchemeArrow@Element#1{\hbox to#1{\hss%
\tikz[baseline=(X.base)]{%
\draw[stealth-stealth,double distance=0.16em,line width=0.4pt]%
(0pt,0.28em)  coordinate (X) -- (#1,0.28em);}%
\hss}}
%    \end{macrocode}
%  \end{macro}
%
%  \begin{macro}{\schemelarrow}
%  \begin{macro}{\schemerarrow}
%  \begin{macro}{\schemelrarrow}
% \changes{v5.00}{2010/10/01}{Maintained for PDF mode}
%
%    \begin{macrocode}
% ********************************
% * scheme arrows                *
% * 1991 OCT 4 S. Fujita         *
% * Revised 1992 May 24 S.Fujita * 1999/02/02 by S. Fujita
% ******************************** Redefined 2002/05/30 by SF
% After Chapter 9 of ``LaTeX for (Bio)Chemists'' by Shinsaku Fujita
\def\newschemelarrow#1#2#3#4{%
\let\Arrow@Element=\lSchemeArrow@Element
\react@@rlarrow[0pt]{#1}{#2}{#3}{#4}}
\def\newschemerarrow#1#2#3#4{%
\let\Arrow@Element=\rSchemeArrow@Element
\react@@rlarrow[0pt]{#1}{#2}{#3}{#4}}
\def\newschemelrarrow#1#2#3#4{%
\let\Arrow@Element=\lrSchemeArrow@Element
\react@@rlarrow[0pt]{#1}{#2}{#3}{#4}}
%%%%2009/11/19
%\let\schemelarrow=\newschemelarrow
%\let\schemerarrow=\newschemerarrow
%\let\schemelrarrow=\newschemelrarrow
%    \end{macrocode}
%  \end{macro}
%  \end{macro}
%  \end{macro}
%
% \section{Redefinition of verbatim}
% (Deleted)
%
% \section{Shadowed Boxes}
% (Deleted)
%
% \section{Further boxes}
% (Deleted)
%
% \section{Symbols for instant photography} 
% (Deleted)
%
% \section{Various arrows with fixed length}
%
% Because various arrows with a fixed length have been defined in the \textsf{chemist} package, 
% the corresponding codes for the \textsf{chmst-pdf} package are described here. 
%
% \changes{v1.02}{2009/11/19}{Added}
% \changes{v5.00}{2010/10/01}{Maintained for PDF mode}
% 
%  \begin{macro}{\llongrightarrow}
%  \begin{macro}{\llongleftarrow}
%  \begin{macro}{\Equilibarrow}
%
%    \begin{macrocode}
\def\newllongrightarrow{\ensuremath{\mathrel{%
\hbox to28pt{\hss\rArrow@Element{28pt}\hss}}}}
\def\newllongleftarrow{\ensuremath{\mathrel{%
\hbox to28pt{\hss\lArrow@Element{28pt}\hss}}}}
\def\newllongleftrightarrow{\ensuremath{\mathrel{%
\hbox to28pt{\hss\lrArrow@Element{28pt}\hss}}}}
\def\newEquilibarrow{\ensuremath{\mathrel{%
\hbox to28pt{\hss\eqArrow@Element{28pt}\hss}}}}
%\let\llongrightarrow=\newllongrightarrow
%\let\llongleftarrow=\newllongleftarrow
%\let\llongleftrightarrow=\newllongleftrightarrow
%\let\Equilibarrow=\newEquilibarrow
%    \end{macrocode}
%  \end{macro}
%  \end{macro}
%  \end{macro}
%
% \changes{v1.02}{2009/11/19}{Added}
% \changes{v5.00}{2010/10/01}{Maintained for PDF mode}
%  \begin{macro}{\Llongrightarrow}
%  \begin{macro}{\Llongleftarrow}
%  \begin{macro}{\Llongleftrightarrow}
%
%    \begin{macrocode}
\def\newLlongrightarrow{\ensuremath{\mathrel{%
\hbox to28pt{\hss\rSchemeArrow@Element{28pt}\hss}}}}
\def\newLlongleftarrow{\ensuremath{\mathrel{%
\hbox to28pt{\hss\lSchemeArrow@Element{28pt}\hss}}}}
\def\newLlongleftrightarrow{\ensuremath{%
\mathrel{\hbox to28pt{\hss\lrSchemeArrow@Element{28pt}\hss}}}}
%\let\Llongrightarrow=\newLlongrightarrow
%\let\Llongleftarrow=\newLlongleftarrow
%\let\Llongleftrightarrow\newLlongleftrightarrow
%    \end{macrocode}
%  \end{macro}
%  \end{macro}
%  \end{macro}
%
% \changes{v1.02}{2009/11/19}{Added}
% \changes{v5.00}{2010/10/01}{Maintained for PDF mode}
%  \begin{macro}{\lllongrightarrow}
%  \begin{macro}{\lllongleftarrow}
%  \begin{macro}{\lllongleftrightarrow}
%  \begin{macro}{\Equiliblongarrow}
%
%    \begin{macrocode}
\def\newlllongrightarrow{\ensuremath{\mathrel{%
\hbox to35pt{\hss\rArrow@Element{35pt}\hss}}}}
\def\newlllongleftarrow{\ensuremath{\mathrel{%
\hbox to35pt{\hss\lArrow@Element{35pt}\hss}}}}
\def\newlllongleftrightarrow{\ensuremath{\mathrel{%
\hbox to35pt{\hss\lrArrow@Element{35pt}\hss}}}}
\def\newEquiliblongarrow{\ensuremath{\mathrel{%
\hbox to35pt{\hss\eqArrow@Element{35pt}\hss}}}}
%\let\lllongrightarrow=\newlllongrightarrow
%\let\lllongleftarrow=\newlllongleftarrow
%\let\lllongleftrightarrow=\newlllongleftrightarrow
%\let\Equiliblongarrow=\newEquiliblongarrow
%    \end{macrocode}
%  \end{macro}
%  \end{macro}
%  \end{macro}
%  \end{macro}
%
% \changes{v1.02}{2009/11/19}{Added}
% \changes{v5.00}{2010/10/01}{Maintained for PDF mode}
%  \begin{macro}{\Lllongrightarrow}
%  \begin{macro}{\Lllongleftarrow}
%  \begin{macro}{\Lllongleftrightarrow}
%
%    \begin{macrocode}
\def\newLllongrightarrow{\ensuremath{\mathrel{%
\hbox to35pt{\hss\rSchemeArrow@Element{35pt}\hss}}}}
\def\newLllongleftarrow{\ensuremath{\mathrel{%
\hbox to35pt{\hss\lSchemeArrow@Element{35pt}\hss}}}}
\def\newLllongleftrightarrow{\ensuremath{%
\mathrel{\hbox to35pt{\hss\lrSchemeArrow@Element{35pt}\hss}}}}
%\let\Lllongrightarrow=\newLllongrightarrow
%\let\Lllongleftarrow=\newLllongleftarrow
%\let\Lllongleftrightarrow\newLllongleftrightarrow
%    \end{macrocode}
%  \end{macro}
%  \end{macro}
%  \end{macro}
%
%
% Several elements for drawing horizontal harpoons within the scope of 
% pgf package are defined first.  
% \changes{v1.02}{2009/11/19}{Added}
%
%  \begin{macro}{\rightharpoonupElement}
% \changes{v1.02}{2009/11/19}{Added}
% \changes{v1.03}{2009/11/26}{Harpoons added as PostScript procedures}
% \changes{v5.00}{2010/10/01}{Redefined for PDF mode}
%    \begin{macrocode}
\def\rightharpoonupElement#1{%
\leavevmode
\lower0.2pt\hbox to#1{\hss%
\tikz[baseline=(X.base)]{%
\draw[-left to,line width=0.4pt]%
(0pt,0pt) coordinate (X)
(0pt,0.28em) -- (#1,0.28em);}%
\hss}}
%    \end{macrocode}
%  \end{macro}
%
%  \begin{macro}{\rightharpoondownElement}
% \changes{v1.02}{2009/11/19}{Added}
% \changes{v1.03}{2009/11/26}{Harpoons added as PostScript procedures}
% \changes{v5.00}{2010/10/01}{Redefined for PDF mode}
%    \begin{macrocode}
\def\rightharpoondownElement#1{%
\leavevmode
\raise0.2pt\hbox to#1{\hss%
\tikz[baseline=(X.base)]{%
\draw[-right to,line width=0.4pt]%
(0pt,0pt) coordinate (X)
(0pt,0.28em) -- (#1,0.28em);}%
\hss}}
%    \end{macrocode}
%  \end{macro}
%
%  \begin{macro}{\leftharpoonupElement}
% \changes{v1.02}{2009/11/19}{Added}
% \changes{v1.03}{2009/11/26}{Harpoons added as PostScript procedures}
% \changes{v5.00}{2010/10/01}{Redefined for PDF mode}
%    \begin{macrocode}
\def\leftharpoonupElement#1{%
\leavevmode
\lower0.2pt\hbox to#1{\hss%
\tikz[baseline=(X.base)]{%
\draw[right to-,line width=0.4pt]%
(0pt,0pt) coordinate (X)
(0pt,0.28em) -- (#1,0.28em);}%
\hss}}
%    \end{macrocode}
%  \end{macro}
%
%  \begin{macro}{\leftharpoondownElement}
% \changes{v1.02}{2009/11/19}{Added}
% \changes{v1.03}{2009/11/26}{Harpoons added as PostScript procedures}
% \changes{v5.00}{2010/10/01}{Redefined for PDF mode}
%    \begin{macrocode}
\def\leftharpoondownElement#1{%
\leavevmode
\lower0.2pt\hbox to#1{\hss%
\tikz[baseline=(X.base)]{%
\draw[left to-,line width=0.4pt]%
(0pt,0pt) coordinate (X)
(0pt,0.28em) -- (#1,0.28em);}%
\hss}}
%    \end{macrocode}
%  \end{macro}
%
% \changes{v1.02}{2009/11/19}{Added}
% \changes{v5.00}{2010/10/01}{Maintained for PDF mode}
%  \begin{macro}{\llongrightharpoonup}
%  \begin{macro}{\llongrightharpoondown}
%  \begin{macro}{\llongleftharpoonup}
%  \begin{macro}{\llongleftharpoondown}
%    \begin{macrocode}
\def\newllongrightharpoonup{\ensuremath{\mathrel{\rightharpoonupElement{28pt}}}}
\def\newllongrightharpoondown{\ensuremath{\mathrel{\rightharpoondownElement{28pt}}}}
\def\newllongleftharpoonup{\ensuremath{\mathrel{\leftharpoonupElement{28pt}}}}
\def\newllongleftharpoondown{\ensuremath{\mathrel{\leftharpoondownElement{28pt}}}}
%\let\llongrightharpoonup=\newllongrightharpoonup
%\let\llongrightharpoondown=\newllongrightharpoondown
%\let\llongleftharpoonup=\newllongleftharpoonup
%\let\llongleftharpoondown=\newllongleftharpoondown
%    \end{macrocode}
%  \end{macro}
%  \end{macro}
%  \end{macro}
%  \end{macro}
%
% \changes{v1.02}{2009/11/19}{Added}
% \changes{v5.00}{2010/10/01}{Maintained for PDF mode}
%  \begin{macro}{\lllongrightharpoonup}
%  \begin{macro}{\lllongrightharpoondown}
%  \begin{macro}{\lllongleftharpoonup}
%  \begin{macro}{\lllongleftharpoondown}
%    \begin{macrocode}
\def\newlllongrightharpoonup{\ensuremath{\mathrel{\rightharpoonupElement{35pt}}}}
\def\newlllongrightharpoondown{\ensuremath{\mathrel{\rightharpoondownElement{35pt}}}}
\def\newlllongleftharpoonup{\ensuremath{\mathrel{\leftharpoonupElement{35pt}}}}
\def\newlllongleftharpoondown{\ensuremath{\mathrel{\leftharpoondownElement{35pt}}}}
%\let\lllongrightharpoonup=\newlllongrightharpoonup
%\let\lllongrightharpoondown=\newlllongrightharpoondown
%\let\lllongleftharpoonup=\newlllongleftharpoonup
%\let\lllongleftharpoondown\newlllongleftharpoondown
%    \end{macrocode}
%  \end{macro}
%  \end{macro}
%  \end{macro}
%  \end{macro}
%
%
% \changes{v1.02}{2009/11/19}{Added}
% \changes{v5.00}{2010/10/01}{Maintained for PDF mode}
%  \begin{macro}{\equilibarrow}
%  \begin{macro}{\equiliblongarrow}
%    \begin{macrocode}
\def\newequilibarrow{\ensuremath{\mathrel{%
\hbox to28pt{\hss\eqHarpoon@Element{28pt}\hss}}}}
\def\newequiliblongarrow{\ensuremath{\mathrel{%
\hbox to35pt{\hss\eqHarpoon@Element{35pt}\hss}}}}
%\let\equilibarrow=\newequilibarrow
%\let\equiliblongarrow=\newequiliblongarrow
%    \end{macrocode}
%  \end{macro}
%  \end{macro}
%
% 
% \section{Chemical Corrections}
% (Deleted)
% 
%
% \section{XyMcompd and XyMtab Environments}
% (Deleted)
%
% \section{Useful commands}
% (Deleted)
%
% \section{Double and triple bonds}
% (Deleted)
%
% \section{Curved Arrows and Harpoons}
%
% To illustrate the mechanisms of organic reactions, 
% curved arrows are used to show an electron shift. 
% First, the macros for drawing them are 
% defined as follows. 
% \changes{v1.01}{2004/08/30}{Added at XyMTeX V4.01}
% \changes{v5.00}{2010/10/01}{Redefined for PDF mode}
%
%  \begin{macro}{\electronAHshift}
%
%    \begin{macrocode}
\def\electronAHshift{%
\@ifnextchar[{\@electronAHshift}{\@electronAHshift[0]}}
\def\@electronAHshift[#1](#2,#3)(#4,#5)#6{%
\@tempcntXa=#2 \@tempcntYa=#3
\@tempcntXb=#4 \@tempcntYb=#5
\calcontrolpoints{#1}%defined in chemist package
\tikznodimension{%
\draw [#6,line width=0.4pt]%%%% (0pt,0pt) coordinate (X) 
(#2\unitlength,#3\unitlength) .. 
controls (\@cnttempa\unitlength,\@cnttempb\unitlength) and 
(\@tempcntXc\unitlength,\@tempcntYc\unitlength) .. 
(#4\unitlength,#5\unitlength);}%
}
%    \end{macrocode}
%  \end{macro}
%
% Arrows having a usual arrow head are defined as follows. 
% Keywords for differentiation are:  r (right) and l (left), 
% which are contained in the middle of each command name. 
%
%  \begin{macro}{\electronrshiftarrow}
% \changes{v5.00}{2010/10/01}{Redefined for PDF mode}
%    \begin{macrocode}
\def\electronrshiftarrow{%
\@ifnextchar[{\@electronrshiftarrow}{\@electronrshiftarrow[0]}}
\def\@electronrshiftarrow[#1](#2,#3)(#4,#5){%
\ifnum#2<#4\relax
\electronAHshift[#1](#2,#3)(#4,#5){-stealth}%
\else 
\ifcase#1\relax
\electronAHshift[1](#2,#3)(#4,#5){-stealth}%
\or
\electronAHshift[0](#2,#3)(#4,#5){-stealth}%
\fi\fi}
%    \end{macrocode}
%  \end{macro}
%
%  \begin{macro}{\electronlshiftarrow}
% \changes{v5.00}{2010/10/01}{Redefined for PDF mode}
%    \begin{macrocode}
\def\electronlshiftarrow{%
\@ifnextchar[{\@electronlshiftarrow}{\@electronlshiftarrow[0]}}
\def\@electronlshiftarrow[#1](#2,#3)(#4,#5){%
\ifnum#2<#4\relax
\electronAHshift[#1](#2,#3)(#4,#5){stealth-}%
\else 
\ifcase#1\relax
\electronAHshift[1](#2,#3)(#4,#5){stealth-}%
\or
\electronAHshift[0](#2,#3)(#4,#5){stealth-}%
\fi\fi}
\def\futuresubst#1{\kern140\unitlength #1}
%    \end{macrocode}
%  \end{macro}
% 
% Arrows having a harpoon-type head are defined as follows. 
% Keywords for differentiation are: 
% Hru (harpoon right upward), Hrd (harpoon right downward), 
% Hlu (harpoon left upward), and Hld (harpoon left downward).
%
%  \begin{macro}{\electronHrushiftarrow}
% \changes{v5.00}{2010/10/01}{Redefined for PDF mode}
%    \begin{macrocode}
\def\electronHrushiftarrow{%
\@ifnextchar[{\@electronHrushiftarrow}{\@electronHrushiftarrow[0]}}
\def\@electronHrushiftarrow[#1](#2,#3)(#4,#5){%
\ifnum#2<#4\relax
\electronAHshift[#1](#2,#3)(#4,#5){-left to}%
\else 
\ifcase#1\relax
\electronAHshift[1](#2,#3)(#4,#5){-left to}%
\or
\electronAHshift[0](#2,#3)(#4,#5){-left to}%
\fi\fi}
%    \end{macrocode}
%  \end{macro}

%  \begin{macro}{\electronHrdshiftarrow}
% \changes{v5.00}{2010/10/01}{Redefined for PDF mode}
%    \begin{macrocode}
\def\electronHrdshiftarrow{%
\@ifnextchar[{\@electronHrdshiftarrow}{\@electronHrdshiftarrow[0]}}
\def\@electronHrdshiftarrow[#1](#2,#3)(#4,#5){%
\ifnum#2<#4\relax
\electronAHshift[#1](#2,#3)(#4,#5){-right to}%
\else 
\ifcase#1\relax
\electronAHshift[1](#2,#3)(#4,#5){-right to}%
\or
\electronAHshift[0](#2,#3)(#4,#5){-right to}%
\fi\fi}
%    \end{macrocode}
%  \end{macro}
%
%  \begin{macro}{\electronHlushiftarrow}
% \changes{v5.00}{2010/10/01}{Redefined for PDF mode}
%    \begin{macrocode}
\def\electronHlushiftarrow{%
\@ifnextchar[{\@electronHlushiftarrow}{\@electronHlushiftarrow[0]}}
\def\@electronHlushiftarrow[#1](#2,#3)(#4,#5){%
\ifnum#2<#4\relax
\electronAHshift[#1](#2,#3)(#4,#5){right to-}%
\else 
\ifcase#1\relax
\electronAHshift[1](#2,#3)(#4,#5){right to-}%
\or
\electronAHshift[0](#2,#3)(#4,#5){right to-}%
\fi\fi}
%    \end{macrocode}
%  \end{macro}

%  \begin{macro}{\electronHrdshiftarrow}
% \changes{v5.00}{2010/10/01}{Redefined for PDF mode}
%    \begin{macrocode}
\def\electronHldshiftarrow{%
\@ifnextchar[{\@electronHldshiftarrow}{\@electronHldshiftarrow[0]}}
\def\@electronHldshiftarrow[#1](#2,#3)(#4,#5){%
\ifnum#2<#4\relax
\electronAHshift[#1](#2,#3)(#4,#5){left to-}%
\else 
\ifcase#1\relax
\electronAHshift[1](#2,#3)(#4,#5){left to-}%
\or
\electronAHshift[0](#2,#3)(#4,#5){left to-}%
\fi\fi}
%    \end{macrocode}
%  \end{macro}
% 
% \section{Initial Setting and Switch to PDF (pgf package) Arrows}
%
%  \begin{macro}{\chmstpdfsw}
%    \begin{macrocode}
\def\chmstpdfsw{%setting of the chmst-pdf package
\let\reactrarrow=\newreactrarrow%
\let\reactlarrow=\newreactlarrow
\let\reactlrarrow=\newreactlrarrow
\let\reactEqarrow=\newreactEqarrow
\let\reacteqarrow=\newreacteqarrow
%%%%
\let\reactREqarrow=\newreactREqarrow%
\let\reactLEqarrow=\newreactLEqarrow
\let\reactreqarrow=\newreactreqarrow
\let\reactleqarrow=\newreactleqarrow
%%%%
\let\reactdarrow=\newreactdarrow%
\let\reactuarrow=\newreactuarrow
\let\reactduarrow=\newreactduarrow
\let\reactVEqarrow=\newreactVEqarrow
\let\reactveqarrow=\newreactveqarrow
%%%%
\let\reactnearrow=\newreactnearrow%
\let\reactswarrow=\newreactswarrow
%%%%
\let\reactsearrow=\newreactsearrow%
\let\reactnwarrow=\newreactnwarrow
%%%%
\let\reactDEqarrow=\newreactDEqarrow%
\let\reactUEqarrow=\newreactUEqarrow
\let\reactdeqarrow=\newreactdeqarrow
\let\reactueqarrow=\newreactueqarrow
%%%%
\let\reactulrarrow=\newreactulrarrow%
\let\reactdlrarrow=\newreactdlrarrow
%%%%
\let\schemelarrow=\newschemelarrow%
\let\schemerarrow=\newschemerarrow
\let\schemelrarrow=\newschemelrarrow
%%%%
\let\llongrightarrow=\newllongrightarrow%
\let\llongleftarrow=\newllongleftarrow
\let\llongleftrightarrow=\newllongleftrightarrow
\let\Equilibarrow=\newEquilibarrow
%%%
\let\Llongrightarrow=\newLlongrightarrow%
\let\Llongleftarrow=\newLlongleftarrow
\let\Llongleftrightarrow\newLlongleftrightarrow
%%%
\let\lllongrightarrow=\newlllongrightarrow%
\let\lllongleftarrow=\newlllongleftarrow
\let\lllongleftrightarrow=\newlllongleftrightarrow
\let\Equiliblongarrow=\newEquiliblongarrow
%%%
\let\Lllongrightarrow=\newLllongrightarrow%
\let\Lllongleftarrow=\newLllongleftarrow
\let\Lllongleftrightarrow\newLllongleftrightarrow
%%%
\let\llongrightharpoonup=\newllongrightharpoonup%
\let\llongrightharpoondown=\newllongrightharpoondown
\let\llongleftharpoonup=\newllongleftharpoonup
\let\llongleftharpoondown=\newllongleftharpoondown
%%%
\let\lllongrightharpoonup=\newlllongrightharpoonup%
\let\lllongrightharpoondown=\newlllongrightharpoondown
\let\lllongleftharpoonup=\newlllongleftharpoonup
\let\lllongleftharpoondown\newlllongleftharpoondown
%%%
\let\equilibarrow=\newequilibarrow%
\let\equiliblongarrow=\newequiliblongarrow
%%%
}
\let\chmstpspdfsw=\chmstpdfsw%compatibility to postscript mode
%    \end{macrocode}
%  \end{macro}
%
%    \begin{macrocode}
\chmstpdfsw%initial setting
%</chmstpdf>
%    \end{macrocode}
%
% \Finale
\endinput
