% -*- coding: utf-8; time-stamp-format: "%02d-%02m-%:y at %02H:%02M:%02S %Z" -*-
%<*none>
\def\dtxtimestamp {Time-stamp: <07-11-2019 at 17:09:56 CET>}
%</none>
%<*!readme>
%%
%% Package: footnotehyper
%% Version: 1.1a (2019/11/07)
%% License: LPPL 1.3c
%% Copyright (C) 2016-2019 Jean-Francois Burnol <jfbu at free dot fr>.
%%
%</!readme>
%<*tex>
\def\pkgname        {footnotehyper}
\def\pkgdate        {2019/11/07}
\def\docdate        {2019/11/07}
\def\pkgversion     {v1.1a}
\def\pkgdescription {hyperref aware footnote.sty (JFB)}
%</tex>
%<*none>
% Definition of \pkgLicense
\begingroup% cette méthode ne marcherait pas avec caractères en dehors de 32-127
    \long\def\firstofone #1{#1}\catcode1=14\catcode2=0
    \catcode`\%=12\catcode`\_=12\endlinechar13\catcode13=13 ^^A
    \catcode32=13\catcode`\\=12^^Brelax^^A
^^Bfirstofone{^^Bendgroup^^Bdef^^BpkgLicense^^A
{% Package: footnotehyper
% Version: 1.1a (2019/11/07)
% License: LPPL 1.3c
% Copyright (C) 2016-2019 Jean-Francois Burnol <jfbu at free dot fr>.
%
% This Work may be distributed and/or modified under the conditions
% of the LaTeX Project Public License, version 1.3c. This version of
% this license is in:
%
% > <http://www.latex-project.org/lppl/lppl-1-3c.txt>
%
% and the latest version of this license is in:
%
% > <http://www.latex-project.org/lppl.txt>
%
% Version 1.3 or later is part of all distributions of
% LaTeX version 2005/12/01 or later.
%
% The Author of this Work is: Jean-Francois Burnol `<jfbu at free dot fr>`
%
% This Work consists of the main source file footnotehyper.dtx and the
% derived files footnotehyper.sty, footnotehyper.ins, footnotehyper.tex,
% footnotehyper.pdf, footnotehyper.dvi.
}}%
\begingroup\catcode1 0 \catcode`\\ 12 
^^Aiffalse
%</none>
%<*readme>
<!-- -->

    Source:  footnotehyper.dtx (v1.1a 2019/11/07)
    Author:  Jean-Francois Burnol
    Info:    hyperref aware footnote.sty
    License: LPPL 1.3c
    Copyright (C) 2016-2019 Jean-Francois Burnol <jfbu at free dot fr>.


ABSTRACT
========

The `footnote` package by Mark Wooding (`1997/01/28` `1.13`)
allows to gather footnotes (`\begin{savenotes}`) and later insert
them (after `\end{savenotes}`) at the bottom of the page, even
if the intervening material consists of tabulars, minipages or
framed contents for example. One can also use the
`\savenotes/\spewnotes` syntax.

Also, `footnote.sty` provides a `footnote` environment which
allows to insert verbatim material.

Earlier releases of the present `footnotehyper` package added
patches for `hyperref` compatibility and some bugfixes, addressing
in particular the incompatibility with `color/xcolor`, and with
`babel-frenchb`, and also fixing the `footnote` environment with
optional argument `[NUM]`. Since `v0.99` all macros are defined
internally and the `footnote` package is not loaded at all.

The same user interface is kept. Since `v1.0` it is possible to
use `footnotehyper` also in absence of `hyperref` or when the latter is
loaded with its `hyperfootnotes=false` option. The order of loading of
`footnotehyper` and `hyperref` is inconsequential.

INSTALLATION
============

The recommended way is to first extract the package (.sty)
and driver (.tex) files from footnotehyper.dtx via

    tex footnotehyper.dtx

and then produce the documentation via

    latex footnotehyper.tex (twice)
    dvipdfmx footnotehyper.dvi

It is also possible to produce simultaneously the package
and the documentation via one of these two routes:

    pdflatex footnotehyper.dtx (twice)

or

    latex footnotehyper.dtx (twice)
    dvips
    ps2pdf

The method using the extracted file footnotehyper.tex produces
the smallest pdf file and is the officially preferred one as
it allows to set options in footnotehyper.tex to customize the
footnotehyper.pdf file:

- scrdoc class options (paper size, font size, ...)
- with or without source code,
- with dvipdfmx or with latex+dvips or pdflatex.

Installation:

    footnotehyper.sty    -> TDS:tex/latex/footnotehyper/footnotehyper.sty
    footnotehyper.dtx    -> TDS:source/latex/footnotehyper/footnotehyper.dtx
    footnotehyper.pdf    -> TDS:doc/latex/footnotehyper/footnotehyper.pdf
    README.md            -> TDS:doc/latex/footnotehyper/README.md

The other files may be discarded.


LICENSE
=======

This Work may be distributed and/or modified under the conditions
of the LaTeX Project Public License, version 1.3c. This version of
this license is in:

> <http://www.latex-project.org/lppl/lppl-1-3c.txt>

and the latest version of this license is in:

> <http://www.latex-project.org/lppl.txt>

Version 1.3 or later is part of all distributions of
LaTeX version 2005/12/01 or later.

The Author of this Work is:

- Jean-Francois Burnol `<jfbu at free dot fr>`

This Work consists of the main source file footnotehyper.dtx
and the derived files footnotehyper.sty, footnotehyper.tex,
footnotehyper.pdf, footnotehyper.dvi.
%</readme>
%<*tex>-------------------------------------------------------------------------
%%
%% run latex twice on this file footnotehyper.tex then dvipdfmx on
%% footnotehyper.dvi to produce the documentation footnotehyper.pdf, with
%% source code included.
%%
\chardef\Withdvipdfmx 1 % replace 1 by 0 for using latex+dvips or pdflatex
\chardef\NoSourceCode 0 % replace 0 by 1 for the doc *without* the source code
\NeedsTeXFormat{LaTeX2e}
\ProvidesFile {\pkgname.tex}[Driver for \pkgname\space documentation]%
\PassOptionsToClass   {a4paper,fontsize=11pt,oneside}{scrdoc}
\PassOptionsToPackage {english}{babel}
\input \pkgname.dtx
%%% Local Variables:
%%% mode: latex
%%% End:
%</tex>-------------------------------------------------------------------------
%<*none>------------------------------------------------------------------------
^^Afi^^Aendgroup
%
\chardef\noetex 0
\ifx\numexpr\undefined\chardef\noetex 1 \fi
\ifnum\noetex=1 \chardef\extractfiles 0 % extract files, then stop
\else
    \ifx\ProvidesFile\undefined
      \chardef\extractfiles 0 % etex etc.. on \pkgname.dtx, only file extraction.
    \else % latex/pdflatex on \pkgname.tex or on \pkgname.dtx
      \ifx\Withdvipdfmx\undefined
        % latex/pdflatex run is on \pkgname.dtx
        \chardef\extractfiles 1 % 1 = extract files and typeset manual, 2 = only typeset
        \chardef\Withdvipdfmx 0 % 0 = pdflatex or latex+dvips, 1 = dvipdfmx
        \chardef\NoSourceCode 0 % 0 =  include source code, 1 = do not
        \NeedsTeXFormat {LaTeX2e}%
        \PassOptionsToClass   {a4paper,fontsize=11pt,oneside}{scrdoc}%
        \PassOptionsToPackage {english}{babel}%
      \else % latex run is on \pkgname.tex
        \chardef\extractfiles 2 % do not extract files, only typeset
      \fi
      \ProvidesFile{\pkgname.dtx}%
        [\pkgname\space source and documentation (\dtxtimestamp)]%
    \fi
\fi
\ifnum\extractfiles<2 % extract files
\def\MessageDeFin{\newlinechar10 \let\Msg\message
\Msg{********************************************************************^^J}%
\Msg{*^^J}%
\Msg{* To finish the installation you have to move the following^^J}%
\Msg{* file into a directory searched by TeX:^^J}%
\Msg{*^^J}%
\Msg{*\space\space\space\space \pkgname.sty^^J}%
\Msg{*^^J}%
\Msg{* To produce the documentation with source code included run latex^^J}%
\Msg{* twice on file \pkgname.tex and then dvipdfmx on \pkgname.dvi^^J}%
\Msg{*^^J}%
\Msg{* Happy TeXing!^^J}%
\Msg{*^^J}%
\Msg{********************************************************************^^J}%
}%
\begingroup
    \input docstrip.tex
    \askforoverwritefalse
    \def\pkgpreamble{\defaultpreamble^^J\MetaPrefix^^J%
    \string\NeedsTeXFormat{LaTeX2e}^^J%
    \string\ProvidesPackage{\pkgname}\perCent^^J%
    \space[\pkgdate\space\pkgversion\space\pkgdescription]}%
    \generate{\nopreamble\nopostamble
    \file{README.md}{\from{\pkgname.dtx}{readme}}%
    \usepostamble\defaultpostamble
    \file{\pkgname.tex}{\from{\pkgname.dtx}{tex}}%
    \usepreamble\pkgpreamble
    \file{\pkgname.sty}{\from{\pkgname.dtx}{package}}}%
\endgroup
\fi % end of file extraction (from etex/latex/pdflatex run on \pkgname.dtx)
\ifnum\extractfiles=0 % tex/etex/xetex/etc on \pkgname.dtx, files extracted, stop
      \MessageDeFin\expandafter\end
\fi
% From this point on, run is necessarily with e-TeX.
% Check if \MessageDeFin got defined, if yes put it at end of run.
\ifdefined\MessageDeFin\AtEndDocument{\MessageDeFin}\fi
%-------------------------------------------------------------------------------
% START OF USER MANUAL TEX SOURCE
\documentclass[abstract]{scrdoc}

\ifnum\NoSourceCode=1 \OnlyDescription\fi

\usepackage{ifpdf}
\ifpdf\chardef\Withdvipdfmx 0 \fi

\makeatletter
\ifnum\Withdvipdfmx=1
   \@for\@tempa:=hyperref,bookmark,graphicx,xcolor,pict2e\do
            {\PassOptionsToPackage{dvipdfmx}\@tempa}
   %
   \PassOptionsToPackage{dvipdfm}{geometry}
   \PassOptionsToPackage{bookmarks=true}{hyperref}
   \PassOptionsToPackage{dvipdfmx-outline-open}{hyperref}
   \PassOptionsToPackage{dvipdfmx-outline-open}{bookmark}
   %
   \def\pgfsysdriver{pgfsys-dvipdfm.def}
\else
   \PassOptionsToPackage{bookmarks=true}{hyperref}
\fi
\makeatother

\usepackage[T1]{fontenc}
\usepackage[utf8]{inputenc}
\usepackage {babel}

\usepackage[hscale=0.72,vscale=0.7]{geometry}

\def\MacroFont{\ttfamily\small\hyphenchar\font45 \baselineskip11pt\relax}

\usepackage{amsmath}
\usepackage{newtxtext, newtxmath}
\usepackage[straightquotes, scaled=0.97]{newtxtt}
\usepackage{xspace}

\usepackage[dvipsnames]{xcolor}
\definecolor{joli}{RGB}{225,95,0}
\definecolor{JOLI}{RGB}{225,95,0}
\definecolor{BLUE}{RGB}{0,0,255}
\colorlet{niceone}{green!35!blue!75}

\usepackage{framed}

\usepackage[pdfencoding=pdfdoc]{hyperref}
\hypersetup{%
linktoc=all,%
breaklinks=true,%
colorlinks,%
linkcolor=RoyalBlue,%
urlcolor=OliveGreen,%
pdfauthor={Jean-Fran\c cois Burnol},%
pdftitle={The \pkgname\space package},%
pdfsubject={\pkgdescription},%
pdfkeywords={LaTeX, footnotes},%
pdfstartview=FitH,%
pdfpagemode=UseNone}
% added usage of package bookmark 2013/10/10
\usepackage{bookmark}
\usepackage{enumitem}
\usepackage{\pkgname}

\newcommand\fnh{%
        \texorpdfstring{{\color{joli}\ttfamily\bfseries \pkgname}}{\pkgname}\xspace}

\frenchspacing

\renewcommand\familydefault\sfdefault
\pagestyle{headings}

\begin{document}\thispagestyle{empty}
% the \MacroFont valid at begin document will be the one used in Implementation
% the one defined here will be used by verbatim (not by \verb)
\def\MacroFont{\ttfamily\small}
\rmfamily
\bookmark[named=FirstPage,level=1]{Title page}

\begin{center}
  {\normalfont\LARGE The \fnh  package}\\
\textsc{\small Jean-François Burnol}\par
  \footnotesize \ttfamily
  jfbu (at) free (dot) fr\par
  Package version: \pkgversion\ (\pkgdate)\par
  From source file \texttt{\pkgname.dtx} of \dtxtimestamp.\par
\end{center}

\MakeShortVerb{\`}%
\begin{abstract}
The `footnote` package by \textsc{Mark~Wooding} (`1997/01/28` `1.13`)
allows to gather footnotes (`\begin{savenotes}`) and later insert
them (after `\end{savenotes}`) at the bottom of the page, even
if the intervening material consists of tabulars, minipages or
framed contents for example. One can also use the
`\savenotes/\spewnotes` syntax.

Also, `footnote.sty` provides a `footnote` environment which
allows to insert verbatim material.

Earlier releases of the present \fnh package added
patches for `hyperref` compatibility and some bugfixes, addressing
in particular the incompatibility with `color/xcolor`, and with
`babel-frenchb`, and also fixing the `footnote` environment with
optional argument `[NUM]`. Since `v0.99` all macros are defined
internally and the `footnote` package is not loaded at all.

The same user interface is kept. Since `v1.0` it is possible to
use \fnh also in absence of `hyperref` or when the latter is
loaded with its `hyperfootnotes=false` option. The order of loading of
\fnh and `hyperref` is inconsequential.
\end{abstract}
\DeleteShortVerb{\`}

\section{License}

\begingroup\ttfamily\footnotesize\hyphenchar\font -1
           \parindent0pt 
           \obeyspaces\obeylines %
\pkgLicense\endgroup

\section{Usage}

Since |v0.99|, \fnh does not
load package
|footnote.sty|\footnote{\url{http://ctan.org/pkg/footnote}}
anymore, and is even incompatible with it at it uses the same user
interface.

Since |v1.0|, \fnh is usable also in absence of |hyperref| or when
the latter was passed |hyperfootnotes=false| option.

Version |v1.1| fixes a bug which arose when |savenotes| environment was used
\emph{inside} a minipage: footnotes were disappearing!%
%
\footnote{Thanks to François Pantigny for reporting the bug.
  A later suggestion of the same is to let the package do nothing under
  Beamer class, and this is what |v1.1a 2019/11/07| does.}
%
See related remarks at end of \autoref{sec:notes}.

The package thus provides:
\begin{itemize}
\item a |savenotes| environment which re-routes footnotes and delivers them at
  the end (there is also the |\savenotes/\spewnotes| syntax; it does create a
  group like the environment),
\item |footnote| and |footnotetext| environments to allow footnotes
  with verbatim material.
\end{itemize}

But the constructability of the |footnote/footnotetext| environments depends
on how |\@makefntext| has been customized by the class or other packages; a
warning is issued if the situation appears to be desperate.\footnote{original
  |footnote.sty| could end up in a low level \TeX\ error.}

It should be recalled that in case of |\footnotemark[N]| and
|\footnotetext[N]{...}| mark-up |hyperref| creates no hyperlink. This is not
changed by \fnh and applies also to the |\begin{footnotetext}[N]| case.
Without optional argument the link is created, and the link is created also
for |\footnote[N]| or |\begin{footnote}[N]|.

This package does not handle especially floating environments, except that one
can always surround them in the source in a |savenotes| environment and one
knows that the footnotes will be delivered at the |\end{savenotes}|\dots which
may well be one page earlier than the actual location of the floating material
in the produced document !

Environments typesetting multiple times their contents are the most hostile to
footnotes. Currently, \fnh only handles especially the |amsmath|
environments (as in |footnote.sty|.)

Finally there is a |\makesavenoteenv| command which takes as
argument an environment name and patches it to do the
|\savenotes/\spewnotes| automatically.%
\begin{footnote}
  For the
record the syntax is either |\makesavenoteenv{foo}| which patches
environment |foo| or |\makesavenoteenv[bar]{foo}| which defines
environment |bar| as this patched version of |foo|.
\end{footnote}
It is safer to avoid it, as
one never knows what happens with such patches: for example the |[H]|
specifier provided by the |float| package overwrites the |\end{table}|
definition during the execution of |\begin{table}|\dots !

\begin{savenotes}
\begin{framed}
\DeleteShortVerb{\|}\MakeShortVerb{\*}%
{\centering
  \begin{tabular}{|c|c|}
    \hline
    \strut Inside\footnote{Notice that as  frame
      extended to the next page,
      the end of the *savenotes* environment delivered its
      footnotes only here.} a & tabular\begin{footnote}Alternatively a
      *\savenotes/\spewnotes* pair
      could have been used.\end{footnote}\\\hline
\end{tabular}\par}
\DeleteShortVerb{\*}\MakeShortVerb{\|}%
An example:
\begin{verbatim}
\begin{savenotes}
\begin{framed}
  Please refer to the documentation of the |footnote| package.%
  \footnote{\url{http://ctan.org/pkg/footnote}}

  Particularly you may check its |savenotes| environment.%
  \footnote{It doesn't bring any
  feature to especially handle the issues related to footnotes in floating
  environments, though.}
\end{framed}
\end{savenotes}
\end{verbatim}
and the present frame has \cs{footnote}'s from inside a |tabular| and is
inside a |savenotes| environment.%
\begin{footnote}
  Here is an issue which has nothing (as I finally figured out) to
  do with |footnote|, and only indirectly with \LaTeX: if you embed
  a \emph{full-width} minipage (with initial \cs{noindent}) in any
  environment not doing \cs{ignorespacesafterend}, be careful to add
  a \% either immediately after the |\end{minipage}| (or a |\relax|
  or a |\par|) or after the surrounding environment |\end{foo}| or
  use |\end{minipage}\end{foo}| else the output may have an extra
  blank line if the source has a blank line after the |foo| environment.
  Here is a typical example, with a |tabular| rather:
  \makeatletter

  \medskip\noindent\hspace{1cm}\begin{minipage}{\dimexpr\linewidth-1cm}%
  \def\MacroFont{\baselineskip10pt\relax}%
\begin{verbatim}
\newenvironment{foo}{}{}

\noindent\begin{tabular}{p{\dimexpr\linewidth-2\tabcolsep\relax}}
  A\dotfill B
\end{tabular}

C

\begin{foo}
\noindent\begin{tabular}{p{\dimexpr\linewidth-2\tabcolsep\relax}}
  A\dotfill B
\end{tabular}
\end{foo}

C
\end{verbatim}\end{minipage}\par\medskip

  If you try it out you will see an extra blank line in PDF output above the
  second C. Starting with |v0.99| the |\end{savenotes}| emits an
  \cs{ignorespacesafterend} which avoids this generic \TeX/\LaTeX\
  problem. For good measure there is now an \cs{ignorespaces} in
  |\begin{savenotes}|.
\end{footnote}
Let's test an
|amsmath| environment with |\intertext|. As
\begin{align}
  E&=mc^2\;,
\intertext{was too easy\footnote{There is also $E=h\nu$.}, let's
  try:}
  a^n + b^n &=c^n\;.
\end{align}
And a footnote with some verbatim material%
\begin{footnote}
  \verb|&$^%\[}$|
\end{footnote}.
\end{framed}
\end{savenotes}

The last one was coded as:
\begin{verbatim}
And a footnote with some verbatim material%
\begin{footnote}
  \verb|&$^%\[}$|
\end{footnote}.
\end{verbatim}
Now some use of |\footnotemark|\footnotemark\ followed by a |footnotetext|
environment.
\begin{footnotetext}
  This one uses the normal footnote counter and the hyperlink works.
\end{footnotetext}
And use of |\footnotemark[99]|\footnotemark[99] in association with a
|footnotetext| environment using the same optional argument |[99]|.
No hyperfootnote link was inserted.%
\begin{footnotetext}[99]
  |hyperref| creates no hyperlink in this case, or in the
  |\footnotemark[N]/\footnotetext[N]{<foo>}| case. It does when
  the |[N]| is absent or when it is used with a |\footnote| command
  (or a |footnote| environment.)

  By the way, \fnh |v0.9f|'s |\footnotetext[N]| and |\begin{footnotetext}[N]|
  each had a bug and they were unusable \emph{inside} the |savenotes|
  environment. There was no issue \emph{outside}.
\end{footnotetext}
And a final footnote, done with |\begin{footnote}[57]|%
\begin{footnote}[57]\csname @minipagetrue\endcsname % to reduce
                                % framed inserted vertical spacing
  \begin{framed}
    \noindent\fnh works since |v1.0| also in absence of |hyperref| or when the
    latter was passed |hyperfootnotes=false| option.
  \end{framed}
\end{footnote}.
There is no problem with the hyperlink, then.

\section{Notes}\label{sec:notes}

A few items worth of mention:
\begin{itemize}[nosep]
\item the |footnote| package patches the \LaTeX\ kernel |\parbox|.
  \fnh doesn't (but the code can be found commented-out at the
  end of the present file).
\item the |footnote| package defines a |minipage*| environment
  which is |minipage| patched by |\makesavenoteenv|, \fnh doesn't.
\item the |footnote| environment from |footnote.sty| does a
  |\leavevmode\unskip| which \fnh doesn't: hence if one locates
  |\begin{footnote}| at start of a line in the \LaTeX\ source, one will
    typically need a \% at end of text on previous line to avoid the
    end-of-line space.
\item the |hyperref| package inserts no hyperlink in case of
  |\footnotemark[N]/\footnotetext[N]|. This is not modified by \fnh.
\item side-note: there is an interference between |hyperref| and |frenchb|
  regarding the footnote marker when using the syntax |\footnotemark[NUM]|.
  For the record here is a patch (last tested briefly with
% PUTAIN BORDEL ! LaTeX Error: Text for \verb command ended by end of
% line. À CHAQUE FOIS ÇA M'AGAÇE. Bon, je n'ai pas mon patch habituel ici.
  |hyperref 2016/06/24 v6.83q| and |frenchb 2017/01/30 v3.2g|):
% pour le cas où mis dans commentaires de code
% \begingroup\makeatletter\let\check@percent\@gobble
\begin{verbatim}
  \AtBeginDocument{%
   \let\@xfootnotemarkORIFB \@xfootnotemark
   \def\@xfootnotemarkFB {\leavevmode\unskip\unkern\,\@xfootnotemarkORIFB }%
   \ifHy@hyperfootnotes\ifFBAutoSpaceFootnotes
       \let\@xfootnotemark\@xfootnotemarkFB
   \fi\fi
  }%
\end{verbatim}
% \endgroup
\item some environments typeset multiple times their contents,
  which causes issues; \fnh takes provisions only to handle the
  |amsmath| measuring step.
\item
  \LaTeX2e has some ``features'' when using footnotes in |minipage|'s
  which are themselves in a |minipage| which may also have footnotes externally
  to the internal minipages... try it out with some |\fbox|es around the
  sub-|minipages|, to see.

  \fnh behaves like original package |footnote| when the |savenotes|
  environment is used \emph{inside} a minipage. Only reasonable usage in case
  of nested minipages seems to use only a single top level (i.e. external)
  |savenotes| environment. But there will anyhow be collisions of the
  alphabetic enumerations. These collisions are there with or without \fnh (or
  |footnote.sty|.) I did not make any attempt, nor intend to in future, to
  address in an automatized manner these problematic contexts.
\end{itemize}

\StopEventually{\end{document}\endinput}
\makeatletter
    \let\check@percent\original@check@percent
\makeatother

\newgeometry{left=4cm, hscale=0.72}
\section{Implementation}

\small

\makeatletter
\noindent
\begingroup
\topsep\MacrocodeTopsep
\trivlist\parskip\z@\item[]
\macro@font
\leftskip\@totalleftmargin  \advance\leftskip\MacroIndent
\rightskip\z@  \parindent\z@  \parfillskip\@flushglue
\global\@newlistfalse \global\@minipagefalse
\ifcodeline@index
  \everypar{\global\advance\c@CodelineNo\@ne
  \llap{\theCodelineNo\ \hskip\@totalleftmargin}}%
\fi
\string\NeedsTeXFormat\string{LaTeX2e\string}\par
\string\ProvidesPackage\string{\pkgname\string}\@percentchar\par
\noindent\space [\pkgdate\space\pkgversion\space\pkgdescription]\par
\nointerlineskip
\global\@inlabelfalse
\endtrivlist
\endgroup
\makeatother

% The catcode hackery next is to avoid to have <*package> to be listed
% in the commented source code...
% (c) 2012/11/19 jf burnol ;-)

\MakePercentIgnore

%
% \catcode`\<=0 \catcode`\>=11 \catcode`\*=11 \catcode`\/=11
% \let</none>\relax
% \def<*package>{\catcode`\<=12 \catcode`\>=12 \catcode`\*=12 \catcode`\/=12}
%
%</none>
%<*package>
% \begin{macro}{no options}
% The package has no options.
%    \begin{macrocode}
\DeclareOption*%
    {\PackageWarning{footnotehyper}{Option `\CurrentOption' is unknown}}%
\ProcessOptions\relax
%    \end{macrocode}
% \end{macro}
% |v1.1a| lets the package abort under Beamer class and warn user.
%    \begin{macrocode}
\@ifclassloaded{beamer}
    {\PackageWarningNoLine{footnotehyper}{This package is
        incompatible with the beamer class. Aborting input..}%
     \endinput}
    {}%
%    \end{macrocode}
% Versions up to |v0.9f| loaded |footnote.sty|, with lots of patching
% afterwards. Starting with |v0.99|, \fnh does everything by itself with |FNH@|
% prefix. Brief overview of some of the fixed issues:
% \begin{itemize}[nosep]
% \item there was incompatibility with |hyperref|,
% \item and with |color|,
% \item if the \cs{@makefntext} at the time of loading of |footnote.sty|
%   does not have its argument visible at top level in its meaning, or
%   is used multiple times there, then the footnote environment will lead to low
%   level \TeX\ error,
% \item |footnote.sty| modifies |\parbox|,
% \item |footnote.sty| does some too early |\let|,
% \item the footnote environment from |footnote.sty| does not work if used with
% optional argument |[N]|.
% \end{itemize}
%
% \noindent Starting with |v1.0|, \fnh may be used also in absence
% of |hyperref|.
%    \begin{macrocode}
\newbox\FNH@notes
\newdimen\FNH@width
\let\FNH@colwidth\columnwidth
\newif\ifFNH@savingnotes
\AtBeginDocument {%
    \let\FNH@latex@footnote    \footnote
    \let\FNH@latex@footnotetext\footnotetext
    \let\FNH@H@@footnotetext   \@footnotetext
    \let\FNH@H@@mpfootnotetext \@mpfootnotetext
    \newenvironment{savenotes}
        {\FNH@savenotes\ignorespaces}{\FNH@spewnotes\ignorespacesafterend}%
    \let\spewnotes      \FNH@spewnotes
    \let\footnote       \FNH@footnote
    \let\footnotetext   \FNH@footnotetext
    \let\endfootnote    \FNH@endfntext
    \let\endfootnotetext\FNH@endfntext
    \@ifpackageloaded{hyperref}
     {\ifHy@hyperfootnotes
         \let\FNH@H@@footnotetext\H@@footnotetext
         \let\FNH@H@@mpfootnotetext\H@@mpfootnotetext
      \else
         \let\FNH@hyper@fntext\FNH@nohyp@fntext
      \fi}%
     {\let\FNH@hyper@fntext\FNH@nohyp@fntext}%
}%
%    \end{macrocode}
% \begin{macro}{\FNH@hyper@fntext}
% \begin{macro}{\FNH@nohyp@fntext}
% \begin{macro}{\FNH@fntext}
% These are the \fnh replacement for |\@footnotetext| inside the
% |savenotes| environment. There is a version creating an hyperlink
% and another one not creating an hyperlink. The \cs{FNH@fntext}
% macro serves as general dispatch. This may be a place to customize if one
% wants to handle environments doing multiple passes: but the footnote counter
% must have been taken care of elsewhere. The code currently handles only
% the case of |amsmath| environments.
%    \begin{macrocode}
\def\FNH@hyper@fntext{\FNH@fntext\FNH@hyper@fntext@i}%
\def\FNH@nohyp@fntext{\FNH@fntext\FNH@nohyp@fntext@i}%
\def\FNH@fntext #1{\ifx\ifmeasuring@\@undefined
    \expandafter\@secondoftwo\else\expandafter\@firstofone\fi
    {\ifmeasuring@\expandafter\@gobbletwo\fi}#1%
}%
%    \end{macrocode}
% \end{macro}\end{macro}\end{macro}
% \begin{macro}{\FNH@hyper@fntext@i}
% We do the \cs{ifHy@nesting} test although hyperref's manual
% says ``Allows links to be nested; no drivers currently support this.''
%    \begin{macrocode}
\long\def\FNH@hyper@fntext@i#1{%
  \global\setbox\FNH@notes\vbox
  {\unvbox\FNH@notes
   \FNH@startnote
   \@makefntext
    {\rule\z@\footnotesep\ignorespaces
     \ifHy@nesting\expandafter\ltx@firstoftwo
             \else\expandafter\ltx@secondoftwo
     \fi
     {\expandafter\hyper@@anchor\expandafter{\Hy@footnote@currentHref}{#1}}%
     {\Hy@raisedlink
       {\expandafter\hyper@@anchor\expandafter{\Hy@footnote@currentHref}%
       {\relax}}%
      \let\@currentHref\Hy@footnote@currentHref
      \let\@currentlabelname\@empty
      #1}%
     \@finalstrut\strutbox
    }%
   \FNH@endnote
  }%
}%
%    \end{macrocode}
% \end{macro}
% \begin{macro}{\FNH@nohyp@fntext@i}
% The original \cs{fn@fntext} had no \cs{long}.
%    \begin{macrocode}
\long\def\FNH@nohyp@fntext@i#1{%
  \global\setbox\FNH@notes\vbox
  {\unvbox\FNH@notes
   \FNH@startnote
   \@makefntext{\rule\z@\footnotesep\ignorespaces#1\@finalstrut\strutbox}%
   \FNH@endnote
  }%
}%
%    \end{macrocode}
% \end{macro}
% \begin{macro}{\FNH@startnote}
% Same as original (the code comment is kept from original.)
%    \begin{macrocode}
\def\FNH@startnote{%
  \hsize\FNH@colwidth
  \interlinepenalty\interfootnotelinepenalty
  \reset@font\footnotesize
  \floatingpenalty\@MM% Is this right???
  \@parboxrestore
  \protected@edef\@currentlabel{\csname p@\@mpfn\endcsname\@thefnmark}%
  \color@begingroup
}%
%    \end{macrocode}
% \end{macro}
% \begin{macro}{\FNH@endnote}
% Fixed from original.
%    \begin{macrocode}
\def\FNH@endnote{\color@endgroup}%
%    \end{macrocode}
% \end{macro}
% \begin{macro}{\FNH@savenotes}
% Same as original apart from using |hyperref|-aware \cs{FNH@hyper@fntext},
% and taking into account |hyperref|'s custom \cs{@xfootnotenext}. This was
% missed by |v0.9f| hence |\footnotetext[N]{..}| did not work
% inside |savenotes| environment. Fixed for |v0.99|.
% 
% Maybe I should change the way \cs{@minipagerestore} is handled.
%    \begin{macrocode}
\def\FNH@savenotes{%
  \begingroup
  \ifFNH@savingnotes\else
    \FNH@savingnotestrue
    \let\@footnotetext    \FNH@hyper@fntext
    \let\@mpfootnotetext  \FNH@hyper@fntext
    \let\H@@mpfootnotetext\FNH@nohyp@fntext % fool hyperref's \@xfootnotenext
    \FNH@width\columnwidth
    \let\FNH@colwidth\FNH@width
    \global\setbox\FNH@notes\box\voidb@x
    \let\FNH@thempfn\thempfn
    \let\FNH@mpfn\@mpfn
    \ifx\@minipagerestore\relax\let\@minipagerestore\@empty\fi
    \expandafter\def\expandafter\@minipagerestore\expandafter{%
      \@minipagerestore
      \let\thempfn\FNH@thempfn
      \let\@mpfn\FNH@mpfn
    }%
  \fi
}%
%    \end{macrocode}
% \end{macro}
% \begin{macro}{\FNH@spewnotes}
% This uses \cs{FNH@H@@footnotetext} which is the
% \cs{H@@footnotetext} |hyperref|'s preserved original
% meaning of \cs{@footnotetext} not creating a link target.
%
% |v1.1| fixes the bug about disappearing footnotes if |savenotes| environment
% is used inside a minipage. I had never really considered such usage, hence
% missed realizing there was a bug.
%    \begin{macrocode}
\def\FNH@spewnotes {%
  \endgroup
  \ifFNH@savingnotes\else
   \ifvoid\FNH@notes\else
    \begingroup
     \let\@makefntext\@empty
     \let\@finalstrut\@gobble
     \let\rule\@gobbletwo
     \ifx\@footnotetext\@mpfootnotetext
        \expandafter\FNH@H@@mpfootnotetext
     \else
        \expandafter\FNH@H@@footnotetext
     \fi{\unvbox\FNH@notes}%
    \endgroup
   \fi
  \fi
}%
%    \end{macrocode}
% \end{macro}
% \begin{macro}{\FNH@footnote}
% \begin{macro}{\FNH@footnotetext}
%   We now take care of |footnote.sty|'s |footnote| environment. The original
%   \cs{fn@endfntext} is lacking a \cs{fn@endnote}, and this meant that
%   |footnote.sty| was incompatible with |color/xcolor| packages. Also this
%   \cs{fn@endnote} was |\let| to |\color@endgroup| which is wrong.
%
%   Furthermore, independently of presence of the |\color/xcolor| issue, the
%   |footnote.sty|'s |footnote| environment raised an error if used with an
%   optional argument. |v0.9f| addresses this issue.
%
%   The |footnotetext| environment adds a complication, in case of optional
%   argument we should not try to set up a link due to the fact that |hyperref|
%   does not support it for the |\footnotemark[N]/\footnotetext[N]| syntax.
%   And we need to make sure that the |footnote| and |footnotetext| environments
%   obey the |\savenotes/\spewnotes| mechanism.
%
%   To handle all of this we code things completely differently from
%   |footnote.sty|.
% 
%   The |v0.9f| |\begin{footnotetext}[N]| inside |savenotes| tried to create
%   an hyperref target. Fixed for |v0.99|.
%
%   Note: the |footnote.sty| code did a |\leavevmode\unskip| at entrance of
%   |footnote| environment, which \fnh has not kept.
%    \begin{macrocode}
\def\FNH@footnote@envname    {footnote}%
\def\FNH@footnotetext@envname{footnotetext}%
\def\FNH@footnote{%
    \ifx\@currenvir\FNH@footnote@envname
        \expandafter\FNH@footnoteenv
    \else
        \expandafter\FNH@latex@footnote
    \fi
}%
\def\FNH@footnoteenv{%
    \@ifnextchar[%
      \FNH@footnoteenv@i %]
      {\stepcounter\@mpfn
       \protected@xdef\@thefnmark{\thempfn}%
       \@footnotemark
       \def\FNH@endfntext@fntext{\@footnotetext}%
       \FNH@startfntext}%
}%
\def\FNH@footnoteenv@i[#1]{%
    \begingroup
     \csname c@\@mpfn\endcsname #1\relax
     \unrestored@protected@xdef\@thefnmark{\thempfn}%
    \endgroup
    \@footnotemark
    \def\FNH@endfntext@fntext{\@footnotetext}%
    \FNH@startfntext
}%
\def\FNH@footnotetext{%
    \ifx\@currenvir\FNH@footnotetext@envname
        \expandafter\FNH@footnotetextenv
    \else
        \expandafter\FNH@latex@footnotetext
    \fi
}%
\def\FNH@footnotetextenv{%
    \@ifnextchar[%
      \FNH@footnotetextenv@i %]
      {\protected@xdef\@thefnmark{\thempfn}%
       \def\FNH@endfntext@fntext{\@footnotetext}%
       \FNH@startfntext}%
}%
\def\FNH@footnotetextenv@i[#1]{%
    \begingroup
     \csname c@\@mpfn\endcsname #1\relax
     \unrestored@protected@xdef\@thefnmark{\thempfn}%
    \endgroup
    \ifFNH@savingnotes
      \def\FNH@endfntext@fntext{\FNH@nohyp@fntext}%
    \else
      \def\FNH@endfntext@fntext{\FNH@H@@footnotetext}%
    \fi
    \FNH@startfntext
}%
%    \end{macrocode}
% \end{macro}\end{macro}
% \begin{macro}{\FNH@startfntext}
% \begin{macro}{\FNH@endfntext}
% \begin{macro}{\FNH@endfntext@fntext}
% This is used for the environmental form of the footnote environments.
% The use of |\box\z@| originates in |footnote.sty|, should I change that ?
%
% Both of \cs{endfootnote} and \cs{endfootnotetext} are aliases for
% \cs{FNH@endfntext}.
%
% The \cs{FNH@endfntext@fntext} may be \cs{@footnotetext} (which will be
% \cs{FNH@hyper@fntext} in |savenotes| environment), or
% \cs{FNH@H@@footnotetext}, or \cs{FNH@nohyp@fntext} if in |savenotes| scope.
%    \begin{macrocode}
\def\FNH@startfntext{%
  \setbox\z@\vbox\bgroup
    \FNH@startnote
    \FNH@prefntext
    \rule\z@\footnotesep\ignorespaces
}%
\def\FNH@endfntext {%
    \@finalstrut\strutbox
    \FNH@postfntext
    \FNH@endnote
    \egroup
  \begingroup
    \let\@makefntext\@empty\let\@finalstrut\@gobble\let\rule\@gobbletwo
    \FNH@endfntext@fntext {\unvbox\z@}%
  \endgroup
}%
%    \end{macrocode}
% \end{macro}\end{macro}\end{macro}
% \begin{macro}{\@makefntext}
% \begin{macro}{\FNH@prefntext}
% \begin{macro}{\FNH@postfntext}
%   The definitions of |\FNH@prefntext| and |\FNH@postfntext| (which are needed
%   for the |footnote| environment, |\FNH@startfntext| and |\FNH@endfntext|)
%   are extracted from a somewhat daring analysis of |\@makefntext|.
%   Contrarily to |footnote.sty|'s original code (which may result in low level
%   \TeX\ errors when the |footnote| environment is executed)
%   the method here will alert the user if the argument
%   of |\@makefntext| is not visible at top level in its meaning or is used
%   there multiple times. We also insert here some code to handle especially
%   the case of |babel-frenchb|.
%    \begin{macrocode}
\AtBeginDocument{\let\FNH@@makefntext\@makefntext
   \ifx\@makefntextFB\undefined
   \expandafter\@gobble\else\expandafter\@firstofone\fi
   {\ifFBFrenchFootnotes \let\FNH@@makefntext\@makefntextFB \else
                         \let\FNH@@makefntext\@makefntextORI\fi}%
   \expandafter\FNH@check@a\FNH@@makefntext{1.2!3?4,}%
               \FNH@@@1.2!3?4,\FNH@@@\relax
}%
\long\def\FNH@check@a #11.2!3?4,#2\FNH@@@#3{%
    \ifx\relax#3\expandafter\@firstoftwo\else\expandafter\@secondoftwo\fi
    \FNH@bad@makefntext@alert
    {\def\FNH@prefntext{#1}\def\FNH@postfntext{#2}\FNH@check@b}%
}%
\def\FNH@check@b #1\relax{%
    \expandafter\expandafter\expandafter\FNH@check@c
    \expandafter\meaning\expandafter\FNH@prefntext
    \meaning\FNH@postfntext1.2!3?4,\FNH@check@c\relax
}%
\def\FNH@check@c #11.2!3?4,#2#3\relax{%
    \ifx\FNH@check@c#2\expandafter\@gobble\fi\FNH@bad@makefntext@alert
}%
\def\FNH@bad@makefntext@alert{%
    \PackageWarningNoLine{footnotehyper}%
      {^^J The footnote environment will not be fully functional, sorry.^^J
       You may try to email the author this meaning of \string\@makefntext:^^J
       \meaning\@makefntext^^J
       together with the used preamble}%
    \let\FNH@prefntext\@empty\let\FNH@postfntext\@empty
}%
%    \end{macrocode}
% \end{macro}\end{macro}\end{macro}
% \begin{macro}{\makesavenoteenv}
% Same as original. Not recommended. Safer to use explicitely
% |savenotes| environment.
%    \begin{macrocode}
\def\makesavenoteenv{\@ifnextchar[\FNH@msne@ii\FNH@msne@i}%]
\def\FNH@msne@i #1{%
  \expandafter\let\csname FNH$#1\expandafter\endcsname %$
                  \csname #1\endcsname
  \expandafter\let\csname endFNH$#1\expandafter\endcsname %$
                  \csname end#1\endcsname
  \FNH@msne@ii[#1]{FNH$#1}%$
}%
\def\FNH@msne@ii[#1]#2{%
  \expandafter\edef\csname#1\endcsname{%
    \noexpand\savenotes
    \expandafter\noexpand\csname#2\endcsname
  }%
  \expandafter\edef\csname end#1\endcsname{%
    \expandafter\noexpand\csname end#2\endcsname
    \noexpand\expandafter
    \noexpand\spewnotes
    \noexpand\if@endpe\noexpand\@endpetrue\noexpand\fi
  }%
}%
%    \end{macrocode}
% \end{macro}
% Original footnote.sty patches |\parbox|, we don't touch it. Also it defines a
% |minipage*|  environment, we don't do it.
%    \begin{macrocode}
% \makesavenoteenv[minipage*]{minipage}
% \let\fn@parbox\parbox
% \def\parbox{\@ifnextchar[{\fn@parbox@i{}}{\fn@parbox@ii{}}}
% \def\fn@parbox@i#1[#2]{%
%   \@ifnextchar[{\fn@parbox@i{#1[#2]}}{\fn@parbox@ii{#1[#2]}}%
% }
% \long\def\fn@parbox@ii#1#2#3{\savenotes\fn@parbox#1{#2}{#3}\spewnotes}
\endinput
%    \end{macrocode}
% \MakePercentComment
\Finale
%%
%% End of file `footnotehyper.dtx'.
